\subsection{Задание 2}
\subsubsection{Теория}
В этом задании выводится наблюдатель состояния для системы:
\[
        \begin{cases}
                \dot{x} = A x \\
                y = C x \\
                \dot{\hat{x}} = A \hat{x} + L(\hat{y} - y) \\
                \hat{y} = C \hat{x}
        \end{cases} \rightarrow
        \begin{cases}
            \dot{\hat{x}} = A \hat{x} + L(C \hat{x} - y) = (A + LC )\hat{x} - Ly \\
            \dot{e} = (A + LC)e
            
    \end{cases}
\]

Для синтеза наблюдателя подбирается матрица \(\Gamma \in \mathds{R}^{n \times n}\) с желаемыми собственными числами и матрица \(\mathds{Y} \in \mathds{R}^{n \times k}\), такая что пара \((\Gamma, \mathds{Y})\) управляема. После чего:
\[
    \begin{cases}
        \Gamma Q - QA = YC \\
        L = Q^{-1}Y \\
    \end{cases}
\]




\subsubsection{Результаты}
Результаты представленны на рисунках \ref{fig:task2_states_1} - \ref{fig:task2_y_4}. У системы все собственные числа наблюдаемы, поэтому любое можно изменить и система наблюдаема. Стоит отметить, что из-за мод набора 3, ошибка в этом случае не может сойтись к 0 и бесконечно колеблется.


\begin{figure}[ht!]
    \centering
    \includegraphics[width=\textwidth]{src/figs/task2_states_0.jpg}
    \caption{Результаты моделирования состояний системы и наблюдателя для первого набора собственных чисел. \(L^T = \begin{bmatrix} 0.00 & -7.31 & -2.79 & -1.78 \end{bmatrix}\)}
    \label{fig:task2_states_1}
\end{figure}
\begin{figure}[ht!]
    \centering
    \includegraphics[width=\textwidth]{src/figs/task2_states_1.jpg}
    \caption{Результаты моделирования состояний системы и наблюдателя для второго набора собственных чисел. \(L^T = \begin{bmatrix} 204414.17 &  160532.11 &  126147.31 & -113657.21 \end{bmatrix}\)}
    \label{fig:task2_states_2}
\end{figure}
\begin{figure}[ht!]
    \centering
    \includegraphics[width=\textwidth]{src/figs/task2_states_2.jpg}
    \caption{Результаты моделирования состояний системы и наблюдателя для третьего набора собственных чисел. \(L^T = \begin{bmatrix} 3.09 &  1.03 &  1.95 & -3.05 \end{bmatrix}\)}
    \label{fig:task2_states_3}
\end{figure}
\begin{figure}[ht!]
    \centering
    \includegraphics[width=\textwidth]{src/figs/task2_states_3.jpg}
    \caption{Результаты моделирования состояний системы и наблюдателя для четвертого набора собственных чисел. \(L^T = \begin{bmatrix} 4.34 & -1.60 &  0.93 & -3.97 \end{bmatrix}\)}
    \label{fig:task2_states_4}
\end{figure}

\begin{figure}[ht!]
    \centering
    \includegraphics[width=\textwidth]{src/figs/task2_y_0.jpg}
    \caption{Результаты моделирования выхода системы и наблюдателя для первого набора собственных чисел. \(L^T = \begin{bmatrix} 0.00 & -7.31 & -2.79 & -1.78 \end{bmatrix}\)}
    \label{fig:task2_y_1}
\end{figure}
\begin{figure}[ht!]
    \centering
    \includegraphics[width=\textwidth]{src/figs/task2_y_1.jpg}
    \caption{Результаты моделирования выхода системы и наблюдателя для второго набора собственных чисел. \(L^T = \begin{bmatrix} 204414.17 &  160532.11 &  126147.31 & -113657.21 \end{bmatrix}\)}
    \label{fig:task2_y_2}
\end{figure}
\begin{figure}[ht!]
    \centering
    \includegraphics[width=\textwidth]{src/figs/task2_y_2.jpg}
    \caption{Результаты моделирования выхода системы и наблюдателя для третьего набора собственных чисел. \(L^T = \begin{bmatrix} 3.09 &  1.03 &  1.95 & -3.05 \end{bmatrix}\)}
    \label{fig:task2_y_3}
\end{figure}
\begin{figure}[ht!]
    \centering
    \includegraphics[width=\textwidth]{src/figs/task2_y_3.jpg}
    \caption{Результаты моделирования выхода системы и наблюдателя для четвертого набора собственных чисел. \(L^T = \begin{bmatrix} 4.34 & -1.60 &  0.93 & -3.97 \end{bmatrix}\)}
    \label{fig:task2_y_4}
\end{figure}
% \begin{cases}
%     \dot{x} = P J P^{-1} x + Bu \\
%     y = Cx + Du \\
% \end{cases},

\FloatBarrier