\subsection{Задание 3}

\subsubsection{Теория}
В этом задании выводится наблюдатель регулятор для системы:
\[
        \begin{cases}
                \dot{x} = A x + B u\\
                y = C x \\
                \dot{\hat{x}} = A \hat{x} + B u + L(\hat{y} - y) \\
                \hat{y} = C \hat{x} \\
                u = K \hat{x}
        \end{cases} \xrightarrow{\text{очевидно...}}
        \begin{cases}
            \begin{bmatrix} 
                \dot{x} \\
                \dot{e}
            \end{bmatrix} = 
            \begin{bmatrix} 
                A + BK & -BK\\
                0 & A + LC
            \end{bmatrix}\\
            \hat{x} = x - e \\
            y = Cx \\
            \hat{y} = C \hat{x}
         \end{cases}
\]
Самое важное тут удачно угадать желаемые собственные числа и на забыть сделать их неуправляемыми/ненаблюдаемыми где надо.
\subsubsection{Результаты}
Для системы (полностью управляема и стабилизируема):
\[
    A = \begin{bmatrix}
        5.00 & -5.00 & -9.00 &  3.00\\
       -5.00 &  5.00 & -3.00 &  9.00\\
       -9.00 & -3.00 &  5.00 &  5.00\\
        3.00 &  9.00 &  5.00 &  5.00
      \end{bmatrix};
    B = \begin{bmatrix}
        16.00\\
        12.00\\
        12.00\\
        12.00
      \end{bmatrix};
    C = \begin{bmatrix}
        3.00 & -1.00 &  1.00 &  3.00\\
       -2.00 &  2.00 &  2.00 &  2.00
      \end{bmatrix}
\],
получены следующие матрицы:
\[
    K = \begin{bmatrix}
        11.09 & -3.47 & -11.35 & -3.65
      \end{bmatrix};
    L = \begin{bmatrix}
        36.20 &  36.20\\
        6.46 &  6.46\\
       -36.74 & -36.74\\
        5.91 &  5.91
      \end{bmatrix}
\]

Это позволило стабилизировать систему, результаты приведены на графиках ниже на рисунках \ref{fig:task3_y} - \ref{fig:task3_states}.
\begin{figure}[ht!]
    \centering
    \includegraphics[width=\textwidth]{src/figs/task3_y.jpg}
    \caption{Результаты моделирования выхода системы и наблюдателя.}
    \label{fig:task3_y}
\end{figure}
\begin{figure}[ht!]
    \centering
    \includegraphics[width=\textwidth]{src/figs/task3_u.jpg}
    \caption{Результаты моделирования регулятора.}
    \label{fig:task3_u}
\end{figure}
\begin{figure}[ht!]
    \centering
    \includegraphics[width=\textwidth]{src/figs/task3_states.jpg}
    \caption{Результаты моделирования состояний системы.}
    \label{fig:task3_states}
\end{figure}