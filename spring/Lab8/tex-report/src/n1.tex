\subsection{Задание 1}
\subsubsection{Теория}
В этом задании выводится модальный регулятор для системы:
\[
        \begin{cases}
                \text{Объект управления: }\dot{x} = A x + Bu \\
                \text{Регулятор: }U = K x \\
        \end{cases} \rightarrow
        \dot{x} = A x + BKx = (A+BK)x
\]
По сути, целью данного регулятора является изменение управляемых собственных чисел так\footnote{Что делать с неуправляемыми -- пока непонятно. Видимо надо смириться.}, чтобы \(\forall \lambda \in \sigma(A): Re{\lambda} \leq 0\). Для этого подбирается матрица \(\Gamma \in \mathds{R}^{n \times n}\) с желаемыми собственными числами и матрица \(\mathds{Y} \in \mathds{R}^{m \times n}\), такая что пара \((\mathds{Y}, \Gamma)\) наблюдаема. После чего по подобию:
\[A+BK = P \Gamma P^{-1} \rightarrow
        \begin{cases}
                AP - P\Gamma = BY \\
                K = -YP^{-1} \\
        \end{cases}
\]
На практике, довольно часто \(P\) -- необратима. Приходится использовать псевдообратную.

\subsubsection{Результаты}
Дана система:
\[A = \begin{bmatrix}
        -4.00 &  0.00 &  0.00 &  0.00\\
         0.00 &  1.00 &  0.00 &  0.00\\
         0.00 &  0.00 &  1.00 &  5.00\\
         0.00 &  0.00 & -5.00 &  1.00
       \end{bmatrix},
       B = \begin{bmatrix}
        0.00\\
        2.00\\
        0.00\\
        9.00
      \end{bmatrix},
      x_0 = \begin{bmatrix}
        1.00 &  1.00 &  1.00 &  1.00
      \end{bmatrix}
       \]
У матрицы А только одно неуправляемое собственное число -- -4. Оно меньше 0, поэтому проблем возникнуть не должно. \footnote{Нет, должно, надо подбирать матрицы так, чтобы в паре \((\mathds{Y}, \Gamma)\) оно было ненаблюдаемо. Иначе полученный спектр будет немного отличаться от желаемого.}

Для каждой строки желаемых собственных чисел:
\[\begin{bmatrix}
        -4.00 + 0.00j & -4.00 + 0.00j & -4.00 + 0.00j & -4.00 + 0.00j\\
        -4.00 + 0.00j & -40.00 + 0.00j & -400.00 + 0.00j & -400.00 + 0.00j\\
        -4.00 + 0.00j & -8.00 + 0.00j &  0.00 + 5.00j &  -0.00 + -5.00j\\
        -4.00 + 0.00j & -8.00 + 0.00j & -1.00 + 5.00j & -1.00 + -5.00j
       \end{bmatrix},
\]
необходимо синтезировать регулятор.

На рисунках \ref{fig:task1_1} - \ref{fig:task1_4} видны результаты работы получившихся регуляторов. 

\begin{figure}[ht!]
        \centering
        \includegraphics[width=\textwidth]{src/figs/task1_0.jpg}
        \caption{Результаты моделирования задания 1 для первого набора собственных чисел. \(K = \begin{bmatrix} -0.00 & -2.50 & -1.11 & -1.11 \end{bmatrix}\)}
        \label{fig:task1_1}
\end{figure}
\begin{figure}[ht!]
        \centering
        \includegraphics[width=\textwidth]{src/figs/task1_1.jpg}
        \caption{Результаты моделирования задания 1 для второго набора собственных чисел. \(K = \begin{bmatrix} 0.00 & -131856.82 & -4303.51 &  29207.85 \end{bmatrix}\) }
        \label{fig:task1_2}
\end{figure}
\begin{figure}[ht!]
        \centering
        \includegraphics[width=\textwidth]{src/figs/task1_2.jpg}
        \caption{Результаты моделирования задания 1 для третьего набора собственных чисел. \(K = \begin{bmatrix} 0.00 & -4.68 & -0.42 & -0.18 \end{bmatrix}\)}
        \label{fig:task1_3}
\end{figure}
\begin{figure}[ht!]
        \centering
        \includegraphics[width=\textwidth]{src/figs/task1_3.jpg}
        \caption{Результаты моделирования задания 1 для четвертого набора собственных чисел. \(K = \begin{bmatrix} 0.00 & -5.22 & -0.89 & -0.28 \end{bmatrix}\)}
        \label{fig:task1_4}
\end{figure}

% \begin{cases}
%     \dot{x} = P J P^{-1} x + Bu \\
%     y = Cx + Du \\
% \end{cases},
\FloatBarrier