\subsection{Задание 1}
\subsubsection{Теория}
В этом задании выводится регулятор заданной степени устойчивости для системы:
\[
        \begin{cases}
                \text{Объект управления: }\dot{x} = A x + Bu \\
                \text{Регулятор: }u = K x \\
        \end{cases} \rightarrow
        \dot{x} = A x + BKx = (A+BK)x
\]
По сути, целью данного регулятора является изменение управляемых собственных чисел так, чтобы \(\forall \lambda \in \sigma(A): Re{\lambda} \leq \alpha\), где \(\alpha\) -- степень устойчивости.
Для этого используется LMI критерий экспоненциальной устойчивости:
\[ \exists Q \succ 0 , \alpha > 0 :  A^TQ + QA + 2 \alpha Q \preccurlyeq 0 \rightarrow
\begin{cases}
    \text{\(\forall x(0)\) А ассимптотически устойчива}\\
    \exists c :  ||x(t)|| \le  c e^{-\alpha t} ||x(0)|| \\
\end{cases}
\]

\[      \begin{cases}
                Q \succ 0\\
                A^TQ + QA + 2 \alpha Q \preccurlyeq 0 \\
        \end{cases} \xleftrightarrow{Q = P^{-1}}
        \begin{cases}
                P \succ 0\\
                PA^T + AP + 2 \alpha P \preccurlyeq 0\\
        \end{cases} 
\]
Подставим вместо матрицы \(A\) нашу систему:
\[P(A+BK)^T + (A+BK)P + 2 \alpha P \preccurlyeq 0\]
\[P A^T + P K^T B^T + AP + BKP + 2 \alpha P \preccurlyeq 0\]
\[
        \begin{cases}
                Y = KP\\
                PA^T + AP + 2 \alpha P + Y^T B^T + BY \preccurlyeq 0  \\
        \end{cases} 
\]

Тогда для подбора матрицы \(K\), чтобы задать система степень устойчивости \(\alpha\), достаточно решить через библиотеку CVX следующую систему с двумя неизвестнями:
\[
        \begin{cases}
                K = YP^{-1}\\
                P \succ 0 \\
                PA^T + AP + 2 \alpha P + Y^T B^T + BY \preccurlyeq 0  \\
        \end{cases} 
\]

На практике, довольно часто \(P\) -- необратима. Приходится использовать псевдообратную.

\subsubsection{Результаты}
Дана система:
\[A = \begin{bmatrix}
        -4.00 &  0.00 &  0.00 &  0.00\\
         0.00 &  1.00 &  0.00 &  0.00\\
         0.00 &  0.00 &  1.00 &  5.00\\
         0.00 &  0.00 & -5.00 &  1.00
       \end{bmatrix},
       B = \begin{bmatrix}
        0.00\\
        2.00\\
        0.00\\
        9.00
      \end{bmatrix},
      x_0 = \begin{bmatrix}
        1.00 &  1.00 &  1.00 &  1.00
      \end{bmatrix}
       \]
У матрицы \(A\) одно неуправляемое собственное число -- -4.
 Это наталкивает на мысль, что степень устойчивости больше 4 получить не получится, так как это собственное число подвинуть не получится. 
 На деле так и есть, CVX решает систему для \(\forall \alpha < 4\).

 При рассмотрении собственных чисел получившихся систем видно, что передвигает собственные числа так, чтобы минимальное расстояние по вещественной оси от собственных чисел до 0 было с небольшим запасом больше степени устойчивости. 
 На рисунках \ref{fig:task1_states} - \ref*{fig:task1_us} приведены полученные графики.

Как можно заметить, действительно чем больше \(\alpha\), тем жестче управление и быстрее сходимость. Так же видно, что состояние 1 не меняется, так как собственное число неуправляемое.
\begin{figure}[ht!]
        \centering
        \includegraphics[width=\textwidth]{src/figs/task1_states.jpg}
        \caption{Результаты моделирования состояний для разных значений \(\alpha\).}
        \label{fig:task1_states}
\end{figure}

\begin{figure}[ht!]
        \centering
        \includegraphics[width=\textwidth]{src/figs/task1_us.jpg}
        \caption{Результаты моделирования управления для разных значений \(\alpha\).}
        \label{fig:task1_us}
\end{figure}


% \begin{figure}[ht!]
%         \centering
%         \includegraphics[width=\textwidth]{src/figs/task1_0.jpg}
%         \caption{Результаты моделирования задания 1 для первого набора собственных чисел. \(K = \begin{bmatrix} -0.00 & -2.50 & -1.11 & -1.11 \end{bmatrix}\)}
%         \label{fig:task1_1}
% \end{figure}

% \begin{cases}
%     \dot{x} = P J P^{-1} x + Bu \\
%     y = Cx + Du \\
% \end{cases},
\FloatBarrier