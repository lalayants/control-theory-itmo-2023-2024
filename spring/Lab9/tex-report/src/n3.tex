\subsection{Задание 3}

\subsubsection{Теория}

В этом задании выводится наблюдатель заданной степени устойчивости для системы:
\[
        \begin{cases}
                \dot{x} = A x \\
                y = C x \\
                \dot{\hat{x}} = A \hat{x} + L(\hat{y} - y) \\
                \hat{y} = C \hat{x}
        \end{cases} \rightarrow
        \begin{cases}
            \dot{\hat{x}} = A \hat{x} + L(C \hat{x} - y) = (A + LC )\hat{x} - Ly \\
            \dot{e} = (A + LC)e
            
    \end{cases}
\]

Для этого достаточно решить систему:
\[
        \begin{cases}
                L = Q^{-1}Y\\
                Q \succ 0 \\
                A^TQ + QA + 2 \alpha Q + C^T Y^T + YC \preccurlyeq 0  \\
        \end{cases} 
\]
\subsubsection{Результаты}
У полученных наблюдателей, собственные числа всегда находятся с небольшим запасом левее \(\alpha\). На рис. \ref{fig:task3_states} видна динамика систем покомпонентно, а на рис. \ref{fig:task3_errors} покомпонентная динамика ошибок.
\begin{figure}[ht!]
  \centering
  \includegraphics[width=\textwidth]{src/figs/task3_states.jpg}
  \caption{Динамика наблюдателя и системы.}
  \label{fig:task3_states}
\end{figure}
\begin{figure}[ht!]
  \centering
  \includegraphics[width=\textwidth]{src/figs/task3_errors.jpg}
  \caption{Динамика ошибки.}
  \label{fig:task3_errors}
\end{figure}

\[
\alpha = 0.1; 
L = \begin{bmatrix}
 -0.05\\
 -0.03\\
  0.01\\
 -0.03
\end{bmatrix};
\sigma(A+LC) = \begin{bmatrix}
 -0.14 + 4.05j\\
 -0.14 + -4.05j\\
 -0.12 + 3.07j\\
 -0.12 + -3.07j
\end{bmatrix};
\]
\[
\alpha = 1.0; 
L = \begin{bmatrix}
  2.40\\
 -3.04\\
 -0.38\\
 -2.35
\end{bmatrix};
\sigma(A+LC) = \begin{bmatrix}
 -2.64 + 6.01j\\
 -2.64 + -6.01j\\
 -1.92 + 2.93j\\
 -1.92 + -2.93j
\end{bmatrix};
\]
\[
\alpha = 4.0; 
L = \begin{bmatrix}
  53.19\\
  7.93\\
  17.02\\
 -31.83
\end{bmatrix};
\sigma(A+LC) = \begin{bmatrix}
 -5.37 + 12.68j\\
 -5.37 + -12.68j\\
 -4.87 + 2.89j\\
 -4.87 + -2.89j
\end{bmatrix};
\]

\FloatBarrier



Как видно, наблюдатели с большой степень устойчивости сходятся быстрее, но при этом имеют большую начальную ошибку.
