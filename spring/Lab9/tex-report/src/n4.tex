\subsection{Задание 3}

\subsubsection{Теория}
В этом задании выводится наблюдатель регулятор для системы:
\[
        \begin{cases}
                \dot{x} = A x + B u\\
                y = C x \\
                \dot{\hat{x}} = A \hat{x} + B u + L(\hat{y} - y) \\
                \hat{y} = C \hat{x} \\
                u = K \hat{x}
        \end{cases} \rightarrow
        \begin{cases}
            \begin{bmatrix} 
                \dot{x} \\
                \dot{e}
            \end{bmatrix} = 
            \begin{bmatrix} 
                A + BK & -BK\\
                0 & A + LC
            \end{bmatrix}
            \begin{bmatrix} 
              x \\
              e
          \end{bmatrix} 
            \\
            \hat{x} = x - e \\
            y = Cx \\
            \hat{y} = C \hat{x}
         \end{cases}
\]
Это объединяет задания 1 и 3, описанные выше.

\subsubsection{Результаты}
По результат (рис. \ref{fig:task4}) явно видно, что чем больше \(\alpha\), тем быстрее сходится система. При этом, максимальное значение ошибки -- растет.

\begin{figure}[ht!]
    \centering
    \includegraphics[width=\textwidth]{src/figs/task4_states.jpg}
    \caption{Результаты моделирования.}
    \label{fig:task4}
\end{figure}




\FloatBarrier

