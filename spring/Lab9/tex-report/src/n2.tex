\subsection{Задание 2}
\subsubsection{Теория}
В этом задании выводится ограничение на управление \(||u(t)|| \leq \mu\). Тогда система уравнений принимает вид: 
\[
        \begin{cases}
                \begin{bmatrix}
                    P &  x_0\\
                    x_0^T &  1 \\
                \end{bmatrix} \succ 0 \\
                \\
                \begin{bmatrix}
                    P &  Y^T\\
                    Y &  \mu^2I \\
                \end{bmatrix} \succ 0 \\
                P \succ 0 \\
                PA^T + AP + 2 \alpha P + Y^T B^T + BY \preccurlyeq 0  \\
                K = YP^{-1}\\
        \end{cases} 
\]




\subsubsection{Результаты}
Сначала проведено исследование влияния ограничения для \(alpha = 1\).  На рисунках \ref{fig:task2_1_states} -- \ref{fig:task2_1_us}.

При минимальном \(\mu\) система имеет собственные числа \( \begin{bmatrix}
    -1.00 + 5.50j & -1.00 + -5.50j & -1.00 + 0.00j & -4.00 + 0.00j
   \end{bmatrix} \), максимально прижавшись к необходимой степени устойчивости всеми управляемыми числами. По мере ослабления ограничения, числа начинают отдаляться от границы.
\begin{figure}[ht!]
    \centering
    \includegraphics[width=\textwidth]{src/figs/task2_1_states.jpg}
    \caption{Результаты моделирования состояний системы для \(\alpha = 1\) при различных ограничениях.}
    \label{fig:task2_1_states}
\end{figure}
\begin{figure}[ht!]
    \centering
    \includegraphics[width=\textwidth]{src/figs/task2_1_us.jpg}
    \caption{Результаты моделирования управления системой для \(\alpha = 1\) при различных ограничениях.}
    \label{fig:task2_1_us}
\end{figure}

Для всех остальных степеней устойчивости результат аналогичен. Все управляемы собственные числа принимают значение степени устойчивости.
\begin{figure}[ht!]
    \centering
    \includegraphics[width=\textwidth]{src/figs/task2_2_states.jpg}
    \caption{Результаты моделирования состояний системы для различных \(\alpha\) с минимизированным управлением.}
    \label{fig:task2_2_states}
\end{figure}
% \begin{cases}
%     \dot{x} = P J P^{-1} x + Bu \\
%     y = Cx + Du \\
% \end{cases},

\[
\alpha = 0.1; 
K = \begin{bmatrix}
  0.00\\
 -0.61\\
 -0.11\\
 -0.23
\end{bmatrix};
\sigma(A+BK) = \begin{bmatrix}
 -0.10 + 5.15j\\
 -0.10 + -5.15j\\
 -0.10 + 0.00j\\
 -4.00 + 0.00j
\end{bmatrix};
\]
\[
\alpha = 1.0; 
K = \begin{bmatrix}
  0.00\\
 -1.37\\
 -0.38\\
 -0.36
\end{bmatrix};
\sigma(A+BK) = \begin{bmatrix}
 -1.00 + 5.50j\\
 -1.00 + -5.50j\\
 -1.00 + 0.00j\\
 -4.00 + 0.00j
\end{bmatrix};
\]
\[
\alpha = 2.0; 
K = \begin{bmatrix}
  0.00\\
 -2.80\\
 -0.88\\
 -0.38
\end{bmatrix};
\sigma(A+BK) = \begin{bmatrix}
 -2.00 + 6.13j\\
 -2.00 + -6.13j\\
 -2.00 + 0.00j\\
 -4.00 + 0.00j
\end{bmatrix};
\]
\[
\alpha = 3.0; 
K = \begin{bmatrix}
  0.00\\
 -5.13\\
 -1.58\\
 -0.19
\end{bmatrix};
\sigma(A+BK) = \begin{bmatrix}
 -3.00 + 6.94j\\
 -3.00 + -6.94j\\
 -3.00 + 0.00j\\
 -4.00 + 0.00j
\end{bmatrix};
\]



\FloatBarrier