\subsection{Следящий регулятор по состоянию}

Рассмотрим систему вида:
\[
    \begin{cases}
        \dot{x} = A_1x + B_1u + B_2w \\
        z = C_2x + D_2w
    \end{cases},
\]
где $w$:
\[
    \dot{w} = A_2w
\]

Для данной системы можем синтезировать регулятор вида $u = K_1x + K_2w$, гарантирующий:
\begin{equation*}
    \lim_{t\to\infty} z(t) = 0
\end{equation*}

$K_1$ можем выбрать как матрицу регулятора, синтезированного любым способом. Матрицу $K_2$ найдем следующим образом:
\[
    \begin{cases}
        PA_2 - A_1P = B_1Y + B_2\\
        C_2P + D_2 = 0 \\
        K_2 = Y - K_1P
    \end{cases}
\]

\[
    \begin{cases}
        \dot{x} = A_1x + B_1 (K_1x + K_2w) + B_2w = (A_1 + B_1 K_1) x + (B_2 + B_1 K_2)w \\
        z = C_2x + D_2 w
    \end{cases},
\]

\[A_1 = \begin{bmatrix}
  1.00 &  0.00 &  0.00\\
  0.00 &  2.00 &  0.00\\
  0.00 &  0.00 &  3.00
\end{bmatrix}\]
\[A_2 = \begin{bmatrix}
  0.00 &  1.00 &  1.00 &  0.00\\
 -1.00 &  0.00 &  0.00 &  1.00\\
  0.00 &  0.00 &  0.00 &  1.00\\
  0.00 &  0.00 & -1.00 &  0.00
\end{bmatrix}\]
\[B_1 = \begin{bmatrix}
  1.00\\
  1.00\\
  1.00
\end{bmatrix}\]
\[B_1 = \begin{bmatrix}
  0.00 &  0.00 &  0.00 &  0.00\\
  0.00 &  0.00 &  0.00 &  0.00\\
  0.00 &  0.00 &  0.00 &  0.00
\end{bmatrix}\]
\[C_2 = \begin{bmatrix}
  1.00 &  0.00 &  1.00
\end{bmatrix}\]
\[D_2 = \begin{bmatrix}
 -1.00 &  0.00 & -1.00 &  0.00
\end{bmatrix}\]


Из LQR:
\[K_1 = \begin{bmatrix}
  -2.41 &  -0.00 &  -0.00\\
   -0.00 & -4.24 &  -0.00\\
   -0.00 &  -0.00 & -6.16\\
   -0.00 &  -0.00 &  -0.00
 \end{bmatrix}\]
 \[spec(A + B_1 K_1) = \begin{bmatrix}
  -1.41 + 0.00j & -3.16 + 0.00j & -2.24 + 0.00j
 \end{bmatrix}\]
Из матричных уравнений:
 \[K_2 = \begin{bmatrix}
  -1.04 & -0.24 &  0.06 &  0.03\\
   0.48 & -1.40 &  0.28 & -0.27\\
   0.21 &  0.41 & -1.27 & -0.04\\
   0.00 &  0.00 &  0.00 &  0.00
 \end{bmatrix}\]


\begin{figure}[ht!]
  \centering
  \includegraphics[width=\textwidth]{src/figs/task2_z.jpg}
  \caption{Регулируемый выход системы.}
  \label{fig:task2_z}
\end{figure}

\begin{figure}[ht!]
  \centering
  \includegraphics[width=\textwidth]{src/figs/task2_ws.jpg}
  \caption{Внешнее возмущение.}
  \label{fig:task2_ws}
\end{figure}

\begin{figure}[ht!]
  \centering
  \includegraphics[width=\textwidth]{src/figs/task2_states.jpg}
  \caption{Состояния системы.}
  \label{fig:task2_states}
\end{figure}

\begin{figure}[ht!]
  \centering
  \includegraphics[width=\textwidth]{src/figs/task2_follow.jpg}
  \caption{Слежение компонент за входным воздействием.}
  \label{fig:task2_follow}
\end{figure}

\FloatBarrier