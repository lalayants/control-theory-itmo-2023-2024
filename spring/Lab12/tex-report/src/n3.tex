\subsection{Регулятор по выходу при одинаковых y и z.}

\[
    \begin{cases}
        \dot{x} = A_1x + B_1u + B_2w \\
        y = C_1x + D_1w \\
        z = C_2x + D_2w \\
        \dot{\hat{x}} = A_1\hat{x} + B_1u + B_2\hat{w} + L_1(\hat{y} - y) \\
        \hat{y} = C_1\hat{x} + D_1\hat{w} \\
        \dot{\hat{w}} = A_2\hat{w} + L_2(\hat{y} - y)
    \end{cases},
\]
где $u = K_1\hat{x} + K_2\hat{w}$. 
\[A_1 = \begin{bmatrix}
    1.00 &  0.00 &  0.00 &  0.00\\
    0.00 &  2.00 &  0.00 &  0.00\\
    0.00 &  0.00 &  3.00 &  0.00\\
    0.00 &  0.00 &  0.00 &  4.00
  \end{bmatrix}\]
  \[A_2 = \begin{bmatrix}
    0.00 &  1.00 &  0.00 &  0.00\\
   -1.00 &  0.00 &  0.00 &  0.00\\
    0.00 &  0.00 &  0.00 &  2.00\\
    0.00 &  0.00 & -2.00 &  0.00
  \end{bmatrix}\]
  \[B_1 = \begin{bmatrix}
    21.00\\
    22.00\\
    23.00\\
    24.00
  \end{bmatrix}\]
  \[B_2 = \begin{bmatrix}
    11.00 &  0.00 &  0.00 &  0.00\\
    0.00 &  12.00 &  0.00 &  0.00\\
    0.00 &  0.00 &  13.00 &  0.00\\
    0.00 &  0.00 &  0.00 &  14.00
  \end{bmatrix}\]
  \[C_2 = \begin{bmatrix}
    11.00 &  12.00 &  13.00 &  14.00
  \end{bmatrix}\]
  \[D_2 = \begin{bmatrix}
    3.00 &  1.00 &  1.00 &  2.00
  \end{bmatrix}\]
  \[C_1 = \begin{bmatrix}
    11.00 &  12.00 &  13.00 &  14.00
  \end{bmatrix}\]
  \[D_1 = \begin{bmatrix}
    3.00 &  1.00 &  1.00 &  2.00
  \end{bmatrix}\]
  
\[
    \begin{bmatrix}
        \dot{e_x} \\
        \dot{e_w}
    \end{bmatrix} = 
    \begin{bmatrix}
        A_1 + L_1C_1 & B_2 + L_1D_1 \\
        L_2C_1 & A_2 + L_2D_1
    \end{bmatrix}
    \begin{bmatrix}
        e_x \\
        e_w
    \end{bmatrix} = A_e 
    \begin{bmatrix}
        e_x \\
        e_w
    \end{bmatrix}
\] 
Убедившись, что матрица \(A_e\) -- гурвицева, можем синтезировать регулятор (матрицы $K_1$ и $K_2$) аналогично предыдущим разделам.

Через LQE (Q = I, R = 1) найдем:
\[L_1 = \begin{bmatrix}
    242.43\\
   -1039.35\\
    1339.23\\
   -546.87
  \end{bmatrix}\]
  \[L_2 = \begin{bmatrix}
    0.15\\
   -1.41\\
   -1.05\\
   -0.95
  \end{bmatrix}\]
  \[\sigma (A_e) = \begin{bmatrix}
   -20.90 & -14.31 & -4.21 & -0.44 + 1.65j & -0.44 -1.65j & -2.3 + 0.1j & -2.3 -0.1j & -0.7
  \end{bmatrix}\]
Через LQR  (Q = I, R = 1) найдем:
\[K_1 = \begin{bmatrix}
    14.90 & -95.40 &  172.40 & -93.42
  \end{bmatrix}\]
  \[spec(A + B_1 K_1) = \begin{bmatrix}
   -45.15 & -1.47 & -3.64 & -2.55
  \end{bmatrix}\]
  \[K_2 = \begin{bmatrix}
    314.84 & -372.26 &  645.75 &  84.98
  \end{bmatrix}\]

Получаем систему:
\[
    \begin{bmatrix}
        \dot{x} \\
        \dot{e_x} \\
        \dot{e_w}
    \end{bmatrix} = 
    \begin{bmatrix}
        A_1 + B_1K_1 & -B_1K_1 & -B_1K_2 \\
        0 & A_1 + L_1C_1 & B_2 + L_1 D_1 \\
        0 & L_2 C_1 & A_2 + L_2 D_1
    \end{bmatrix}
    \begin{bmatrix}
        x \\
        e_x \\
        e_w
    \end{bmatrix} 
    + 
    \begin{bmatrix}
        B_2 + B_1 K_2 \\
        0 \\
        0
    \end{bmatrix} w
\]
\[
    \begin{bmatrix}
        \dot{\hat{x}} \\
        \dot{\hat{w}}
    \end{bmatrix} = 
    \begin{bmatrix}
        A_1 + B_1K_1 + L_1C_1 & B_2 + B_1K_2 + L_1D_1 \\
        L_2C_2 & A_2 + L_2D_2 \\
    \end{bmatrix}
    \begin{bmatrix}
        \hat{x} \\
        \hat{w}
    \end{bmatrix}
    +
    \begin{bmatrix}
        -L_1 \\ -L_2 
    \end{bmatrix}y, u = \begin{bmatrix}
        K_1 & K_2 
    \end{bmatrix}\begin{bmatrix}
        {\hat{x}} \\ {\hat{w}} 
    \end{bmatrix}
\]
Собственные числа регулятора включают в себя числа генератора!
\[\sigma (R) = \begin{bmatrix}
    20789.02 & -20901.90 &  3.10 &  1.39 &  2.00j &  -2.00j &  1.00j &  -1.00j
  \end{bmatrix}\]
  \[\sigma (A_2) = \begin{bmatrix}
    1.00j &  -1.00j &  2.00j &  -2.00j
  \end{bmatrix}\]
\begin{figure}[ht!]
    \centering
    \includegraphics[width=\textwidth]{src/figs/task4_xs.jpg}
    \caption{Поведение компонент вектора состояния.}
    \label{fig:task4_xs}
  \end{figure} 

  \begin{figure}[ht!]
    \centering
    \includegraphics[width=\textwidth]{src/figs/task4_exs.jpg}
    \caption{Поведение компонент вектора ошибки наблюдателя состояния системы.}
    \label{fig:task4_exs}
  \end{figure} 
  \begin{figure}[ht!]
    \centering
    \includegraphics[width=\textwidth]{src/figs/task4_ews.jpg}
    \caption{Поведение компонент вектора ошибки наблюдателя входного воздействия.}
    \label{fig:task4_ews}
  \end{figure} 

  \begin{figure}[ht!]
    \centering
    \includegraphics[width=\textwidth]{src/figs/task4_z.jpg}
    \caption{Поведение регулируемого выхода.}
    \label{fig:task4_z}
  \end{figure} 