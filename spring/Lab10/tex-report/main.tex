\documentclass[16pt]{article}

\usepackage{report}

\usepackage[utf8]{inputenc} % allow utf-8 input
\usepackage[english, russian]{babel}
\usepackage[T1]{fontenc}    % use 8-bit T1 fonts
\usepackage[colorlinks=true, linkcolor=black, citecolor=blue, urlcolor=blue]{hyperref}       % hyperlinks
\usepackage{url}            % simple URL typesetting
\usepackage{booktabs}       % professional-quality tables
\usepackage{amsfonts}       % blackboard math symbols
\usepackage{nicefrac}       % compact symbols for 1/2, etc.
\usepackage{microtype}      % microtypography
\usepackage{graphicx}
\usepackage{natbib}
\usepackage{doi}
\usepackage{mathtools}
\usepackage{minted}
\usepackage{ dsfont }
\usepackage{graphicx}
\usepackage{placeins}
\usepackage{ amssymb }
\graphicspath{ {./figs/} }

\setcitestyle{aysep={,}}



\title{ЛР \textnumero 10 <<Линейно-квадратичные радости>>}

\author{
Студент \\
Кирилл Лалаянц\\
R33352\\
336700\\
Вариант - 11\\
\\
Преподаватель\\
Пашенко А.В. \\
\\
\\
\\
Факультет Систем Управления и Робототехники\\
\\
ИТМО\\
}

% Uncomment to remove the date
\date{28.03.2024}

% Uncomment to override  the `A preprint' in the header
\renewcommand{\headeright}{ЛР \textnumero 10 <<Линейно-квадратичные радости>>}
\renewcommand{\undertitle}{Отчет}
\renewcommand{\shorttitle}{}


\begin{document}
\maketitle
\newpage
\tableofcontents
\thispagestyle{empty}

\newpage
\setcounter{page}{1}
\section{Вводные данные}
\subsection{Цель работы}
В этой работе пройдет изучение LQR, LQE, LGC.

\subsubsection{Программная реализация}
C исходным кодом можно ознакомиться \href{https://github.com/lalayants/control-theory-itmo-2023-2024}{в репозитории на Github.}


\newpage
\section{Основная часть}
\[
   A = \begin{bmatrix}
   -6.00 &  19.00 &  10.00 & -13.00\\
    0.00 & -9.00 &  0.00 &  6.00\\
   -4.00 &  8.00 &  6.00 & -7.00\\
    0.00 & -15.00 &  0.00 &  9.00
  \end{bmatrix};
  B = \begin{bmatrix}
   4.00 &  0.00\\
   2.00 &  0.00\\
   6.00 &  0.00\\
   4.00 &  0.00
 \end{bmatrix};
\]
\[
 C = \begin{bmatrix}
   -3.00 &  9.00 &  3.00 & -6.00\\
    0.00 & -2.00 &  0.00 &  1.00
  \end{bmatrix};
  D = \begin{bmatrix}
   0.00 &  0.00\\
   0.00 &  2.00
 \end{bmatrix}
\]
\subsection{Задание 1. LQR}
\subsubsection{Теория}
В этом задании выводится регулятор заданной степени устойчивости для системы:
\[
        \begin{cases}
                \text{Объект управления: }\dot{x} = A x + Bu \\
                \text{Регулятор: }u = -K x \\
        \end{cases} \rightarrow
        \dot{x} = A x - BKx = (A-BK)x
\]
LQR позволяет оптимизировать критерий качества:
\[J = \int_0^\infty (x^T Q x + u^T R u)dt \]
Выбор cотношения матриц \(Q\) и \(R\) позволяет управлять временем сходимости и величиной подаваего управления: чем больше \(\frac{Q}{R}\), тем больше управление и быстрее сходимость.

\(K\) получается решением следующих уравнений:
\[
\begin{cases}
    A^T P + P A + Q - PBR^{-1}B^TP = 0\\
    K = -R^{-1} B^T P \\
\end{cases}
\]

Теоретический минимум критерия качества:
\[J_{min} = x_0^T P x_0\]

\subsubsection{Результаты}
На рис. \ref{fig:task1_states} - \ref{fig:task1_us} видно, что чем больше \(\frac{Q}{R}\), тем больше управление и быстрее сходимость.
\[Q = 0.1; R = 10.0; K_0 = \begin{bmatrix}
        -0.24 &  0.97 &  0.51 & -0.60\\
         0.00 &  0.00 &  0.00 &  0.00
       \end{bmatrix}\]
       \[spec(A-BK_0) = \begin{bmatrix}
        -0.53 + 2.10j & -0.53 + -2.10j & -0.27 + 2.93j & -0.27 + -2.93j
       \end{bmatrix}\]
\[Q = 1.0; R = 1.0; K_1 = \begin{bmatrix}
-1.20 &  10.88 &  3.57 & -6.80\\
        0.00 &  0.00 &  0.00 &  0.00
\end{bmatrix}\]
\[spec(A-BK_1) = \begin{bmatrix}
-0.29 + 2.68j & -0.29 + -2.68j & -3.90 + 0.00j & -6.70 + 0.00j
\end{bmatrix}\]
\[Q = 10.0; R = 0.1; K_2 = \begin{bmatrix}
        -6.23 &  99.18 &  27.80 & -62.96\\
         0.00 &  0.00 &  0.00 &  0.00
       \end{bmatrix}\]
       \[spec(A-BK_2) = \begin{bmatrix}
        -84.73 + 0.00j & -0.29 + 2.67j & -0.29 + -2.67j & -3.03 + 0.00j
       \end{bmatrix}\]

В таблице \ref{table:task1} видно, что критерии качества практически совпали с теортическими.
\begin{figure}[ht!]
        \centering
        \includegraphics[width=\textwidth]{src/figs/task1_states.jpg}
        \caption{Результаты моделирования состояний для разных значений \(Q\) и \(R\).}
        \label{fig:task1_states}
\end{figure}

\begin{figure}[ht!]
        \centering
        \includegraphics[width=\textwidth]{src/figs/task1_us.jpg}
        \caption{Результаты моделирования управления для разных значений \(Q\) и \(R\).}
        \label{fig:task1_us}
\end{figure}

\begin{table}[h!]
        \centering
        \begin{tabular}{| l | l | l | l |} 
            \hline
            $Q$ & $R$ & $J_{theory}$ & $J_{real}$  \\  
            \hline
            $0.1$ & $10$ & $6.2$ & $6.2$  \\  
            $1$ & $1$ & $24.5$ & $24.51$  \\  
            $10$ & $0.1$ & $208.25$ & $208.39$  \\  
        \end{tabular}
        \caption{Критерии качества}
        \label{table:task1}
    \end{table}

% \begin{cases}
%     \dot{x} = P J P^{-1} x + Bu \\
%     y = Cx + Du \\
% \end{cases},
\FloatBarrier
\FloatBarrier

\subsection{Регулятор по выходу при различных y и z.}

\[
    \begin{cases}
        \dot{x} = A_1x + B_1u + B_2w \\
        y = C_1x + D_1w \\
        z = C_2x + D_2w \\
        \dot{\hat{x}} = A_1\hat{x} + B_1u + B_2\hat{w} + L_1(\hat{y} - y) \\
        \hat{y} = C_1\hat{x} + D_1\hat{w} \\
        \dot{\hat{w}} = A_2\hat{w} + L_2(\hat{y} - y)
    \end{cases},
\]
где $u = K_1\hat{x} + K_2\hat{w}$. 
\[A_1 = \begin{bmatrix}
    1.00 &  0.00 &  0.00 &  0.00\\
    0.00 &  2.00 &  0.00 &  0.00\\
    0.00 &  0.00 &  3.00 &  0.00\\
    0.00 &  0.00 &  0.00 &  4.00
  \end{bmatrix}\]
  \[A_2 = \begin{bmatrix}
    0.00 &  1.00 &  0.00 &  0.00\\
   -1.00 &  0.00 &  0.00 &  0.00\\
    0.00 &  0.00 &  0.00 &  2.00\\
    0.00 &  0.00 & -2.00 &  0.00
  \end{bmatrix}\]
  \[B_1 = \begin{bmatrix}
    21.00\\
    22.00\\
    23.00\\
    24.00
  \end{bmatrix}\]
  \[B_2 = \begin{bmatrix}
    11.00 &  0.00 &  0.00 &  0.00\\
    0.00 &  12.00 &  0.00 &  0.00\\
    0.00 &  0.00 &  13.00 &  0.00\\
    0.00 &  0.00 &  0.00 &  14.00
  \end{bmatrix}\]
  \[C_2 = \begin{bmatrix}
    11.00 &  12.00 &  13.00 &  14.00
  \end{bmatrix}\]
  \[D_2 = \begin{bmatrix}
    3.00 &  1.00 &  1.00 &  2.00
  \end{bmatrix}\]
  \[C_1 = \begin{bmatrix}
    1.00 &  2.00 &  3.00 &  4.00
  \end{bmatrix}\]
  \[D_1 = \begin{bmatrix}
    3.00 &  2.00 &  1.00 &  4.00
  \end{bmatrix}\]
  
\[
    \begin{bmatrix}
        \dot{e_x} \\
        \dot{e_w}
    \end{bmatrix} = 
    \begin{bmatrix}
        A_1 + L_1C_1 & B_2 + L_1D_1 \\
        L_2C_1 & A_2 + L_2D_1
    \end{bmatrix}
    \begin{bmatrix}
        e_x \\
        e_w
    \end{bmatrix} = A_e 
    \begin{bmatrix}
        e_x \\
        e_w
    \end{bmatrix}
\] 
Убедившись, что матрица \(A_e\) -- гурвицева, можем синтезировать регулятор (матрицы $K_1$ и $K_2$) аналогично предыдущим разделам.

Через LQE (Q = I, R = 1) найдем:
\[L_1 = \begin{bmatrix}
    390.07\\
   -1139.99\\
    1225.92\\
   -453.69
  \end{bmatrix}\]
  \[L_2 = \begin{bmatrix}
    0.10\\
   -1.41\\
   -1.35\\
   -0.43
  \end{bmatrix}\]
  \[\sigma (A_e) = \begin{bmatrix}
    -5.8 + 3.1j & -5.8 -3.1j & -5.06  & -2.46  & -0.5 + 1.4j & -0.5 -1.42j & -1.74  & -0.69 
   \end{bmatrix}\]
Через LQR (Q = I, R = 1) найдем:
   \[K_1 = \begin{bmatrix}
    14.90 & -95.40 &  172.40 & -93.42
  \end{bmatrix}\]
  \[spec(A + B_1 K_1) = \begin{bmatrix}
   -45.15 & -1.47 & -3.64 & -2.55
  \end{bmatrix}\]
  \[K_2 = \begin{bmatrix}
    314.84 & -372.26 &  645.75 &  84.98
  \end{bmatrix}\]

Получаем систему:
\[
    \begin{bmatrix}
        \dot{x} \\
        \dot{e_x} \\
        \dot{e_w}
    \end{bmatrix} = 
    \begin{bmatrix}
        A_1 + B_1K_1 & -B_1K_1 & -B_1K_2 \\
        0 & A_1 + L_1C_1 & B_2 + L_1 D_1 \\
        0 & L_2 C_1 & A_2 + L_2 D_1
    \end{bmatrix}
    \begin{bmatrix}
        x \\
        e_x \\
        e_w
    \end{bmatrix} 
    + 
    \begin{bmatrix}
        B_2 + B_1 K_2 \\
        0 \\
        0
    \end{bmatrix} w
\]
\[
    \begin{bmatrix}
        \dot{\hat{x}} \\
        \dot{\hat{w}}
    \end{bmatrix} = 
    \begin{bmatrix}
        A_1 + B_1K_1 + L_1C_1 & B_2 + B_1K_2 + L_1D_1 \\
        L_2C_2 & A_2 + L_2D_2 \\
    \end{bmatrix}
    \begin{bmatrix}
        \hat{x} \\
        \hat{w}
    \end{bmatrix}
    +
    \begin{bmatrix}
        -L_1 \\ -L_2 
    \end{bmatrix}y, u = \begin{bmatrix}
        K_1 & K_2 
    \end{bmatrix}\begin{bmatrix}
        {\hat{x}} \\ {\hat{w}} 
    \end{bmatrix}
\]
\[\sigma (Reg) = \begin{bmatrix}
    9125.98 & -9237.91 &  23.85 &  6.07 & -2.34 & -0.07 + 1.76j & -0.07 -1.76j &  1.41
  \end{bmatrix}\]
  \[\sigma (A_2) = \begin{bmatrix}
    0.00 + 1.00j &  0.00 + -1.00j &  0.00 + 2.00j &  0.00 + -2.00j
  \end{bmatrix}\]

\begin{figure}[ht!]
    \centering
    \includegraphics[width=\textwidth]{src/figs/task3_xs.jpg}
    \caption{Поведение компонент вектора состояния.}
    \label{fig:task3_xs}
  \end{figure} 

  \begin{figure}[ht!]
    \centering
    \includegraphics[width=\textwidth]{src/figs/task3_exs.jpg}
    \caption{Поведение компонент вектора ошибки наблюдателя состояния системы.}
    \label{fig:task3_exs}
  \end{figure} 
  \begin{figure}[ht!]
    \centering
    \includegraphics[width=\textwidth]{src/figs/task3_ews.jpg}
    \caption{Поведение компонент вектора ошибки наблюдателя входного воздействия.}
    \label{fig:task3_ews}
  \end{figure} 

  \begin{figure}[ht!]
    \centering
    \includegraphics[width=\textwidth]{src/figs/task3_z.jpg}
    \caption{Поведение регулируемого выхода.}
    \label{fig:task3_z}
  \end{figure} 
\FloatBarrier

\subsection{Assignment 3}
We have the system:
\[
\begin{cases}
    \dot{x} = A x \\
    y = Cx  \\
\end{cases}
\]
\[
        y(t) = C e^{At} x(0)
\]
\subsubsection{Observability through Observability Matrix}
\(V = [C | CA | \hdots | CA^{n-1}]^T\) for \(A \in R^{n \times n}\) and \(C \in R^{k \times n}\) -- the observability matrix of the system. If its rank is equal to \(n\), the system is observable.

\subsubsection{Observability and Observability Gramian}
\[Q(t_1) = \int_0^{t_1} e^{A^T t} C^T C e^{A t} dt\]
For an observable system, the observability Gramian is positive definite at any time \(t\).

\subsubsection{Observability through Eigenvalues of the System Matrix}
\[\forall \lambda \in spec(A): rank(\begin{bmatrix}
        A - \lambda I  \\
        C \\
        \end{bmatrix}) = n \Longleftrightarrow \text{The system is observable} \]

\subsubsection{Observability through Jordan Form}
The Jordan form of matrix \(A = P J P^{-1}\):
\[
        \begin{cases}
                \dot{x} = P J P^{-1} x \\
                y = Cx  \\
        \end{cases}
\]

Let \(\hat{x} = P^{-1}x\), then the system's Jordan form becomes:
\[
        \begin{cases}
                \dot{\hat{x}} = J\hat{x} \\
                y = CP\hat{x} = \hat{C}\hat{x} \\
        \end{cases}
\]

The system in Jordan form is fully observable if:
\begin{itemize}
    \item each eigenvalue corresponds to only one Jordan block.
    \item the elements of the output matrix corresponding to the first columns of the blocks are nonzero.
\end{itemize}

\subsubsection{Initial Conditions of the System}
To compute the initial conditions of the system, it suffices to calculate:
\[x(0) = (P(t_1))^{-1} \int_{0}^{t_1} e^{A^T t} C^T y(t) dt\]


\subsubsection{Results}
Assignment variant:
\[ A = \begin{bmatrix}
        -21 & -38 & 6 \\
        8 & 13 & -4 \\
        -6 & -14 & -1 \\
        \end{bmatrix}; 
        C = \begin{bmatrix}
                9  & 18 & -2
                \end{bmatrix}; 
        y = 3 e^{-5x}\cos{2x} - e^{-5x}\sin{2x}; 
        t_1 = 3
\]

The observability matrix \(V\):
\[ V = \begin{bmatrix}
        9 & 18 & -2 \\
        -33 & -80 & -16 \\
        149 & 438 & 138 \\
        \end{bmatrix};
\]
\[rankU = V = n \rightarrow \text{the system is observable.}\]

Eigenvalues obtained \(spec(A) = [-5 + 2j, -5 - 2j, 1+0j]\). Each of them is observable, which further demonstrates the observability of the system.

Let's also consider observability of eigenvalues through Jordan form:
\[
        J = \begin{bmatrix}
                1 & 0 & 0 \\
                0 & -5 - 2j & 0 \\
                0 & 0 & -5 + 2j \\
                \end{bmatrix};
        \hat{C} = \begin{bmatrix}
                -2 &
                7 &
                7 \\
                \end{bmatrix};
\]
As can be seen, the conditions are met -- another confirmation that the pair \((A, C)\) is observable.

The observability Gramian obtained:
\[
        Q(t_1) = \begin{bmatrix}
        815 & 1627 & -809\\
        1627 & 3251 & -1618 \\
        -809 & -1618 & 807 \\
        \end{bmatrix};
\]
\[
        spec(P(t_1)) = [4872, 0.057, 2.37]
\]The Gramian is positive definite, indicating full observability of the system.

Figure \ref{fig:task3} illustrates the results of the system's simulation. As observed, during the simulation, it indeed arrives at the desired observation vector within the specified time.
\begin{figure}[ht!]
        \centering
        \includegraphics[width=\textwidth]{src/figs/task3.jpg}
        % \caption{Simulation results of Assignment 3.}
        \label{fig:task3}
\end{figure}
Since the system is fully observable, there can't be any other initial states. The dimension of \(Nullspace(V) = dimV - rankV = 0\).

\FloatBarrier

% \subsection{Задание 4}

\subsubsection{Теория}
В этом задании выводится наблюдатель регулятор для системы:
\[
        \begin{cases}
                \dot{x} = A x + B u\\
                y = C x \\
                \dot{\hat{x}} = A \hat{x} + B u + L(\hat{y} - y) \\
                \hat{y} = C \hat{x} \\
                u = K \hat{x}
        \end{cases} \rightarrow
        \begin{cases}
            \begin{bmatrix} 
                \dot{x} \\
                \dot{e}
            \end{bmatrix} = 
            \begin{bmatrix} 
                A + BK & -BK\\
                0 & A + LC
            \end{bmatrix}
            \begin{bmatrix} 
              x \\
              e
          \end{bmatrix} 
            \\
            \hat{x} = x - e \\
            y = Cx \\
            \hat{y} = C \hat{x}
         \end{cases}
\]
Это объединяет задания 1 и 3, описанные выше.

\subsubsection{Результаты}
По результат (рис. \ref{fig:task4}) явно видно, что чем больше \(\alpha\), тем быстрее сходится система. При этом, максимальное значение ошибки -- растет.

\begin{figure}[ht!]
    \centering
    \includegraphics[width=\textwidth]{src/figs/task4_states.jpg}
    \caption{Результаты моделирования.}
    \label{fig:task4}
\end{figure}




\FloatBarrier


% \FloatBarrier

\newpage
\section{Заключение}
В этой работе были изучены LQR, LQE, LGC.
\subsection{Выводы}
\begin{enumerate}
   \item чем больше \(\frac{Q}{R}\) у LQR, тем больше управление и быстрее сходимость.
\end{enumerate}

\end{document}