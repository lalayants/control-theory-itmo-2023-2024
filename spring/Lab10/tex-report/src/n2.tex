\subsection{Задание 2. LQR vs LMI.}
\subsubsection{Теория}
В этом задании будет сравнение LQR с LMI \(\alpha = 0\).

\subsubsection{Результаты}
Как видно, LQR действительно сходится чуть дольше, но зато имеет управления напорядок меньше.
\[spec(A+BK_{LMI}) = \begin{bmatrix}
    -161.92 + 0.00j & -0.29 + 2.63j & -0.29 + -2.63j & -2.31 + 0.00j
   \end{bmatrix}
\]
\[spec(A-BK_{LQR}) = 
\begin{bmatrix}
    -0.29 + 2.68j & -0.29 + -2.68j & -3.90 + 0.00j & -6.70 + 0.00j
   \end{bmatrix}
\]
Несмотря на \(\alpha = 0\), у LMI все равно почему-то появился сильно отрицательный корень, который и вызывает всплеск управления.
\begin{table}[h!]
    \centering
    \begin{tabular}{| l | l | l | l |} 
        \hline
        $J_{theory}$ & $J_{LQR}$ & $J_{LMI}$ \\  
        \hline
        $24.5$ & $24.52$ & 212.4  \\   
        \hline
    \end{tabular}
    \caption{Критерии качества}
    \label{table:task2}
\end{table}



\begin{figure}[ht!]
    \centering
    \includegraphics[width=\textwidth]{src/figs/task2_states.jpg}
    \caption{Результаты моделирования состояний системы.}
    \label{fig:task2_states}
\end{figure}
\begin{figure}[ht!]
    \centering
    \includegraphics[width=\textwidth]{src/figs/task2_us.jpg}
    \caption{Результаты моделирования состояний системы.}
    \label{fig:task2_us}
\end{figure}

% \begin{cases}
%     \dot{x} = P J P^{-1} x + Bu \\
%     y = Cx + Du \\
% \end{cases},
\FloatBarrier