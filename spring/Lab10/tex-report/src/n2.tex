\subsection{Задание 2. LQR vs LMI.}
\subsubsection{Теория}
В этом задании будет сравнение LQR с LMI \(\alpha = 1\).

\subsubsection{Результаты}
Сначала проведено исследование влияния ограничения для \(alpha = 1\).  На рисунках \ref{fig:task2_states} -- \ref{fig:task2_us}.



\begin{figure}[ht!]
    \centering
    \includegraphics[width=\textwidth]{src/figs/task2_states.jpg}
    \caption{Результаты моделирования состояний системы.}
    \label{fig:task2_states}
\end{figure}
\begin{figure}[ht!]
    \centering
    \includegraphics[width=\textwidth]{src/figs/task2_us.jpg}
    \caption{Результаты моделирования состояний системы.}
    \label{fig:task2_us}
\end{figure}
% \begin{cases}
%     \dot{x} = P J P^{-1} x + Bu \\
%     y = Cx + Du \\
% \end{cases},
\FloatBarrier