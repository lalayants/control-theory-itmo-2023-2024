\subsection{Задание 3. Исследование LQE (фильтра Калмана)}

\subsubsection{Теория}
\[      
        \text{Объект управления: }
        \begin{cases}
                \dot{x} = A x + Bu + f, \text{ \(f\) -- внешнее возмущение}\\
                y = Cx + \xi, \text{ \(\xi\) -- помеха измерений}
        \end{cases} 
\]
Матрицы \(Q\) и \(R\) обозначают, насколько сильно мы оцениваем влиянию \(f\) и \(\xi\).

\(L\) получается решением следующих уравнений:
\[
\begin{cases}
    A P + P A^T + Q - PC^TR^{-1}CP = 0\\
    K = -P C^T R^{-1}\\
\end{cases}
\]
\subsubsection{Результаты}
\subsubsubsection{Разные шумы для одного наблюдателя}
\[Q = 1; R = 1; L = \begin{bmatrix}
        -12.29 & -0.54\\
         3.19 & -1.35\\
        -7.98 &  0.26\\
         4.65 & -1.33
       \end{bmatrix}\]

На графиках ниже видно, что наблюдатели отрабатывают как надо.
\begin{figure}[ht!]
  \centering
  \includegraphics[width=\textwidth]{src/figs/task3_1_1_diff_noises.jpg}
  \caption{Сумма ошибок.}
  \label{fig:task3_1}
\end{figure}
\begin{figure}[ht!]
        \centering
        \includegraphics[width=\textwidth]{src/figs/task3_1_10_diff_noises.jpg}
        \caption{Сумма ошибок.}
        \label{fig:task3_2}
\end{figure}
\begin{figure}[ht!]
\centering
\includegraphics[width=\textwidth]{src/figs/task3_10_1_diff_noises.jpg}
\caption{Сумма ошибок.}
\label{fig:task3_3}
\end{figure}


\subsubsubsection{Разные наблюдатели для одного шума}
\[Q = 1; R = 10; L = \begin{bmatrix}
        -1.33 & -0.08\\
         0.09 & -0.22\\
        -0.89 &  0.08\\
         0.04 & -0.22
       \end{bmatrix}\]
       \[Q = 10; R = 1; L = \begin{bmatrix}
        -108.11 & -27.74\\
         31.97 & -10.01\\
        -72.07 & -17.20\\
         46.43 & -6.82
       \end{bmatrix}\]
       \[Q = 1; R = 1; L = \begin{bmatrix}
        -12.29 & -0.54\\
         3.19 & -1.35\\
        -7.98 &  0.26\\
         4.65 & -1.33
       \end{bmatrix}\]

На графиках ниже видно, что наблюдатели отрабатывают как надо. Действительно, матрицы \(Q\) и \(R\), совпадающие с дисперсией шума, дают лучший наблюдатель.
\begin{figure}[ht!]
  \centering
  \includegraphics[width=\textwidth]{src/figs/task3_1_1_diff_qr.jpg}
  \caption{Сумма ошибок.}
  \label{fig:task3_4}
\end{figure}
\begin{figure}[ht!]
        \centering
        \includegraphics[width=\textwidth]{src/figs/task3_1_10_diff_qr.jpg}
        \caption{Сумма ошибок.}
        \label{fig:task3_5}
\end{figure}
\begin{figure}[ht!]
\centering
\includegraphics[width=\textwidth]{src/figs/task3_10_1_diff_qr.jpg}
\caption{Сумма ошибок.}
\label{fig:task3_6}
\end{figure}

\FloatBarrier