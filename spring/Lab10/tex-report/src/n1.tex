\subsection{Задание 1. LQR}
\subsubsection{Теория}
В этом задании выводится регулятор заданной степени устойчивости для системы:
\[
        \begin{cases}
                \text{Объект управления: }\dot{x} = A x + Bu \\
                \text{Регулятор: }u = -K x \\
        \end{cases} \rightarrow
        \dot{x} = A x - BKx = (A-BK)x
\]
LQR позволяет оптимизировать критерий качества:
\[J = \int_0^\infty (x^T Q x + u^T R u)dt \]
Выбор cотношения матриц \(Q\) и \(R\) позволяет управлять временем сходимости и величиной подаваего управления: чем больше \(\frac{Q}{R}\), тем больше управление и быстрее сходимость.

\(K\) получается решением следующих уравнений:
\[
\begin{cases}
    A^T P + P A + Q - PBR^{-1}B^TP = 0\\
    K = -R^{-1} B^T P \\
\end{cases}
\]

Теоретический минимум критерия качества:
\[J_{min} = x_0^T P x_0\]

\subsubsection{Результаты}
На рис. \ref{fig:task1_states} - \ref{fig:task1_us} видно, что чем больше \(\frac{Q}{R}\), тем больше управление и быстрее сходимость.
\[Q = 0.1; R = 10.0; K_0 = \begin{bmatrix}
        -0.24 &  0.97 &  0.51 & -0.60\\
         0.00 &  0.00 &  0.00 &  0.00
       \end{bmatrix}\]
       \[spec(A-BK_0) = \begin{bmatrix}
        -0.53 + 2.10j & -0.53 + -2.10j & -0.27 + 2.93j & -0.27 + -2.93j
       \end{bmatrix}\]
\[Q = 1.0; R = 1.0; K_1 = \begin{bmatrix}
-1.20 &  10.88 &  3.57 & -6.80\\
        0.00 &  0.00 &  0.00 &  0.00
\end{bmatrix}\]
\[spec(A-BK_1) = \begin{bmatrix}
-0.29 + 2.68j & -0.29 + -2.68j & -3.90 + 0.00j & -6.70 + 0.00j
\end{bmatrix}\]
\[Q = 10.0; R = 0.1; K_2 = \begin{bmatrix}
        -6.23 &  99.18 &  27.80 & -62.96\\
         0.00 &  0.00 &  0.00 &  0.00
       \end{bmatrix}\]
       \[spec(A-BK_2) = \begin{bmatrix}
        -84.73 + 0.00j & -0.29 + 2.67j & -0.29 + -2.67j & -3.03 + 0.00j
       \end{bmatrix}\]

В таблице \ref{table:task1} видно, что критерии качества практически совпали с теортическими.
\begin{figure}[ht!]
        \centering
        \includegraphics[width=\textwidth]{src/figs/task1_states.jpg}
        \caption{Результаты моделирования состояний для разных значений \(Q\) и \(R\).}
        \label{fig:task1_states}
\end{figure}

\begin{figure}[ht!]
        \centering
        \includegraphics[width=\textwidth]{src/figs/task1_us.jpg}
        \caption{Результаты моделирования управления для разных значений \(Q\) и \(R\).}
        \label{fig:task1_us}
\end{figure}

\begin{table}[h!]
        \centering
        \begin{tabular}{| l | l | l | l |} 
            \hline
            $Q$ & $R$ & $J_{theory}$ & $J_{real}$  \\  
            \hline
            $0.1$ & $10$ & $6.2$ & $6.2$  \\  
            $1$ & $1$ & $24.5$ & $24.51$  \\  
            $10$ & $0.1$ & $208.25$ & $208.39$  \\  
        \end{tabular}
        \caption{Критерии качества}
        \label{table:task1}
    \end{table}

% \begin{cases}
%     \dot{x} = P J P^{-1} x + Bu \\
%     y = Cx + Du \\
% \end{cases},
\FloatBarrier