\subsection{Assignment 4}
\subsubsection{Belonging to the Unobservable Subspace}
In this assignment, it's additionally necessary to know how to check the state's belonging to the unobservable subspace.

To do this, it's required to verify the equality \(Vx = 0\). If this holds true, the state is unobservable.

Accordingly, the basis of the unobservable subspace is equal to \(Nullspace(V)\).

\subsubsection{Results}
Assignment variant:
\[ A = \begin{bmatrix}
        -21 & -38 & 6 \\
        8 & 13 & -4 \\
        -6 & -14 & -1 \\
        \end{bmatrix}; 
        C = \begin{bmatrix}
                7 & 14 & 0
                \end{bmatrix}; 
        y = 3 e^{-5x}\cos{2x} - e^{-5x}\sin{2x}; 
        t_1 = 3
\]

The observability matrix \(V\):
\[ V = \begin{bmatrix}
        7 & 14 & 0 \\
        -35 & -84 & -14 \\
        147 & 434 & 140 \\
        \end{bmatrix};
\]
\[rankV = 2 = n \rightarrow \text{the system is partially observable.}\]

Eigenvalues obtained \(spec(A) = [-5+2j, -5 -2j, 1+0j]\). The third one is unobservable, which once again emphasizes the unobservability of the system.

Let's also consider the observability of eigenvalues through the Jordan form:
\[
        J = \begin{bmatrix}
                1 & 0 & 0 \\
                0 & -5 - 2j & 0 \\
                0 & 0 & -5 + 2j \\
                \end{bmatrix};
        \hat{C} = \begin{bmatrix}
                0 &
                -3.5 + 3.5j &
                -3.5 - 3.5j \\
                \end{bmatrix};
\]
As can be seen, the conditions are met -- another confirmation that the pair \((A, C)\) is unobservable.

The observability Gramian obtained:
\[
        P(t_1) = \begin{bmatrix}
        4.56 & 8.27 & -0.84 \\
        8.27 & 15.2 & -1.35 \\
        -0.84 & -1.35 & 0.33 \\
        \end{bmatrix};
\]
\[
        spec(P(t_1)) = [19.8,  0.25, 0]
\]
One of the eigenvalues of the Gramian is 0, indicating that it's not positive definite, hence the pair is unobservable.

Figure \ref{fig:task4} illustrates the results of the system's simulation. As observed, during the simulation, it indeed arrives at the desired observation vector within the specified time.
\begin{figure}[ht!]
        \centering
        \includegraphics[width=\textwidth]{src/figs/task4.jpg}
        % \caption{Simulation results of Assignment 4.}
        \label{fig:task4}
\end{figure}

Since the system is only partially observable, there might be other initial states leading to the same observation. To find them, let's find the basis of \(Nullspace(V)\). Since the rank of the observability matrix is 2, and the dimension of the system matrix is 3, the basis of the null space will represent a line, defined by one vector \([-0.81, 0.41, -0.41]\).
As shown in Figure \ref{fig:task4_null}, despite different initial conditions and state behaviors, the observations completely coincide.
\begin{figure}[ht!]
        \centering
        \includegraphics[width=\textwidth]{src/figs/task4_nullspace.jpg}
        % \caption{Simulation results of the unobservable subspace of Assignment 4.}
        \label{fig:task4_null}
\end{figure}
