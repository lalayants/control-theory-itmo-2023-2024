\subsection{Assignment 1}
Consider the system:
\[\dot{x} = A x + Bu \]
\[x(t_1) = \int_0^{t_1} B u(t) dt\]

\subsubsection{Controllability through Controllability Matrix}
\(U = [B | AB | \hdots | A^{n-1}B]\) for \(A \in R^{n \times n}\) and \(B \in R^{n \times m}\) -- the controllability matrix of the system. If its rank is equal to \(n\), the system is controllable.

\subsubsection{Controllability and Controllability Gramian}
\[P(t_1) = \int_0^{t_1} e^{At} B B^T e^{A^T t} dt\]
For a controllable system, the controllability Gramian is positive definite at any time \(t\).

\subsubsection{Controllability through Eigenvalues of the System Matrix}
\[\forall \lambda \in spec(A): rank(A - \lambda I | B) = n \Longleftrightarrow \text{The system is controllable} \]

\subsubsection{Controllability through Jordan Form}
The Jordan form of matrix \(A = P J P^{-1}\):
\[
        \dot{x} = P J P^{-1} x + Bu 
\]

Let \(\hat{x} = P^{-1}x\), then the system's Jordan form becomes:
\[
        \dot{\hat{x}} = J\hat{x} + P^{-1}Bu = J\hat{x} + \hat{B}u  \\
\]

The system in Jordan form is fully controllable if:
\begin{itemize}
    \item each eigenvalue corresponds to only one Jordan block.
    \item the elements of the input matrix corresponding to the last rows of the blocks are nonzero.
\end{itemize}

\subsubsection{Software Control of the System}
To compute the control necessary to reach the state \(x_1\) at time \(t_1\), it suffices to calculate:
\[u(t) = B^T e^{A^T (t_1 - t)} (P(t_1))^{-1} x_1\]


\subsubsection{Results}
Assignment variant:
\[ A = \begin{bmatrix}
        3 & 4 & -1 \\
        -10 & -11 & -4 \\
        10 & 10 & 3 \\
        \end{bmatrix}; 
        B = \begin{bmatrix}
                -2 \\
                5 \\
                -3 \\
                \end{bmatrix}; 
        x_1 = \begin{bmatrix}
                -2 \\
                1 \\
                -1 \\
                \end{bmatrix}; 
        t_1 = 3
\]

The controllability matrix \(U\):
\[ U = \begin{bmatrix}
        -2 & 17 & -62 \\
        5 & -23 & -1 \\
        -3 & 21 & 3 \\
        \end{bmatrix};
\]
\[rankU = 3 = n \rightarrow \text{the system is controllable.}\]

Eigenvalues obtained \(spec(A) = [-2+5j, -2 -5j, -1+0j]\). Each of them satisfies the expression \(rank(A - \lambda I | B) = n \) -- thus controllable.

Let's also consider controllability of eigenvalues through Jordan form:
\[
        J = \begin{bmatrix}
                -1 & 0 & 0 \\
                0 & -2 - 5j & 0 \\
                0 & 0 & -2 + 5j \\
                \end{bmatrix};
        \hat{B} = \begin{bmatrix}
                2 \\
                -1.5 + 1.5j \\
                -1.5 - 1.5j \\
                \end{bmatrix};
\]
As can be seen, the conditions are met -- another confirmation that the pair \((A, B)\) is controllable.

Since the matrix is fully controllable, any point belongs to the controllable subspace of the system, including \(x_1\).

Controllability Gramian obtained:
\[
        P(t_1) = \begin{bmatrix}
        1.20 & -1.34 & 0.23 \\
        -1.34 & 2.76 & -1.12 \\
        0.23 & -1.12 & 1.47 \\
        \end{bmatrix};
\]
\[
        spec(P(t_1)) = [4, 0.27, 1.15]
\]The Gramian is positive definite, indicating full controllability of the system.

Figure \ref{fig:task1} illustrates the results of the system's simulation. As observed, it reaches the desired state within the specified time.
\begin{figure}[ht!]
        \centering
        \includegraphics[width=\textwidth]{src/figs/task1.jpg}
        % \caption{Simulation results of Assignment 1.}
        \label{fig:task1}
\end{figure}
