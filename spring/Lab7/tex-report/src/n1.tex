\subsection{Задание 1. Одноканальная система в форме вход-выход.}
Имеем систему:
\[\dot{x} = A x + Bu \]

\(U = [B | AB | \hdots | A^{n-1}B]\) для \(A \in R^{n \times n}\) и \(B \in R^{n \times m}\) -- матрица управляемости системы. Если ее ранг равен \(n\) -- система управляема.

Жорданова форма матрицы \(A = P J P^{-1}\):
\[
        \dot{x} = P J P^{-1} x + Bu 
\]

Пусть \(\hat{x} = P^{-1}x\), тогда получим Жорданову форму системы:
\[
        \dot{\hat{x}} = J\hat{x} + P^{-1}Bu = J\hat{x} + \hat{B}u  \\
\]

Система в Жордановой форме полностью управляема, если:
\begin{itemize}
    \item каждому собственному числу соответсвует только одна Жорданова клетка. 
    \item элементы матрицы входного воздействия, соответсвующие последним строкам клеток -- не нулевые.
\end{itemize}
\subsubsection{Программная реализация}
\subsubsection{Результаты}
\begin{figure}[ht!]
        \centering
        \includegraphics[width=\textwidth]{src/figs/task1.jpg}
        \caption{Результаты моделирования задания 1.}
        \label{fig:task1}
\end{figure}


% \begin{cases}
%     \dot{x} = P J P^{-1} x + Bu \\
%     y = Cx + Du \\
% \end{cases},