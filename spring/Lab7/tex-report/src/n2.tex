\subsection{Assignment 2}
\subsubsection{Belonging of the State to the Controllable Subspace}
In this assignment, it is additionally necessary to know how to check if the state belongs to the controllable subspace.

To do this, it is necessary to compare \(rank(U)\) and \(rank(U | x_1)\). If they match -- the state belongs.

\subsubsection{Results}
Assignment variant:
\[ A = \begin{bmatrix}
        3 & 4 & -1 \\
        -10 & -11 & -4 \\
        10 & 10 & 3 \\
        \end{bmatrix}; 
        B = \begin{bmatrix}
                2 \\
                1 \\
                -1 \\
                \end{bmatrix}; 
        x_1' = \begin{bmatrix}
                -2 \\
                1 \\
                -1 \\
                \end{bmatrix}; 
        x_1'' = \begin{bmatrix}
                -5 \\
                4 \\
                -1 \\
                \end{bmatrix}; 
        t_1 = 3
\]

The controllability matrix \(U\):
\[ U = \begin{bmatrix}
        2 & 11 & -102 \\
        1 & -27 & 79 \\
        -1 & 27 & -79 \\
        \end{bmatrix};
\]
\[rankU = 2 = n \rightarrow \text{the system is incompletely controllable.}\]

Eigenvalues obtained \(spec(A) = [-2+5j, -2 -5j, -1+0j]\). The third one is uncontrollable.

Let's also consider controllability of eigenvalues through Jordan form:
\[
        J = \begin{bmatrix}
                -1 & 0 & 0 \\
                0 & -2 - 5j & 0 \\
                0 & 0 & -2 + 5j \\
                \end{bmatrix};
        \hat{B} = \begin{bmatrix}
                0 \\
                -1.5 + 1.5j \\
                -1.5 - 1.5j \\
                \end{bmatrix};
\]
As can be seen, for the first eigenvalue, the corresponding element in the control vector is 0. This further confirms its uncontrollability.

Out of \(x_1'\) and \(x_1''\), only the first state belongs to the controllable subspace.

Controllability Gramian obtained:
\[
        P(t_1) = \begin{bmatrix}
        2.05 & -1.63 & 1.63 \\
        -1.63 & 2.40 & -2.40 \\
        1.63 & -2.40 & 2.40 \\
        \end{bmatrix};
\]
\[
        spec(P(t_1)) = [0.74,  6.12, 0]
\]
One of the eigenvalues of the Gramian is 0. To find the software control, it is necessary to use the pseudo-inverse matrix.

Figure \ref{fig:task2} illustrates the results of the system's simulation. As observed, it reaches the desired state within the specified time.
\begin{figure}[ht!]
        \centering
        \includegraphics[width=\textwidth]{src/figs/task2.jpg}
        % \caption{Simulation results of Assignment 2.}
        \label{fig:task2}
\end{figure}
