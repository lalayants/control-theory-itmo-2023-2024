\subsection{Задание 2}
\subsubsection{Принадлежность состояния управвляемому подпространству}
В этом задании дополнительно лишь нужно знать, как проверить состояние на принадлежность управляемому подпространству.

Для этого необходимо сравнить \(rank(U)\) и \(rank(U | x_1)\). Если они совпали -- точка принадлежит.


\subsubsection{Результаты}
Вариант задания:
\[ A = \begin{bmatrix}
        3 & 4 & -1 \\
        -10 & -11 & -4 \\
        10 & 10 & 3 \\
        \end{bmatrix}; 
        B = \begin{bmatrix}
                2 \\
                1 \\
                -1 \\
                \end{bmatrix}; 
        x_1' = \begin{bmatrix}
                -2 \\
                1 \\
                -1 \\
                \end{bmatrix}; 
        x_1'' = \begin{bmatrix}
                -5 \\
                4 \\
                -1 \\
                \end{bmatrix}; 
        t_1 = 3
\]

Матрица управляемости U:
\[ U = \begin{bmatrix}
        2 & 11 & -102 \\
        1 & -27 & 79 \\
        -1 & 27 & -79 \\
        \end{bmatrix};
\]
\[rankU = 2 = n \rightarrow \text{система  неполностью управляема.}\]

Получены собственные числа \(spec(A) = [-2+5j, -2 -5j, -1+0j]\). Третье из них неуправляемо.

Так же рассмотрим управляемость собственных чисел через форму Жордана:
\[
        J = \begin{bmatrix}
                -1 & 0 & 0 \\
                0 & -2 - 5j & 0 \\
                0 & 0 & -2 + 5j \\
                \end{bmatrix};
        \hat{B} = \begin{bmatrix}
                0 \\
                -1.5 + 1.5j \\
                -1.5 - 1.5j \\
                \end{bmatrix};
\]
Как видно, у первого числа соотвествующий элемент в векторе управления 0. Это еще раз подтверждает его неуправляемость.

Из \(x_1'\) и \(x_1''\) только первое состояние принадлежит управляемому подпространству.

Получен Грамиан управляемости:
\[
        P(t_1) = \begin{bmatrix}
        2.05 & -1.63 & 1.63 \\
        -1.63 & 2.40 & -2.40 \\
        1.63 & -2.40 & 2.40 \\
        \end{bmatrix};
\]
\[
        spec(P(t_1)) = [0.74,  6.12, 0]
\]
Одно из собственных чисел Грамиана -- 0. Для нахождения програмного управления необходимо использовать псевдообратную матрицу.

На рисунке \ref{fig:task2} приведены результаты моделирования системы. Как видно, она приняла желаемо состояние за нужное время.
\begin{figure}[ht!]
        \centering
        \includegraphics[width=\textwidth]{src/figs/task2.jpg}
        \caption{Результаты моделирования задания 2.}
        \label{fig:task2}
\end{figure}


% \begin{cases}
%     \dot{x} = P J P^{-1} x + Bu \\
%     y = Cx + Du \\
% \end{cases},