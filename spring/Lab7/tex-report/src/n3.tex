\subsection{Задание 3}
Имеем систему:
\[
\begin{cases}
    \dot{x} = A x \\
    y = Cx  \\
\end{cases}
\]
\[
        y(t) = C e^{At} x(0)
\]
\subsubsection{Наблюдаемость через матрицу наблюдаемости}
\(V = [C | CA | \hdots | CA^{n-1}]^T\) для \(A \in R^{n \times n}\) и \(C \in R^{k \times n}\) -- матрица наблюдаемости системы. Если ее ранг равен \(n\) -- система наблюдаема.

\subsubsection{Наблюдаемость и Грамиан наблюдаемости}
\[Q(t_1) = \int_0^{t_1} e^{A^T t} C^T C e^{A t} dt\]
У наблюдаемой системы Грамиан наблюдаемости положительно определен в любой момент t.

\subsubsection{Управляемость через собственные числа матрицы системы}
\[\forall \lambda \in spec(A): rank(\begin{bmatrix}
        A - \lambda I  \\
        C \\
        \end{bmatrix}) = n \Longleftrightarrow \text{Матрица наблюдаема} \]

\subsubsection{Управляемость через Жорданову форму}
Жорданова форма матрицы \(A = P J P^{-1}\):
\[
        \begin{cases}
                \dot{x} = P J P^{-1} x \\
                y = Cx  \\
        \end{cases}
\]

Пусть \(\hat{x} = P^{-1}x\), тогда получим Жорданову форму системы:
\[
        \begin{cases}
                \dot{\hat{x}} = J\hat{x} \\
                y = CP\hat{x} = \hat{C}\hat{x} \\
        \end{cases}
\]

Система в Жордановой форме полностью наблюдаема, если:
\begin{itemize}
    \item каждому собственному числу соответсвует только одна Жорданова клетка. 
    \item элементы матрицы выходов, соответсвующие первым столбцам клеток -- не нулевые.
\end{itemize}

\subsubsection{Начальные условия системы}
Для вычисления начальных условий системы, достаточно рассчитать:
\[x(0) = (P(t_1))^{-1} \int_{0}^{t_1} e^{A^T t} C^T y(t) dt\]


\subsubsection{Результаты}
Вариант задания:
\[ A = \begin{bmatrix}
        3 & 4 & -1 \\
        -10 & -11 & -4 \\
        10 & 10 & 3 \\
        \end{bmatrix}; 
        B = \begin{bmatrix}
                -2 \\
                5 \\
                -3 \\
                \end{bmatrix}; 
        x_1 = \begin{bmatrix}
                -2 \\
                1 \\
                -1 \\
                \end{bmatrix}; 
        t_1 = 3
\]

Матрица управляемости U:
\[ U = \begin{bmatrix}
        -2 & 17 & -62 \\
        5 & -23 & -1 \\
        -3 & 21 & 3 \\
        \end{bmatrix};
\]
\[rankU = 3 = n \rightarrow \text{система управляема.}\]

Получены собственные числа \(spec(A) = [-2+5j, -2 -5j, -1+0j]\). Каждое из них удовлетворяет выражению\(rank(A - \lambda I | B) = n \) -- то есть управлямо по ранговому критерию.

Так же рассмотрим управляемость собственных чисел через форму Жордана:
\[
        J = \begin{bmatrix}
                -1 & 0 & 0 \\
                0 & -2 - 5j & 0 \\
                0 & 0 & -2 + 5j \\
                \end{bmatrix};
        \hat{B} = \begin{bmatrix}
                2 \\
                -1.5 + 1.5j \\
                -1.5 - 1.5j \\
                \end{bmatrix};
\]
Как видно, условия выполнены -- еще одно подтверждение, что матрица А управляема.

Так как матрица полностью управляема, любая точка принадлежит управвляемому подпространству системы, в том числе и \(x_1\).

Получен Грамиван управляемости:
\[
        P(t_1) = \begin{bmatrix}
        1.20 & -1.34 & 0.23 \\
        -1.34 & 2.76 & -1.12 \\
        0.23 & -1.12 & 1.47 \\
        \end{bmatrix};
\]
\[
        spec(P(t_1)) = [4, 0.27, 1.15]
\]Грамиан положительно определен, система полностью управляема.

На рисунке \ref{fig:task1} приведены результаты моделирования системы. Как видно, она приняла желаемо состояние за нужное время.
\begin{figure}[ht!]
        \centering
        \includegraphics[width=\textwidth]{src/figs/task1.jpg}
        \caption{Результаты моделирования задания 1.}
        \label{fig:task1}
\end{figure}


% \begin{cases}
%     \dot{x} = P J P^{-1} x + Bu \\
%     y = Cx + Du \\
% \end{cases},