\subsection{Assignment 3}
We have the system:
\[
\begin{cases}
    \dot{x} = A x \\
    y = Cx  \\
\end{cases}
\]
\[
        y(t) = C e^{At} x(0)
\]
\subsubsection{Observability through Observability Matrix}
\(V = [C | CA | \hdots | CA^{n-1}]^T\) for \(A \in R^{n \times n}\) and \(C \in R^{k \times n}\) -- the observability matrix of the system. If its rank is equal to \(n\), the system is observable.

\subsubsection{Observability and Observability Gramian}
\[Q(t_1) = \int_0^{t_1} e^{A^T t} C^T C e^{A t} dt\]
For an observable system, the observability Gramian is positive definite at any time \(t\).

\subsubsection{Observability through Eigenvalues of the System Matrix}
\[\forall \lambda \in spec(A): rank(\begin{bmatrix}
        A - \lambda I  \\
        C \\
        \end{bmatrix}) = n \Longleftrightarrow \text{The system is observable} \]

\subsubsection{Observability through Jordan Form}
The Jordan form of matrix \(A = P J P^{-1}\):
\[
        \begin{cases}
                \dot{x} = P J P^{-1} x \\
                y = Cx  \\
        \end{cases}
\]

Let \(\hat{x} = P^{-1}x\), then the system's Jordan form becomes:
\[
        \begin{cases}
                \dot{\hat{x}} = J\hat{x} \\
                y = CP\hat{x} = \hat{C}\hat{x} \\
        \end{cases}
\]

The system in Jordan form is fully observable if:
\begin{itemize}
    \item each eigenvalue corresponds to only one Jordan block.
    \item the elements of the output matrix corresponding to the first columns of the blocks are nonzero.
\end{itemize}

\subsubsection{Initial Conditions of the System}
To compute the initial conditions of the system, it suffices to calculate:
\[x(0) = (P(t_1))^{-1} \int_{0}^{t_1} e^{A^T t} C^T y(t) dt\]


\subsubsection{Results}
Assignment variant:
\[ A = \begin{bmatrix}
        -21 & -38 & 6 \\
        8 & 13 & -4 \\
        -6 & -14 & -1 \\
        \end{bmatrix}; 
        C = \begin{bmatrix}
                9  & 18 & -2
                \end{bmatrix}; 
        y = 3 e^{-5x}\cos{2x} - e^{-5x}\sin{2x}; 
        t_1 = 3
\]

The observability matrix \(V\):
\[ V = \begin{bmatrix}
        9 & 18 & -2 \\
        -33 & -80 & -16 \\
        149 & 438 & 138 \\
        \end{bmatrix};
\]
\[rankU = V = n \rightarrow \text{the system is observable.}\]

Eigenvalues obtained \(spec(A) = [-5 + 2j, -5 - 2j, 1+0j]\). Each of them is observable, which further demonstrates the observability of the system.

Let's also consider observability of eigenvalues through Jordan form:
\[
        J = \begin{bmatrix}
                1 & 0 & 0 \\
                0 & -5 - 2j & 0 \\
                0 & 0 & -5 + 2j \\
                \end{bmatrix};
        \hat{C} = \begin{bmatrix}
                -2 &
                7 &
                7 \\
                \end{bmatrix};
\]
As can be seen, the conditions are met -- another confirmation that the pair \((A, C)\) is observable.

The observability Gramian obtained:
\[
        Q(t_1) = \begin{bmatrix}
        815 & 1627 & -809\\
        1627 & 3251 & -1618 \\
        -809 & -1618 & 807 \\
        \end{bmatrix};
\]
\[
        spec(P(t_1)) = [4872, 0.057, 2.37]
\]The Gramian is positive definite, indicating full observability of the system.

Figure \ref{fig:task3} illustrates the results of the system's simulation. As observed, during the simulation, it indeed arrives at the desired observation vector within the specified time.
\begin{figure}[ht!]
        \centering
        \includegraphics[width=\textwidth]{src/figs/task3.jpg}
        % \caption{Simulation results of Assignment 3.}
        \label{fig:task3}
\end{figure}
Since the system is fully observable, there can't be any other initial states. The dimension of \(Nullspace(V) = dimV - rankV = 0\).
