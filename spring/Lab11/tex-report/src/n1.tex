\subsection{Синтез H2-регулятора по состоянию.}
\subsubsection{Теория}
\[
        \begin{cases}
                \dot{x} = A x + B_1 w + B_2 u \\
                z = C_2x + D_2 u  
        \end{cases}
\]
Принято, что \(C_2^T D_2 = 0\).
Можем синтезировать $H_2$-регулятор по состоянию ($u=Kx$) следующим образом:
\[
    \begin{cases}
        A^TQ + QA + C_2^TC_2 - QB_2(D_2^TD_2)^{-1}B_2^TQ=0 \\
        K = -(D_2^TD_2)^{-1}B_2^TQ
    \end{cases}
\]
Существует \(P > 0\) решение уравнения Рикатти, если:
\begin{enumerate}
  \item \(C_2^TD_2 = 0\) (\(C_2C_2^T = Q; D_2^TD_2 = R;\))
  \item \(D_2D_2^T\) -- обратима
  \item \((C_2, A)\) -- обнаруживаема
\end{enumerate}
\[||W(s)_{w \rightarrow z}||_{H_2} = \sqrt{trace(B_1^T Q B_1)}\]

Пусть \(w = [\sin \sin \cos]^T\), 

\subsubsubsection{Вариант 1}
\[C_2 = \begin{bmatrix}
  1.00 &  1.00\\
  0.00 &  1.00\\
  0.00 &  0.00
\end{bmatrix}; D_2 = \begin{bmatrix}
  0.00\\
  0.00\\
  1.00
\end{bmatrix};\]
\[C_2^T D_2 = 0: True\]
\[D_2^T D_2 \text{ обратима}: True\]
\[spec(A-B_2 K) = \begin{bmatrix}
 -1.00 + 0.00j & -1.00 + -0.00j
\end{bmatrix}\]
\[K = \begin{bmatrix}
  1.00 &  2.00
\end{bmatrix}\]
\[ W = \left[\begin{matrix}\frac{1.0 i \omega}{- 1.0 \omega^{2} + 1.0 i \omega} & \frac{1.0 i \omega}{- 1.0 \omega^{2} + 1.0 i \omega} & 0\\- \frac{1.0}{- 1.0 \omega^{2} + 2.0 i \omega + 1.0} & \frac{1.0 i \omega}{- 1.0 \omega^{2} + 2.0 i \omega + 1.0} & 0\\- \frac{1.0 i \omega}{- 1.0 \omega^{2} + 2.0 i \omega + 1.0} & \frac{- 2.0 i \omega - 1.0}{- 1.0 \omega^{2} + 2.0 i \omega + 1.0} & 0\end{matrix}\right]\]
\[||W||_{H_2} = 1.7320508075688776\]
\[||W||_{H_\infty} = 1.8027749569092566 \]

\begin{figure}[ht!]
  \centering
  \includegraphics[width=\textwidth]{src/figs/task1_0_amps.jpg}
  \caption{АЧХ системы.}
  \label{fig:task1_0_amps}
\end{figure}

\begin{figure}[ht!]
  \centering
  \includegraphics[width=\textwidth]{src/figs/task1_0_zs.jpg}
  \caption{Регулируемый выход системы.}
  \label{fig:task1_0_zs}
\end{figure}

\begin{figure}[ht!]
  \centering
  \includegraphics[width=\textwidth]{src/figs/task1_0_sing.jpg}
  \caption{Сингулярные числы.}
  \label{fig:task1_0_sing}
\end{figure}


\FloatBarrier
\subsubsubsection{Вариант 2}
\[C_2 = \begin{bmatrix}
  0.00 &  0.00\\
  1.00 &  0.00\\
  0.00 &  0.00
\end{bmatrix}; D_2 = \begin{bmatrix}
  1.00\\
  0.00\\
  1.00
\end{bmatrix};\]
\[C_2^T D_2 = 0: True\]
\[D_2^T D_2 \text{ обратима}: True\]
\[spec(A-B_2 K) = \begin{bmatrix}
 -0.59 + 0.59j & -0.59 + -0.59j
\end{bmatrix}\]

\[K = \begin{bmatrix}
  0.71 &  1.19
\end{bmatrix}\]
\[ W = \left[\begin{matrix}- \frac{0.707106781186547 i \omega}{- 1.0 \omega^{2} + 1.18920711500272 i \omega + 0.707106781186547} & \frac{- 1.18920711500272 i \omega - 0.707106781186547}{- 1.0 \omega^{2} + 1.18920711500272 i \omega + 0.707106781186547} & 0\\\frac{1.0 i \omega + 1.18920711500272}{- 1.0 \omega^{2} + 1.18920711500272 i \omega + 0.707106781186547} & \frac{1.0}{- 1.0 \omega^{2} + 1.18920711500272 i \omega + 0.707106781186547} & 0\\- \frac{0.707106781186547 i \omega}{- 1.0 \omega^{2} + 1.18920711500272 i \omega + 0.707106781186547} & \frac{- 1.18920711500272 i \omega - 0.707106781186547}{- 1.0 \omega^{2} + 1.18920711500272 i \omega + 0.707106781186547} & 0\end{matrix}\right]\]
\[||W||_{H_2} = 2.014995548509443\]
\[||W||_{H_\infty} = 2.759749617001796 \]

\begin{figure}[ht!]
  \centering
  \includegraphics[width=\textwidth]{src/figs/task1_1_amps.jpg}
  \caption{АЧХ системы.}
  \label{fig:task1_1_amps}
\end{figure}

\begin{figure}[ht!]
  \centering
  \includegraphics[width=\textwidth]{src/figs/task1_1_zs.jpg}
  \caption{Регулируемый выход системы.}
  \label{fig:task1_1_zs}
\end{figure}

\begin{figure}[ht!]
  \centering
  \includegraphics[width=\textwidth]{src/figs/task1_1_sing.jpg}
  \caption{Сингулярные числы.}
  \label{fig:task1_1_sing}
\end{figure}

\FloatBarrier