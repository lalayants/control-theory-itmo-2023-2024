\subsection{Синтез \(H_\infty\)-регулятора по состоянию.}
Пусть \(||W||_{H_\infty} < \gamma\)

Уравнения ниже позволяют синтезировать такой регулятор.
\begin{equation}
    \begin{cases}
        A^TQ + QA +C_2^TC_2 - QB_2(D_2^TD_2)^{-1}B_2^TQ + \gamma^{-2}QB_1B_1^TQ=0 \\
        K = -(D_2^TD_2)^{-1}B_2^TQ
    \end{cases}
\end{equation}
Существует \(P > 0\) решение уравнения Рикатти, если:
\begin{enumerate}
  \item \(C_2^TD_2 = 0\) (\(C_1C_1^T = Q; D_2D_2^T = Q;\))
  \item \(D_2D_2^T\) -- обратима
  \item \((C_2, A)\) -- обнаруживаема
  \item \((A, B_2)\) -- стабилизируема
\end{enumerate}


\subsubsubsection{\(\gamma\) = 1.4}
\[spec(A-B_2 K) = [-0.64 -3.96]\]
\[K = \begin{bmatrix}
  2.54 &  4.61
\end{bmatrix}\]
\[W = \left[\begin{matrix}\frac{1.0 i \omega + 2}{- 1.0 \omega^{2} + 4.6 i \omega + 2.5} & \frac{1.0 i \omega + 1.0}{- 1.0 \omega^{2} + 4.6 i \omega + 2.5} & 0\\- \frac{2.5}{- 1.0 \omega^{2} + 4.61 i \omega + 2.5} & \frac{1.0 i \omega}{- 1.0 \omega^{2} + 4.61 i \omega + 2.5} & 0\\- \frac{2.5 i \omega}{- 1.0 \omega^{2} + 4.6 i \omega + 2.5} & \frac{- 4.61 i \omega - 2.5}{- 1.0 \omega^{2} + 4.6 i \omega + 2.5} & 0\end{matrix}\right]\]
\[||W||_{H_2} = 2.026622146045859\]
\[||W||_{H_\infty} = 1.359310155325042 \]

\begin{figure}[ht!]
    \centering
    \includegraphics[width=\textwidth]{src/figs/task3_0_amps.jpg}
    \caption{АЧХ системы.}
    \label{fig:task3_0_amps}
  \end{figure}
  
  \begin{figure}[ht!]
    \centering
    \includegraphics[width=\textwidth]{src/figs/task3_0_zs.jpg}
    \caption{Регулируемый выход системы.}
    \label{fig:task3_0_zs}
  \end{figure}
  
  \begin{figure}[ht!]
    \centering
    \includegraphics[width=\textwidth]{src/figs/task3_0_sing.jpg}
    \caption{Сингулярные числы.}
    \label{fig:task3_0_sing}
  \end{figure}

\FloatBarrier

\subsubsubsection{\(\gamma\) = 2}
\[spec(A-B_2 K) = [-0.71 -1.92]\]
\[K = \begin{bmatrix}
  1.37 &  2.63
\end{bmatrix}\]
\[ W = \left[\begin{matrix}\frac{1.0 i \omega + 1.2}{- 1.0 \omega^{2} + 2.6 i \omega + 1.3} & \frac{1.0 i \omega + 1.0}{- 1.0 \omega^{2} + 2.6 i \omega + 1.3} & 0\\- \frac{1.3}{- 1.0 \omega^{2} + 2.6 i \omega + 1.3} & \frac{1.0 i \omega}{- 1.0 \omega^{2} + 2.6 i \omega + 1.3} & 0\\- \frac{1.3 i \omega}{- 1.0 \omega^{2} + 2.6 i \omega + 1.3} & \frac{- 2.6 i \omega - 1.3}{- 1.0 \omega^{2} + 2.6 i \omega + 1.3} & 0\end{matrix}\right]\]
\[||W||_{H_2} = 1.7668518255958938\]
\[||W||_{H_\infty} = 1.5881438250859183 \]

\begin{figure}[ht!]
    \centering
    \includegraphics[width=\textwidth]{src/figs/task3_1_amps.jpg}
    \caption{АЧХ системы.}
    \label{fig:task3_1_amps}
  \end{figure}
  
  \begin{figure}[ht!]
    \centering
    \includegraphics[width=\textwidth]{src/figs/task3_1_zs.jpg}
    \caption{Регулируемый выход системы.}
    \label{fig:task3_1_zs}
  \end{figure}
  
  \begin{figure}[ht!]
    \centering
    \includegraphics[width=\textwidth]{src/figs/task3_1_sing.jpg}
    \caption{Сингулярные числы.}
    \label{fig:task3_1_sing}
  \end{figure}

\FloatBarrier

\subsubsubsection{\(\gamma\) = 10}
\[spec(A-B_2 K) = [-0.92 -1.1 ]\]
\[K = \begin{bmatrix}
  1.01 &  2.02
\end{bmatrix}\]
\[W = \left[\begin{matrix}\frac{1.0 i \omega + 1}{- 1.0 \omega^{2} + 2 i \omega + 1} & \frac{1.0 i \omega + 1.0}{- 1.0 \omega^{2} + 2 i \omega + 1} & 0\\- \frac{1}{- 1.0 \omega^{2} + 2 i \omega + 1} & \frac{1.0 i \omega}{- 1.0 \omega^{2} + 2 i \omega + 1} & 0\\- \frac{1 i \omega}{- 1.0 \omega^{2} + 2 i \omega + 1} & \frac{- 2 i \omega - 1}{- 1.0 \omega^{2} + 2 i \omega + 1} & 0\end{matrix}\right]\]
\[||W||_{H_2} = 1.7320883081147758\]
\[||W||_{H_\infty} = 1.79431913434723 \]

\begin{figure}[ht!]
    \centering
    \includegraphics[width=\textwidth]{src/figs/task3_2_amps.jpg}
    \caption{АЧХ системы.}
    \label{fig:task3_2_amps}
  \end{figure}
  
  \begin{figure}[ht!]
    \centering
    \includegraphics[width=\textwidth]{src/figs/task3_2_zs.jpg}
    \caption{Регулируемый выход системы.}
    \label{fig:task3_2_zs}
  \end{figure}
  
  \begin{figure}[ht!]
    \centering
    \includegraphics[width=\textwidth]{src/figs/task3_2_sing.jpg}
    \caption{Сингулярные числы.}
    \label{fig:task3_2_sing}
  \end{figure}

\FloatBarrier