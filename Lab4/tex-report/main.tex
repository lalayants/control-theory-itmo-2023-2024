\documentclass[16pt]{article}

\usepackage{report}

\usepackage[utf8]{inputenc} % allow utf-8 input
\usepackage[english, russian]{babel}
\usepackage[T1]{fontenc}    % use 8-bit T1 fonts
\usepackage[colorlinks=true, linkcolor=black, citecolor=blue, urlcolor=blue]{hyperref}       % hyperlinks
\usepackage{url}            
\usepackage{booktabs}       
\usepackage{amsfonts}       
\usepackage{nicefrac}      
\usepackage{microtype} 
\usepackage{graphicx}
\usepackage{natbib}
\usepackage{mathrsfs}
\usepackage{doi}
\usepackage{mathtools}
\usepackage{graphicx}
\usepackage{listings}
\usepackage{pythonhighlight}
\usepackage{mathtools}
\graphicspath{ {./figs/} }


\setcitestyle{aysep={,}}



\title{ЛР \textnumero 4 <<Астатизмы>>}

\author{
Студент \\
Кирилл Лалаянц\\
R33352\\
336700\\
Вариант - 6\\
\\
Преподаватель\\
Пашенко А.В. \\
\\
\\
Факультет Систем Управления и Робототехники\\
\\
ИТМО\\
}

% Uncomment to remove the date
\date{10.11.2023}

% Uncomment to override  the `A preprint' in the header
\renewcommand{\headeright}{ЛР \textnumero 4 <<Астатизмы>>}
\renewcommand{\undertitle}{Отчет}
\renewcommand{\shorttitle}{}


\begin{document}
\maketitle
\newpage
\tableofcontents
\thispagestyle{empty}

\newpage
\setcounter{page}{1}
\section{Вводные данные}
\subsection{Цель работы}
В этой работе будет проведенно исследование следующих вопросов:
\begin{itemize}
    \item Астатизмы.
    \item Принцип внутренней модели.
    \item Идеальное и реальное дифференцирующие звенья
\end{itemize} 

\subsection{Воспроизведение результатов}
Все результаты можно воспроизвести с помощью \href{https://github.com/lalayants/control-theory-itmo-2023}{репозитория}.

\newpage
\section{Выполнение работы}
\label{sec:headings}


\subsection{Задание 1. Задача стабилизации с идеальным дифференцирующим звеном.}

\subsubsection{Теория}
В этом задании будет проведена симуляция системы с ПД регулятором, используя дифференциальное звено, для open- и closed-loop систем.

Хотим получить ДУ системы вида:
\[\ddot{y} + a_1\dot{y} + a_0y = u\]
Cначала получим характеристическое уравнение для частного решения вида:
\[\ddot{y} + a_1\dot{y} + a_0y = 0\]
Имеем два корня -- \(\lambda_1\) и \(\lambda_2\). Тогда:
\[ (p - \lambda_1)(p-\lambda_2) = p^2 - (\lambda_1 + \lambda_2)p + \lambda_1\lambda_2 = 0,\]
умножив которое на \(y\) получаем:
\[\ddot{y} - (\lambda_1 + \lambda_2)\dot{y} + \lambda_1\lambda_2y = 0.\]
Для полюсов -1 и 2 получаем:
\[\ddot{y} - \dot{y} - 2y = 0.\]

Возьмем \[u = k_0e + k_1\dot{e} = k_0(0 - y) + k_1\dot{(0 - y)} = -k_0y - k_1\dot{y},\] тогда наша система примет вид:
\[\ddot{y} - \dot{y} - 2y =  -k_0y - k_1\dot(y)\]
\[\ddot{y} + (- 1 + k_1)\dot{y} + (-2 + k_0)y = 0\]
Откуда по следствию из критерий Гурвица видно, что:
\begin{equation}
    \begin{cases}
        k_1 > 1\\
        k_0 > 2
    \end{cases}\,.
\end{equation}
Пусть \(k_1 = 4; k_0 = 1000\).

\subsubsection{Результаты}
Ожидаемо, замкнутая система успешно свела ошибку к 0, а открытая -- нет.
\begin{figure}[h!]
    \centering
    \includegraphics[width=\textwidth]{task1}
    \caption{Результат выполнения первого задания.}
    \label{fig:fig1}
\end{figure}

\pagebreak

\subsection{Задание 2. Задача стабилизации с реальным дифференцирующим звеном.}

\subsubsection{Теория}
В этом задании будет проведена симуляция системы с ПД регулятором, используя реальное дифференциальное звено.
\[ W_{rd} = \frac{s}{Ts + 1}\]
Так же исследован параметр Т на предмет устойчивости.

Для этого cначала получим передаточную closed-loop системы:
\[ W_{ol} = (k_0 + k_1W_{rd}) W_{sys}\]
\[ W_{cl} = \frac{W_{ol}}{1 + W_{ol}}\]

Взяв систему и коэффициенты из матричного критерия Гурвица получаем, что система устойчива при \(0 < T < 0.533\)

\subsubsection{Результаты}
Заметно, что параметр \(Т=10^{-3}\) уже достаточно мал, и отличий с \(Т=10^{-5}\) нет. Значение границы устойчивости экспериментально подтвердилось -- система устойчива по Ляпунову при этом значении, а при значении больше -- неустойчива. Так же видно, что чем меньше Т -- тем быстрее переходный процесс.

При T стремящемся к 0, поведение системы близко к поведению с идеальным дифференцирующим звеном.
\begin{figure}[h!]
    \centering
    \includegraphics[width=\textwidth]{task2}
    \caption{Результат выполнения второго задания.}
    \label{fig:fig2}
\end{figure}

\pagebreak

\subsection{Задание 3. Исследование влияния шума.}

\subsubsection{Теория}
В этом задании будет проведено исследование влияния шума на конечный результат.
\subsubsection{Результаты}
Четко видно (рис. \ref{fig:fig3}), что ошибка при использовании идеального звена прямопропорциональна шуму. Однако, более важно тут то, что на грубую сходимость системы это никак не влияет и в начале графики выглядят идентично. Разница становится заметна только при значениях ошибки уже близким к 0 -- система с большой погрешностью в заметно более широкой окрестности цели.
\begin{figure}[h!]
    \centering
    \includegraphics[width=\textwidth]{task3}
    \caption{Результат выполнения третьего задания для идеальной системы.}
    \label{fig:fig3}
\end{figure}

Проведем исследование влияния шума на системы с реальными дифференцирующими звеньями.
Заметно (рис. \ref{fig:fig33}), что при большем Т сходимость дольше и колебания сильнее. Это в целом следует из предыдущего пунтка, но еще услививается шумами.
\begin{figure}[h!]
    \centering
    \includegraphics[width=\textwidth]{task3_ts}
    \caption{Результат выполнения третьего задания для идеальной системы.}
    \label{fig:fig33}
\end{figure}
\pagebreak

\subsection{Задание 4. Задача слежения для системы с астатизмом нулевого порядка.}

\subsubsection{Теория}
В этом задании будет проведено исследование слежения системы с астатизмом нулевого порядка  при различных входных воздействиях. 
\subsubsection{Результаты}
На графике представлены поведение системы при различных коэффициентах \(k\). Заметно, что при константном воздействии (рис. \ref{fig:fig4}) он уменьшает ошибку. Ее предельное значение было посчитано через предельную теорему и представлено в легенде.
\begin{figure}[h!]
    \centering
    \includegraphics[width=\textwidth]{task4_const}
    \caption{Система с астатизмом 0. Константное воздействие.}
    \label{fig:fig4}
\end{figure}

На графике (рис. \ref{fig:fig5}) представлено поведение системы при линейном воздействии. Графики расходятся -- ошибка стремится к бесконечности. 
\begin{figure}[h!]
    \centering
    \includegraphics[width=\textwidth]{task4_lin}
    \caption{Система с астатизмом 0. Линейное воздействие.}
    \label{fig:fig5}
\end{figure}

На графике (рис. \ref{fig:fig6}) представлено поведение системы при переодическом воздействии. Ошибка стремится к 0. 
\begin{figure}[h!]
    \centering
    \includegraphics[width=\textwidth]{task4_sin}
    \caption{Система с астатизмом 0. Переодическое воздействие.}
    \label{fig:fig6}
\end{figure}
\pagebreak

\subsection{Задание 5. Задача слежения для системы с астатизмом первого порядка.}

\subsubsection{Теория}
Задание аналогично предыдущему, только на этот раз ПИ регулятор, который повышает порядок астатизма.
\subsubsection{Результаты}
Сначала было проведенно влияние П коэффициента.
Заметно, что при константном воздействии (рис. \ref{fig:fig7}) его влияние уже не столь очевидно. Так же заметен вклад И части -- ошибка всех графиков сходится к 0.
\begin{figure}[h!]
    \centering
    \includegraphics[width=\textwidth]{task5_k0_const}
    \caption{Система с астатизмом 0. Константное воздействие.}
    \label{fig:fig7}
\end{figure}
При линейном воздействии (рис. \ref{fig:fig8}) ошибка никак не зависит от П коэффициента.
\begin{figure}[h!]
    \centering
    \includegraphics[width=\textwidth]{task5_k0_lin}
    \caption{Система с астатизмом 0. Линейное воздействие.}
    \label{fig:fig8}
\end{figure}

При переодическом воздействии (рис. \ref{fig:fig9}) влияние коэффициента П определить крайне тяжело.
\begin{figure}[h!]
    \centering
    \includegraphics[width=\textwidth]{task5_k0_sin}
    \caption{Система с астатизмом 0. Переодическое воздействие.}
    \label{fig:fig9}
\end{figure}


При константном воздействии (рис. \ref{fig:fig10}) влияние И очень заметно. Он ускорят время переходного процесса, но при этом вызывает перерегулирование.
\begin{figure}[h!]
    \centering
    \includegraphics[width=\textwidth]{task5_k1_const}
    \caption{Система с астатизмом 0. Константное воздействие.}
    \label{fig:fig10}
\end{figure}
При линейном воздействии (рис. \ref{fig:fig11}) ошибка обратно пропорциональна И.
\begin{figure}[h!]
    \centering
    \includegraphics[width=\textwidth]{task5_k1_lin}
    \caption{Система с астатизмом 0. Линейное воздействие.}
    \label{fig:fig11}
\end{figure}

При переодическом воздействии (рис. \ref{fig:fig12}) влияние коэффициента И определить крайне тяжело.
\begin{figure}[h!]
    \centering
    \includegraphics[width=\textwidth]{task5_k1_sin}
    \caption{Система с астатизмом 0. Переодическое воздействие.}
    \label{fig:fig12}
\end{figure}
\pagebreak
\newpage


\subsection{Задание 6. Исследование линейной системы замкнутой регулятором общего вида.}

\subsubsection{Теория}
В этом задании был протестирован принцип внутренней модели и получена управляемая система.
\[u = sin(3t)cos(2t)\]
\[U = \frac{N_g}{D_g} = \frac{3(s^2+5)}{(s^2+1)(s^2+25)}\]
\[W = \frac{N}{D} = \frac{1}{s^2}\]
\[W_r = \frac{N_r}{D_r}\]
\[E = \frac{D_rD}{D_rD + NrN} \frac{N_g}{D_g} = \frac{D_rs^2}{D_rs^2 + Nr} \frac{3(s^2+5)}{(s^2+1)(s^2+25)} \]

Чтобы сократить положительные полюса \(D_g\),пусть \(D_r = D_g(s+r)\)
\[D_r = (s^2+1)(s^2+25)(s+r)\]

Тогда получаем:
\[E = \frac{(s+r)s^2}{(s^2+1)(s^2+25)(s+r)s^2 + Nr} \frac{3(s^2+5)}{1} \]
Пусть \(N_r = s^2(a_3s^3 + a_2s^2 + a_1s^1 + a_0)\), тогда:
\[E = \frac{(s+r)}{(s^2+1)(s^2+25)(s+r) + (a_3s^3 + a_2s^2 + a_1s^1 + a_0)} \frac{3(s^2+5)}{1} \]

Остается лишь принять любые устраивающие нас корни (на графике используется набор от -1 до -5), раскрыть скобки, привеси подобные при степенях s, откуда получим:
\begin{equation}
    \begin{cases}
        a_0 > -255\\
        a_1 > 249\\
        a_2 = -165\\
        a_3 = 59\\
        r = 15
    \end{cases}\,.
\end{equation}


\subsubsection{Результаты}
Благодаря принципу замкнутой модели был синтезирован регулятор для управления системой. Ошибка сходится к 0.
\begin{figure}[h!]
    \centering
    \includegraphics[width=\textwidth]{task6}
    \caption{Результат синтеза регулятора.}
    \label{fig:fig13}
\end{figure}

\pagebreak
\pagebreak
\section{Заключение}
В этой работе было проведенно исследование следующих вопросов:
\begin{itemize}
    \item Астатизмы.
    \item Принцип внутренней модели.
    \item Идеальное и реальное дифференцирующие звенья
\end{itemize} 
\subsection{Выводы}
\begin{enumerate}
   \item На практике изучено реальное и идеальное ДЗ.
   \item Проверена работа систем с разными степенями астатизмов. 
   \item Проверено влияние коэффициентов регулятора на поведение системы.
   \item Синтезирован регулятор методом замкнутой модели.
\end{enumerate}

\end{document}