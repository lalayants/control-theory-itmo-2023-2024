\documentclass[16pt]{article}

\usepackage{report}

\usepackage[utf8]{inputenc} % allow utf-8 input
\usepackage[english, russian]{babel}
\usepackage[T1]{fontenc}    % use 8-bit T1 fonts
\usepackage[colorlinks=true, linkcolor=black, citecolor=blue, urlcolor=blue]{hyperref}       % hyperlinks
\usepackage{url}            
\usepackage{booktabs}       
\usepackage{amsfonts}       
\usepackage{nicefrac}      
\usepackage{microtype} 
\usepackage{graphicx}
\usepackage{natbib}
\usepackage{mathrsfs}
\usepackage{doi}
\usepackage{mathtools}
\usepackage{graphicx}
\usepackage{listings}
\usepackage{pythonhighlight}
\usepackage{mathtools}
\usepackage{amsmath}
\graphicspath{ {./figs/} }


\setcitestyle{aysep={,}}



\title{ЛР \textnumero 6 <<Критерий Найквиста и системы с запаздыванием>>}

\author{
Студент \\
Кирилл Лалаянц\\
R33352\\
336700\\
Вариант - 6\\
\\
Преподаватель\\
Пашенко А.В. \\
\\
\\
Факультет Систем Управления и Робототехники\\
\\
ИТМО\\
}

% Uncomment to remove the date
\date{20.11.2023}

% Uncomment to override  the `A preprint' in the header
\renewcommand{\headeright}{ЛР \textnumero 6 <<Критерий Найквиста и системы с запаздыванием>>}
\renewcommand{\undertitle}{Отчет}
\renewcommand{\shorttitle}{}


\begin{document}
\maketitle
\newpage
\tableofcontents
\thispagestyle{empty}

\newpage
\setcounter{page}{1}

\section{Вводные данные}
\subsection{Цель работы}
В этой работе будет проведенно исследование следующих вопросов:
\begin{itemize}
    \item Типовые динамические звенья.
    \item АЧХ, ЛЧХ, ФЛЧХ.
\end{itemize} 

\subsection{Воспроизведение результатов}
Все результаты можно воспроизвести с помощью \href{https://github.com/lalayants/control-theory-itmo-2023}{репозитория}.

\subsection{Обозначения}
Здесь и далее \(\theta(t)\) -- функция Хэвисайда.

\newpage
\section{Выполнение работы}
\label{sec:headings}


\subsection{Brushed DC motor 2.0.}

\subsubsection{Теория}
\subsubsection{Результаты}
На графике (Рис. \ref*{fig:fig1}) представлен результат выполнения задания. Как видно, практический результат совпал с теоретическим.
\begin{figure}[h!]
    \centering
    \includegraphics[width=\textwidth]{1_2}
    \caption{Результат выполнения задания.}
    \label{fig:fig1}
\end{figure}




\newpage
\section{Заключение}
В этой работе было проведенно исследование следующих вопросов:
\begin{itemize}
    \item Типовые динамические звенья.
    \item АЧХ, ЛЧХ, ФЛЧХ.
\end{itemize} 
\subsection{Выводы}
\begin{enumerate}
   \item На примерах разных систем изучены типовые динамические звенья. Теоретические результаты сошлись с практическими.
\end{enumerate}

\end{document}