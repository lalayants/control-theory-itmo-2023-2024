\section{ГЛАВА 5. СТАБИЛИЗАЦИЯ МАЯТНИКА: LQR И ФИЛЬТР КАЛМАНА}

\subsection{Синтез линейно-квадратичного регулятора}
В этом задании выводится регулятор заданной степени устойчивости для системы:
\[
        \begin{cases}
                \text{Объект управления: }\dot{x} = A x + Bu \\
                \text{Регулятор: }u = -K x \\
        \end{cases} \rightarrow
        \dot{x} = A x - BKx = (A-BK)x
\]
LQR позволяет оптимизировать критерий качества:
\[J = \int_0^\infty (x^T Q x + u^T R u)dt \]
Выбор cотношения матриц \(Q\) и \(R\) позволяет управлять временем сходимости и величиной подаваего управления: чем больше \(\frac{Q}{R}\), тем больше управление и быстрее сходимость.

\(K\) получается решением следующих уравнений:
\[
\begin{cases}
    A^T P + P A + Q - PBR^{-1}B^TP = 0\\
    K = -R^{-1} B^T P \\
\end{cases}
\]

Теоретический минимум критерия качества:
\[J_{min} = x_0^T P x_0\]
\FloatBarrier

\subsection{Исследование линейно-квадратичного регулятора}
Все работает отлично, при небольшом отклонении от горизонтали в начальный момент.
\[Q = 0.1; R = 10.0; K_0 = \begin{bmatrix}
        -0.10 & -0.54 &  25.09 &  7.61
       \end{bmatrix}\]
       \[eig(A+BK_0) = \begin{bmatrix}
        -0.22 + 0.21j & -0.22 + -0.21j & -3.37 + 0.00j & -3.27 + 0.00j
       \end{bmatrix}\]
       \[Q = 1.0; R = 1.0; K_1 = \begin{bmatrix}
        -1.00 & -2.40 &  34.91 &  10.76
       \end{bmatrix}\]
       \[eig(A+BK_1) = \begin{bmatrix}
        -3.86 + 0.00j & -2.87 + 0.00j & -0.81 + 0.50j & -0.81 + -0.50j
       \end{bmatrix}\]
       \[Q = 10.0; R = 0.1; K_2 = \begin{bmatrix}
        -10.00 & -17.70 &  117.61 &  38.08
       \end{bmatrix}\]
       \[eig(A+BK_2) = \begin{bmatrix}
        -14.54 + 0.00j & -2.42 + 1.00j & -2.42 + -1.00j & -1.01 + 0.00j
       \end{bmatrix}\]

\begin{figure}[ht!]
        \centering
        \includegraphics[width=\textwidth]{src/figs/task5_states.jpg}
        \caption{Динамика компонент системы.}
        \label{fig:task5_states}
\end{figure}

\begin{figure}[ht!]
        \centering
        \includegraphics[width=\textwidth]{src/figs/task5_us.jpg}
        \caption{Управление.}
        \label{fig:task5_u}
\end{figure}
\FloatBarrier

\subsection{Синтез фильтра Калмана}
\[      
        \text{Объект управления: }
        \begin{cases}
                \dot{x} = A x + Bu + f, \text{ \(f\) -- внешнее возмущение}\\
                y = Cx + \xi, \text{ \(\xi\) -- помеха измерений}
        \end{cases} 
\]
Матрицы \(Q\) и \(R\) обозначают, насколько сильно мы оцениваем влиянию \(f\) и \(\xi\).

\(L\) получается решением следующих уравнений:
\[
\begin{cases}
    A P + P A^T + Q - PC^TR^{-1}CP = 0\\
    L = -P C^T R^{-1}\\
\end{cases}
\]
Ниже приведена ошибка слежения за системой из прошлог пункта.
\[Q = 0.01; R = 0.01; L = \begin{bmatrix}
        1.76 &  0.37\\
        1.11 &  1.88\\
        0.37 &  6.69\\
        1.26 &  21.97
      \end{bmatrix}\]
\begin{figure}[ht!]
        \centering
        \includegraphics[width=\textwidth]{src/figs/task5_3.jpg}
        \caption{Ошибка.}
        \label{fig:task5_3}
\end{figure}
\FloatBarrier

\subsection{ LQG для линейной модели}
В этом задании выводится наблюдатель регулятор для системы:
\[
        \begin{cases}
                \dot{x} = A x + B  K \hat{x} + f \\
                y = Cx + DK\hat{x} + \xi \\
                \dot{\hat{x}} = A \hat{x} + B  K \hat{x} + L(\hat{y} - y) \\
                \hat{y} = C \hat{x} + D K \hat{x} \\
                \hat{x} = x - e \\
        \end{cases} \rightarrow
        \begin{cases}
            \begin{bmatrix} 
                \dot{x} \\
                \dot{e}
            \end{bmatrix} = 
            A_{new}
            \begin{bmatrix} 
              x \\
              e
            \end{bmatrix} 
          + B_{new} 
          \begin{bmatrix} 
            f \\
            \xi
          \end{bmatrix} 
            \\
            A_{new} = 
            \begin{bmatrix} 
                A + BK & -BK\\
                0 & A + LC
            \end{bmatrix} \\
            B_{new} = 
            \begin{bmatrix} 
                I & 0\\
                I & L
            \end{bmatrix} \in R^{2n \times (n + m)}
         \end{cases}
\]
\[Q = 0.01; R = 0.01; L = \begin{bmatrix}
        -1.76 & -0.37\\
        -1.11 & -1.88\\
        -0.37 & -6.69\\
        -1.26 & -21.97
       \end{bmatrix}\]
       \[spec(A+LC) = \begin{bmatrix}
        -0.87 + 0.50j & -0.87 + -0.50j & -2.88 + 0.00j & -3.84 + 0.00j
       \end{bmatrix}\]
       \[Q = 1; R = 1; K= \begin{bmatrix}
         1.00 &  2.40 & -34.91 & -10.76
       \end{bmatrix}\]
       \[spec(A + BK) = \begin{bmatrix}
        -3.86 + 0.00j & -2.87 + 0.00j & -0.81 + 0.50j & -0.81 + -0.50j
       \end{bmatrix}\]
\begin{figure}[ht!]
        \centering
        \includegraphics[width=\textwidth]{src/figs/task5_LQG_lin.jpg}
        \caption{Динамика системы и ошибки.}
        \label{fig:task5_4}
\end{figure}
\FloatBarrier

\subsection{ LQG для нелинейной модели}
\[Q = 0.01; R = 0.01; L = \begin{bmatrix}
        1.76 &  0.37\\
        1.11 &  1.88\\
        0.37 &  6.69\\
        1.26 &  21.97
      \end{bmatrix}\]
      \[Q = 0.01; R = 0.01; L = \begin{bmatrix}
       -1.76 & -0.37\\
       -1.11 & -1.88\\
       -0.37 & -6.69\\
       -1.26 & -21.97
      \end{bmatrix}\]
      \[spec(A+LC) = \begin{bmatrix}
       -0.87 + 0.50j & -0.87 + -0.50j & -2.88 + 0.00j & -3.84 + 0.00j
      \end{bmatrix}\]
      \[Q = 1; R = 1; K= \begin{bmatrix}
        1.00 &  2.40 & -34.91 & -10.76
      \end{bmatrix}\]
      \[spec(A + BK) = \begin{bmatrix}
       -3.86 + 0.00j & -2.87 + 0.00j & -0.81 + 0.50j & -0.81 + -0.50j
      \end{bmatrix}\]
\begin{figure}[ht!]
        \centering
        \includegraphics[width=\textwidth]{src/figs/task5_LQG_non_lin.jpg}
        \caption{Динамика системы и ошибки.}
        \label{fig:task5_5}
\end{figure}
\FloatBarrier