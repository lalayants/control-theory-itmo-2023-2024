\section{ГЛАВА 4. СТАБИЛИЗАЦИЯ МАЯТНИКА: РЕГУЛЯТОРЫ С ЗАДАННОЙ СТЕПЕНЬЮ УСТОЙЧИВОСТИ}

\subsection{Синтез регулятора по состоянию}
В этом задании выводится регулятор заданной степени устойчивости для системы:
\[
        \begin{cases}
                \text{Объект управления: }\dot{x} = A x + Bu \\
                \text{Регулятор: }u = K x \\
        \end{cases} \rightarrow
        \dot{x} = A x + BKx = (A+BK)x
\]
По сути, целью данного регулятора является изменение управляемых собственных чисел так, чтобы \(\forall \lambda \in \sigma(A): Re{\lambda} \leq \alpha\), где \(\alpha\) -- степень устойчивости.
Для этого используется LMI критерий экспоненциальной устойчивости:
\[ \exists Q \succ 0 , \alpha > 0 :  A^TQ + QA + 2 \alpha Q \preccurlyeq 0 \rightarrow
\begin{cases}
    \text{\(\forall x(0)\) А ассимптотически устойчива}\\
    \exists c :  ||x(t)|| \le  c e^{-\alpha t} ||x(0)|| \\
\end{cases}
\]

\[      \begin{cases}
                Q \succ 0\\
                A^TQ + QA + 2 \alpha Q \preccurlyeq 0 \\
        \end{cases} \xleftrightarrow{Q = P^{-1}}
        \begin{cases}
                P \succ 0\\
                PA^T + AP + 2 \alpha P \preccurlyeq 0\\
        \end{cases} 
\]
Подставим вместо матрицы \(A\) нашу систему:
\[P(A+BK)^T + (A+BK)P + 2 \alpha P \preccurlyeq 0\]
\[P A^T + P K^T B^T + AP + BKP + 2 \alpha P \preccurlyeq 0\]
\[
        \begin{cases}
                Y = KP\\
                PA^T + AP + 2 \alpha P + Y^T B^T + BY \preccurlyeq 0  \\
        \end{cases} 
\]

Тогда для подбора матрицы \(K\), чтобы задать система степень устойчивости \(\alpha\), достаточно решить через библиотеку CVX следующую систему с двумя неизвестнями:
\[
        \begin{cases}
                K = YP^{-1}\\
                P \succ 0 \\
                PA^T + AP + 2 \alpha P + Y^T B^T + BY \preccurlyeq 0  \\
        \end{cases} 
\]

На практике, довольно часто \(P\) -- необратима. Приходится использовать псевдообратную.

\[\alpha = 1\]
\[K = \begin{bmatrix}
  8.51 &  10.41 & -70.15 & -15.94
\end{bmatrix}\]
\[spec(A + B K) = \begin{bmatrix}
 -1.58 + 6.21j & -1.58 + -6.21j & -1.19 + 0.81j & -1.19 + -0.81j
\end{bmatrix}\]
Как видно на графиках ниже, невсегда нелинейная система устойчива. 
\begin{figure}[ht!]
    \centering
    \includegraphics[width=\textwidth]{src/figs/task4_1_0_0.5_0_0.jpg}
    \caption{Динамика системы.}
    \label{fig:task4_1_1}
\end{figure}

\begin{figure}[ht!]
    \centering
    \includegraphics[width=\textwidth]{src/figs/task4_1_0_0_0.1_0.jpg}
    \caption{Динамика системы.}
    \label{fig:task4_1_2}
\end{figure}

\begin{figure}[ht!]
    \centering
    \includegraphics[width=\textwidth]{src/figs/task4_1_0_0_0_0.3.jpg}
    \caption{Динамика системы.}
    \label{fig:task4_1_3}
\end{figure}

\begin{figure}[ht!]
    \centering
    \includegraphics[width=\textwidth]{src/figs/task4_1_0_0_0_1.jpg}
    \caption{Динамика системы.}
    \label{fig:task4_1_4}
\end{figure}

\begin{figure}[ht!]
    \centering
    \includegraphics[width=\textwidth]{src/figs/task4_1_0_0_1_0.jpg}
    \caption{Динамика системы.}
    \label{fig:task4_1_5}
\end{figure}

\begin{figure}[ht!]
    \centering
    \includegraphics[width=\textwidth]{src/figs/task4_1_0_1_0_0.jpg}
    \caption{Динамика системы.}
    \label{fig:task4_1_6}
\end{figure}

\FloatBarrier

\subsection{Исследование регулятора по состоянию}
Ниже приведена таблица сравнений при \(x(0) = [0, 0, 0.1, 0]^T\) для графиков ниже
\begin{center}
    \begin{tabular}{ c | c c c }
$\alpha$ & $\max x$ & $\max \varphi$ & $\max u$ \\
        $0.1$ & 0.27 & 0.1 & 3.7 \\
        $0.8$ & 0.24 & 0.1 & 5.7 \\
        $1.5$ & 0.21 & 0.1 & 11.6 \\
    \end{tabular}
\end{center}

\begin{figure}[ht!]
    \centering
    \includegraphics[width=\textwidth]{src/figs/task4_2_0_0_0.1_0.jpg}
    \caption{Динамика системы.}
    \label{fig:task4_2_1}
\end{figure}
\begin{figure}[ht!]
    \centering
    \includegraphics[width=\textwidth]{src/figs/task4_2_u_0_0_0.1_0.jpg}
    \caption{Динамика воздействия.}
    \label{fig:task4_2_2}
\end{figure}
\FloatBarrier

\subsection{Синтез регулятора по состоянию с ограничением на управление}

В этом задании выводится ограничение на управление \(||u(t)|| \leq \mu\). Тогда система уравнений принимает вид: 
\[
        \begin{cases}
                \begin{bmatrix}
                    P &  x_0\\
                    x_0^T &  1 \\
                \end{bmatrix} \succ 0 \\
                \\
                \begin{bmatrix}
                    P &  Y^T\\
                    Y &  \mu^2I \\
                \end{bmatrix} \succ 0 \\
                P \succ 0 \\
                PA^T + AP + 2 \alpha P + Y^T B^T + BY \preccurlyeq 0  \\
                K = YP^{-1}\\
        \end{cases} 
\]
\begin{figure}[ht!]
    \centering
    \includegraphics[width=\textwidth]{src/figs/task4_3_0.1.jpg}
    \caption{Динамика системы}
    \label{fig:task4_3_0.1}
\end{figure}
\begin{figure}[ht!]
    \centering
    \includegraphics[width=\textwidth]{src/figs/task4_3_u_0.1.jpg}
    \caption{Динамика воздействия}
    \label{fig:task4_3_u_0.1}
\end{figure}

\begin{figure}[ht!]
    \centering
    \includegraphics[width=\textwidth]{src/figs/task4_3_0.5.jpg}
    \caption{Динамика системы}
    \label{fig:task4_3_0.5}
\end{figure}
\begin{figure}[ht!]
    \centering
    \includegraphics[width=\textwidth]{src/figs/task4_3_u_0.5.jpg}
    \caption{Динамика воздействия}
    \label{fig:task4_3_u_0.5}
\end{figure}

\begin{figure}[ht!]
    \centering
    \includegraphics[width=\textwidth]{src/figs/task4_3_1.jpg}
    \caption{Динамика системы}
    \label{fig:task4_3_1}
\end{figure}
\begin{figure}[ht!]
    \centering
    \includegraphics[width=\textwidth]{src/figs/task4_3_u_1.jpg}
    \caption{Динамика воздействия}
    \label{fig:task4_3_u_1}
\end{figure}
\FloatBarrier

\subsection{Синтез наблюдателя}

В этом задании выводится наблюдатель заданной степени устойчивости для системы:
\[
        \begin{cases}
                \dot{x} = A x \\
                y = C x \\
                \dot{\hat{x}} = A \hat{x} + L(\hat{y} - y) \\
                \hat{y} = C \hat{x}
        \end{cases} \rightarrow
        \begin{cases}
            \dot{\hat{x}} = A \hat{x} + L(C \hat{x} - y) = (A + LC )\hat{x} - Ly \\
            \dot{e} = (A + LC)e
            
    \end{cases}
\]

Для этого достаточно решить систему:
\[
        \begin{cases}
                L = Q^{-1}Y\\
                Q \succ 0 \\
                A^TQ + QA + 2 \alpha Q + C^T Y^T + YC \preccurlyeq 0  \\
        \end{cases} 
\]
\[ \alpha = 1\]
\[L = \begin{bmatrix}
 -4.48 & -0.63\\
 -8.92 & -2.11\\
  0.63 & -4.48\\
  1.11 & -19.92
\end{bmatrix}\]
\[spec(A + LC) = \begin{bmatrix}
 -2.31 + 2.32j & -2.31 + -2.32j & -2.17 + 1.69j & -2.17 + -1.69j
\end{bmatrix}\]
\begin{figure}[ht!]
    \centering
    \includegraphics[width=\textwidth]{src/figs/task4_4_0.0_0.5.jpg}
    \caption{Динамика ошибки наблюдателя}
    \label{fig:task4_4_1}
\end{figure}

\begin{figure}[ht!]
    \centering
    \includegraphics[width=\textwidth]{src/figs/task4_4_0.1_0.1.jpg}
    \caption{Динамика ошибки наблюдателя}
    \label{fig:task4_4_2}
\end{figure}

\begin{figure}[ht!]
    \centering
    \includegraphics[width=\textwidth]{src/figs/task4_4_1_0.jpg}
    \caption{Динамика ошибки наблюдателя}
    \label{fig:task4_4_3}
\end{figure}
\FloatBarrier

\subsection{Синтез регулятора по выходу}
В этом задании выводится наблюдатель регулятор для системы. Так бы это выглядело для линейной системы.
\[K = \begin{bmatrix}
    8.51 &  10.41 & -70.15 & -15.94
  \end{bmatrix}\]
  \[spec(A + B K) = \begin{bmatrix}
   -1.58 + 6.21j & -1.58 + -6.21j & -1.19 + 0.81j & -1.19 + -0.81j
  \end{bmatrix}\]
  \[L = \begin{bmatrix}
   -4.48 & -0.63\\
   -8.92 & -2.11\\
    0.63 & -4.48\\
    1.11 & -19.92
  \end{bmatrix}\]
  \[spec(A + L C) = \begin{bmatrix}
   -2.31 + 2.32j & -2.31 + -2.32j & -2.17 + 1.69j & -2.17 + -1.69j
  \end{bmatrix}\]
  \[K = \begin{bmatrix}
    148.44 &  90.81 & -348.09 & -106.21
  \end{bmatrix}\]
  \[spec(A + B K) = \begin{bmatrix}
   -4.30 + 9.96j & -4.30 + -9.96j & -3.40 + 1.03j & -3.40 + -1.03j
  \end{bmatrix}\]
  \[L = \begin{bmatrix}
   -11.59 & -0.29\\
   -53.46 & -2.33\\
    0.29 & -11.59\\
    1.33 & -64.46
  \end{bmatrix}\]
  \[spec(A + L C) = \begin{bmatrix}
   -5.84 + 4.61j & -5.84 + -4.61j & -5.75 + 4.31j & -5.75 + -4.31j
  \end{bmatrix}\]
  

На графике видно, что регулятор справился и с нелинейной системой.
\begin{figure}[ht!]
    \centering
    \includegraphics[width=\textwidth]{src/figs/task4_5.jpg}
    \caption{Динамика системы}
    \label{fig:task4_5_new}
\end{figure}
\FloatBarrier

\FloatBarrier