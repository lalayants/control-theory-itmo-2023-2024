\section{ГЛАВА 1. ПОСТРОЕНИЕ МАТЕМАТИЧЕСКОЙ МОДЕЛИ ОБЪЕКТА}

\subsection{Вывод уравнений}

\begin{figure}[h]
  \centering
  \includegraphics[width=300px]{src/figs/pendulum.png}
  \caption{\label{fig:task1_1}Перевернутый маятник на тележке.}
\end{figure}

Рассмотрим систему перевернутого маятника на тележке (рис. \ref{fig:task1_1}). Введем следущие обозначения физических величин:
\begin{itemize}
  \item $x$ -- линейная координата тележки;
  \item $\dot{x}$ -- линейная скорость тележки;
  \item $\varphi$ -- угол отклонения маятника от вертикали;
  \item $\dot{\varphi}$ -- угловая скорость маятника;
  \item $f$ -- вращающий внешний момент, действующий на маятник;
  \item $u$ -- сила действующая на тележку;
  \item $M, m$ --  массы тележки и маятника соответственно;
  \item $l$ -- длина маятника.
\end{itemize}

В качестве вектора состояния \(s = \begin{bmatrix} x & \dot{x} & \varphi & \dot{\varphi}\end{bmatrix}^T\). 
В роли управляющего воздействия примем $u$, в роли внешнего возмущения -- $f$.

Измеряемыми сигналами \(y = \begin{bmatrix} x & \varphi \end{bmatrix}^T\).

Для вывода математической модели данной физической системы воспользуемся уравнениями Лагранжа \(L = T - V\), где где $T$ -- кинетическая энергия системы, $V$ -- потенциальная.
\[T = T_p + T_c\]
\[
T(t) = M\frac{\dot{x}^2}{2} + m\frac{(\frac{d}{dt}(l\cos(\varphi)))^2 + (-\frac{d}{dt}(l\sin(\varphi)) + \dot{x})^2}{2}
\]

\[V = V_p + V_c = Mgh_c + mgl \cos{\varphi} =  const +  mgl \cos{\varphi}\]
\[
  L = M\frac{\dot{x}^2}{2} + m\frac{(\frac{d}{dt}(l\cos(\varphi)))^2 + (-\frac{d}{dt}(l\sin(\varphi)) + \dot{x})^2}{2} - const -  mgl \cos{\varphi}
\]
\[
  L = M\frac{\dot{x}^2}{2} + m\frac{\dot x^2 - 2l \dot x \dot \varphi \cos \varphi + l^2 \dot{\varphi^2}}{2} - const -  mgl \cos{\varphi}
\]
\begin{equation} \label{eq:2}
  \begin{cases}
      \frac{d}{dt}\frac{\partial L}{\partial \dot{x}} - \frac{\partial L}{\partial x} = u \\
      \frac{d}{dt}\frac{\partial L}{\partial \dot{\varphi}} - \frac{\partial L}{\partial \varphi} = f 
  \end{cases},
\end{equation}

Подставив выражение для $L$ в уравнения \ref{eq:2}, получим уравнения математической модели системы:
\begin{equation} \label{eq:4}
  \begin{cases}
      (M+m)\ddot{x} + ml(\sin(\varphi)\dot{\varphi}^2 - \cos(\varphi)\ddot{\varphi}) = u \\
      ml^2\ddot{\varphi} - ml\ddot{x}\cos{\varphi} - mgl\sin(\varphi)= f
  \end{cases}
\end{equation}

Тогда, выразив $\ddot{a}$ и $\ddot{\varphi}$:
\begin{equation} \label{eq:5}
  \begin{cases}
      \ddot x = -\frac{ml}{M+m}\sin(\varphi)\dot{\varphi}^2 + \frac{ml}{M+m}\cos(\varphi)\ddot{\varphi} + \frac{1}{M+m}u \\
      \ddot \varphi = \frac{1}{l}\ddot x \cos(\varphi) + \frac{g}{l}\sin(\varphi) + \frac{1}{ml^2}f
  \end{cases}
\end{equation}
Решив данную систему уравнений \ref{eq:5} относительно $\ddot a$ и $\ddot \varphi$:
\begin{equation} \label{eq:6}
  \begin{cases}
      \ddot x = \frac{1}{M + m\sin(\varphi)^2}( -ml\sin(\varphi)\dot{\varphi}^2 + mg\cos(\varphi)\sin(\varphi) + \frac{\cos(\varphi)}{l}f + u ) \\
      \ddot \varphi = \frac{1}{M + m\sin(\varphi)^2}( -m\sin(\varphi)\cos(\varphi)\dot{\varphi}^2 + \frac{(M+m)g}{l}\sin(\varphi) + \frac{M+m}{ml^2}f + \frac{\cos(\varphi)}{l}u )
  \end{cases}
\end{equation}

Представим математическую модель в терминах вектора состояния:
\begin{equation} \label{eq:7}
  \begin{cases}
      \dot x = \dot x \\
      \ddot x = \frac{1}{M + m\sin(\varphi)^2}( -ml\sin(\varphi)\dot{\varphi}^2 + mg\cos(\varphi)\sin(\varphi) + \frac{\cos(\varphi)}{l}f + u ) \\
      \dot \varphi = \dot \varphi \\
      \ddot \varphi= \frac{1}{M + m\sin(\varphi)^2}( -m\sin(\varphi)\cos(\varphi)\dot{\varphi}^2 + \frac{(M+m)g}{l}\sin(\varphi) + \frac{M+m}{ml^2}f + \frac{\cos(\varphi)}{l}u ) \\
      y_1 = x \\
      y_2 = \varphi
  \end{cases}
\end{equation}

\FloatBarrier
\subsection{Точки равновесия}
В точках равновесия все компоненты производной вектора \(s\) по времени равны $0$ при отсутсвии внешних возмущений и управления.
\[
  \begin{cases}
    f = 0 \\
    u = 0 \\
    \dot x = 0\\
    \frac{1}{M + m\sin(\varphi)^2}( -ml\sin(\varphi)\dot{\varphi}^2 + mg\cos(\varphi)\sin(\varphi) + \frac{\cos(\varphi)}{l}f + u ) = 0\\
    \dot \varphi = 0\\
    \frac{1}{M + m\sin(\varphi)^2}( -m\sin(\varphi)\cos(\varphi)\dot{\varphi}^2 + \frac{(M+m)g}{l}\sin(\varphi) + \frac{M+m}{ml^2}f + \frac{\cos(\varphi)}{l}u ) = 0\\
  \end{cases}
\]
Упростим выражения, подставив известное
\[
  \begin{cases}
    f = 0 \\
    u = 0 \\
    \dot x = 0\\
    \dot \varphi = 0\\
    \frac{1}{M + m\sin(\varphi)^2}(mg\cos(\varphi)\sin(\varphi)) = 0\\
    \frac{1}{M + m\sin(\varphi)^2}(\frac{(M+m)g}{l}\sin(\varphi)) = 0\\
  \end{cases} \rightarrow
  \begin{cases}
    x \in \mathds{R} \\
    \dot x = 0\\
    \varphi = \pi n, n \in \mathds{Z}\\
    \dot \varphi = 0\\
  \end{cases}
\]
Соответственно, система стабильна, когда маятник замер в верхнем или нижнем положении, а тележка стоит на месте в любой точке.

\subsection{Линеаризация}
Рассматривается тележка с маятником сверху (\(\varphi \approx 0; \dot \varphi \approx 0\)), тогда можно считать: \(\sin(\varphi) = \varphi; \cos(\varphi) = 1; \varphi ^ 2 \ll \varphi; \dot \varphi ^ 2 \ll \dot \varphi\).
\begin{equation} \label{eq:8}
  \begin{cases}
      \dot x = \dot x \\
      \ddot x = \frac{1}{M}( mg\varphi + \frac{1}{l}f + u ) \\
      \dot \varphi = \dot \varphi \\
      \ddot \varphi= \frac{1}{M}(\frac{(M+m)g}{l}\varphi + \frac{M+m}{ml^2}f + \frac{1}{l}u ) \\
      y_1 = x \\
      y_2 = \varphi
  \end{cases}
\end{equation}
Можем представить линеаризованную систему в матричном виде:
\begin{equation} \label{eq:9}
    \begin{cases}
        \dot s = As + Bu + Df \\
        y = Cs
    \end{cases}
\end{equation}
\begin{equation*}
    A = \begin{bmatrix}
        0 & 1 & 0 & 0 \\
        0 & 0 & \frac{mg}{M} & 0 \\
        0 & 0 & 0 & 1 \\
        0 & 0 & \frac{(M+m)g}{Ml} & 0
    \end{bmatrix},
    B = \begin{bmatrix}
        0 \\ \frac{1}{M} \\ 0 \\ \frac{1}{Ml}
    \end{bmatrix},
    D = \begin{bmatrix}
        0 \\ \frac{1}{Ml} \\ 0 \\ \frac{M+m}{Mml^2}
    \end{bmatrix},
    C = \begin{bmatrix}
        1 & 0 & 0 & 0 \\
        0 & 0 & 1 & 0
    \end{bmatrix}
\end{equation*}

\FloatBarrier
\subsection{Параметры системы}
Далее принимается, что:
\begin{equation*}
  M = 1, m = 0.1, l = 1, g = 10
\end{equation*}
\[A = \begin{bmatrix}
  0  &  1  &  0  &  0 \\
  0  &  0  &  1  &  0 \\
  0  &  0  &  0  &  1 \\
  0  &  0  &  11  &  0 
\end{bmatrix}; B = \begin{bmatrix}
  0 \\
  1 \\
  0 \\
  1 
\end{bmatrix}; D = \begin{bmatrix}
  0 \\
  1 \\
  0 \\
  11 
\end{bmatrix}; C = \begin{bmatrix}
  1  &  0  &  0  &  0 \\
  0  &  0  &  1  &  0 
\end{bmatrix}\]
