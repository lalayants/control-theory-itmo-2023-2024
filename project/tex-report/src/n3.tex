\section{ГЛАВА 3. СТАБИЛИЗАЦИЯ МАЯТНИКА: МОДАЛЬНОЕ УПРАВЛЕНИЕ}

\subsection{Синтез регулятора по состоянию}
В этом задании выводится модальный регулятор для системы:
\[
        \begin{cases}
                \text{Объект управления: }\dot{x} = A x + Bu \\
                \text{Регулятор: }u = K x \\
        \end{cases} \rightarrow
        \dot{x} = A x + BKx = (A+BK)x
\]
Для этого подбирается матрица \(\Gamma \in \mathds{R}^{n \times n}\) с желаемыми собственными числами и матрица \(\mathds{Y} \in \mathds{R}^{m \times n}\), такая что пара \((\mathds{Y}, \Gamma)\) наблюдаема. После чего по подобию:
\[A+BK = P \Gamma P^{-1} \rightarrow
        \begin{cases}
                AP - P\Gamma = BY \\
                K = -YP^{-1} \\
        \end{cases}
\]

% \begin{figure}[ht!]
%   \centering
%   \includegraphics[width=\textwidth]{src/figs/task2_4_0_0_0_0.01.jpg}
%   \caption{Динамика системы.}
%   \label{fig:task2_4_0_0_0_0.01}
% \end{figure}
\FloatBarrier