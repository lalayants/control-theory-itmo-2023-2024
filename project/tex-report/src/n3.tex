\section{ГЛАВА 3. СТАБИЛИЗАЦИЯ МАЯТНИКА: МОДАЛЬНОЕ УПРАВЛЕНИЕ}

\subsection{Синтез регулятора по состоянию}
В этом задании выводится модальный регулятор для системы:
\[
        \begin{cases}
                \text{Объект управления: }\dot{x} = A x + Bu \\
                \text{Регулятор: }u = K x \\
        \end{cases} \rightarrow
        \dot{x} = A x + BKx = (A+BK)x
\]
Для этого подбирается матрица \(\Gamma \in \mathds{R}^{n \times n}\) с желаемыми собственными числами и матрица \(\mathds{Y} \in \mathds{R}^{m \times n}\), такая что пара \((\mathds{Y}, \Gamma)\) наблюдаема. После чего по подобию:
\[A+BK = P \Gamma P^{-1} \rightarrow
        \begin{cases}
                AP - P\Gamma = BY \\
                K = -YP^{-1} \\
        \end{cases}
\]

Получен регулятор:
\[K = \begin{bmatrix}
        2.40 &  5.00 & -48.40 & -15.00
      \end{bmatrix}\]
\[spec(A + B K) = \begin{bmatrix}
-4.00 & -3.00 & -2.00 & -1.00
\end{bmatrix}\]

Он подходит для нелинейной системой, если \(\varphi\) близко к 0. Чем дальше от 0 -- тем хуже справляется. Угловое ускорение такое сильное влияние не оказывает.
\begin{figure}[ht!]
  \centering
  \includegraphics[width=\textwidth]{src/figs/task3_1_0_0_0_1.jpg}
  \caption{Динамика системы.}
  \label{fig:task3_1_0_0_0_1.jpg}
\end{figure}
\begin{figure}[ht!]
        \centering
        \includegraphics[width=\textwidth]{src/figs/task3_1_0_0_0_2.jpg}
        \caption{Динамика системы.}
        \label{fig:task3_1_0_0_0_2.jpg}
\end{figure}
\begin{figure}[ht!]
        \centering
        \includegraphics[width=\textwidth]{src/figs/task3_1_0_0_1_0.jpg}
        \caption{Динамика системы.}
        \label{fig:task3_1_0_0_1_0.jpg}
\end{figure}
\begin{figure}[ht!]
        \centering
        \includegraphics[width=\textwidth]{src/figs/task3_1_0_0_2_0.jpg}
        \caption{Динамика системы.}
        \label{fig:task3_1_0_0_2_0.jpg}
\end{figure}
\FloatBarrier

\subsection{Исследование регулятора по состоянию}
\begin{center}
        \begin{tabular}{ c | c c c }
    $\sigma G$ & $\max x$ & $\max \varphi$ & $\max u$ \\
            $\begin{bmatrix}
     -1 & -2 & -3 & -4
    \end{bmatrix}$ & 5.1 & 4.0 & 48.4 \\
            $\begin{bmatrix}
     -0.10 & -0.20 & -0.30 & -0.40
    \end{bmatrix}$ & 110.4 & 13.6 & 76.7 \\
            $\begin{bmatrix}
     -1 + 1j & -1 + -1j & -2 + 2j & -2 + -2j
    \end{bmatrix}$ & 31482.2 & 194.8 & 256515.5 \\
        \end{tabular}
    \end{center}
Как видно, система довольно требовательна к спекту, набор из затихающих гармоник приводит к неустойчивому поведению.
\FloatBarrier

\subsection{Синтез наблюдателя}
В этом задании выводится наблюдатель состояния для системы:
\[
        \begin{cases}
                \dot{x} = A x \\
                y = C x \\
                \dot{\hat{x}} = A \hat{x} + L(\hat{y} - y) \\
                \hat{y} = C \hat{x}
        \end{cases} \rightarrow
        \begin{cases}
            \dot{\hat{x}} = A \hat{x} + L(C \hat{x} - y) = (A + LC )\hat{x} - Ly \\
            \dot{e} = (A + LC)e
            
    \end{cases}
\]

Для синтеза наблюдателя подбирается матрица \(\Gamma \in \mathds{R}^{n \times n}\) с желаемыми собственными числами и матрица \(\mathds{Y} \in \mathds{R}^{n \times k}\), такая что пара \((\Gamma, \mathds{Y})\) управляема. После чего:
\[
    \begin{cases}
        \Gamma Q - QA = YC \\
        L = Q^{-1}Y \\
    \end{cases}
\]
\[L = \begin{bmatrix}
        2.40 &  5.00 & -48.40 & -15.00
      \end{bmatrix}\]
      \[\Gamma = \begin{bmatrix}
       -1.00 &  0.00 &  0.00 &  0.00\\
        0.00 & -2.00 &  0.00 &  0.00\\
        0.00 &  0.00 & -3.00 &  0.00\\
        0.00 &  0.00 &  0.00 & -4.00
      \end{bmatrix}\]
      \[Y = \begin{bmatrix}
        1.00 &  1.00\\
        1.00 &  1.00\\
        1.00 &  1.00\\
        1.00 &  1.00
      \end{bmatrix}\]
На графиках видно, что не при любых начальных условиях наблюдатель сходится.

\begin{figure}[ht!]
        \centering
        \includegraphics[width=\textwidth]{src/figs/task3_3_0.1_0.1.jpg}
        \caption{Динамика ошибки.}
        \label{fig:task3_3_0.1_0.1.jpg}
\end{figure}
\begin{figure}[ht!]
        \centering
        \includegraphics[width=\textwidth]{src/figs/task3_3_0_1.jpg}
        \caption{Динамика ошибки.}
        \label{fig:task3_3_0_1.jpg}
\end{figure}

\begin{figure}[ht!]
        \centering
        \includegraphics[width=\textwidth]{src/figs/task3_3_1_0.jpg}
        \caption{Динамика ошибки.}
        \label{fig:task3_3_1_0.jpg}
\end{figure}
\FloatBarrier

\subsection{Исследование наблюдателя}
Не при любом устойчивом спектре наблюдатель устойчив. 
\[x(0) = \begin{bmatrix}
        0.10 &  0.00 &  0.10 &  0.00
      \end{bmatrix}\]

\[L = \begin{bmatrix}
        2.40 &  5.00 & -48.40 & -15.00
      \end{bmatrix}\]
      \[\Gamma = \begin{bmatrix}
       -1.00 &  0.00 &  0.00 &  0.00\\
        0.00 & -2.00 &  0.00 &  0.00\\
        0.00 &  0.00 & -3.00 &  0.00\\
        0.00 &  0.00 &  0.00 & -4.00
      \end{bmatrix}\]
\begin{figure}[ht!]
        \centering
        \includegraphics[width=\textwidth]{src/figs/task3_4_-1.0_-2.0_-3.0_-4.0.jpg}
        \caption{Динамика ошибки.}
        \label{fig:task3_4_1.jpg}
\end{figure}
\FloatBarrier
      \[L = \begin{bmatrix}
        2.40 &  5.00 & -48.40 & -15.00
      \end{bmatrix}\]
      \[\Gamma = \begin{bmatrix}
       -0.10 &  0.00 &  0.00 &  0.00\\
        0.00 & -0.20 &  0.00 &  0.00\\
        0.00 &  0.00 & -0.30 &  0.00\\
        0.00 &  0.00 &  0.00 & -0.40
      \end{bmatrix}\]
\begin{figure}[ht!]
        \centering
        \includegraphics[width=\textwidth]{src/figs/task3_4_-0.1_-0.2_-0.3_-0.4.jpg}
        \caption{Динамика ошибки.}
        \label{fig:task3_4_2.jpg}
\end{figure}
\FloatBarrier
      \[L = \begin{bmatrix}
        2.40 &  5.00 & -48.40 & -15.00
      \end{bmatrix}\]
      \[\Gamma = \begin{bmatrix}
       -1.00 & -1.00 &  0.00 &  0.00\\
        1.00 & -1.00 &  0.00 &  0.00\\
        0.00 &  0.00 & -2.00 & -2.00\\
        0.00 &  0.00 &  2.00 & -2.00
      \end{bmatrix}\]
\begin{figure}[ht!]
        \centering
        \includegraphics[width=\textwidth]{src/figs/task3_4_(-1+1j)_(-1-1j)_(-2+2.0000000000000004j)_(-2-2.0000000000000004j).jpg}
        \caption{Динамика ошибки.}
        \label{fig:task3_4_3.jpg}
\end{figure}
\FloatBarrier

\subsection{Синтез регулятора по выходу}
На основе двух предыдущих пунктов получен регулятор по выходу.
\[L = \begin{bmatrix}
        3.33 &  3.33\\
       -1.83 & -1.83\\
       -13.33 & -13.33\\
       -44.17 & -44.17
      \end{bmatrix}\]
      \[K = \begin{bmatrix}
        2.40 &  5.00 & -48.40 & -15.00
      \end{bmatrix}\]
      \[\Gamma = \begin{bmatrix}
       -1.00 & -1.00 &  0.00 &  0.00\\
        1.00 & -1.00 &  0.00 &  0.00\\
        0.00 &  0.00 & -2.00 & -2.00\\
        0.00 &  0.00 &  2.00 & -2.00
      \end{bmatrix}\]
      \[L = \begin{bmatrix}
        1.50 &  1.50\\
       -1.08 & -1.08\\
       -7.50 & -7.50\\
       -27.92 & -27.92
      \end{bmatrix}\]
      \[K = \begin{bmatrix}
        1.60 &  2.40 & -30.60 & -8.40
      \end{bmatrix}\]
      \[\Gamma = \begin{bmatrix}
       -1.00 & -1.00 &  0.00 &  0.00\\
        1.00 & -1.00 &  0.00 &  0.00\\
        0.00 &  0.00 & -2.00 & -2.00\\
        0.00 &  0.00 &  2.00 & -2.00
      \end{bmatrix}\]
      
      \begin{figure}[ht!]
        \centering
        \includegraphics[width=\textwidth]{src/figs/task3_5_new.jpg}
        \caption{Динамика компонент системы.}
        \label{fig:task3_5.jpg}
\end{figure}

\FloatBarrier