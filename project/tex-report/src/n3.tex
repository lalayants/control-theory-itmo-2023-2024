\section{ГЛАВА 3. СТАБИЛИЗАЦИЯ МАЯТНИКА: МОДАЛЬНОЕ УПРАВЛЕНИЕ}

\subsection{Синтез регулятора по состоянию}
В этом задании выводится модальный регулятор для системы:
\[
        \begin{cases}
                \text{Объект управления: }\dot{x} = A x + Bu \\
                \text{Регулятор: }u = K x \\
        \end{cases} \rightarrow
        \dot{x} = A x + BKx = (A+BK)x
\]
Для этого подбирается матрица \(\Gamma \in \mathds{R}^{n \times n}\) с желаемыми собственными числами и матрица \(\mathds{Y} \in \mathds{R}^{m \times n}\), такая что пара \((\mathds{Y}, \Gamma)\) наблюдаема. После чего по подобию:
\[A+BK = P \Gamma P^{-1} \rightarrow
        \begin{cases}
                AP - P\Gamma = BY \\
                K = -YP^{-1} \\
        \end{cases}
\]

Получен регулятор:
\[K = \begin{bmatrix}
        2.40 &  5.00 & -48.40 & -15.00
      \end{bmatrix}\]
\[spec(A + B K) = \begin{bmatrix}
-4.00 & -3.00 & -2.00 & -1.00
\end{bmatrix}\]

Он подходит для нелинейной системой, если \(\varphi\) близко к 0. Чем дальше от 0 -- тем хуже справляется. Угловое ускорение такое сильное влияние не оказывает.
\begin{figure}[ht!]
  \centering
  \includegraphics[width=\textwidth]{src/figs/task3_1_0_0_0_1.jpg}
  \caption{Динамика системы.}
  \label{fig:task3_1_0_0_0_1.jpg}
\end{figure}
\begin{figure}[ht!]
        \centering
        \includegraphics[width=\textwidth]{src/figs/task3_1_0_0_0_2.jpg}
        \caption{Динамика системы.}
        \label{fig:task3_1_0_0_0_2.jpg}
\end{figure}
\begin{figure}[ht!]
        \centering
        \includegraphics[width=\textwidth]{src/figs/task3_1_0_0_1_0.jpg}
        \caption{Динамика системы.}
        \label{fig:task3_1_0_0_1_0.jpg}
\end{figure}
\begin{figure}[ht!]
        \centering
        \includegraphics[width=\textwidth]{src/figs/task3_1_0_0_2_0.jpg}
        \caption{Динамика системы.}
        \label{fig:task3_1_0_0_2_0.jpg}
\end{figure}
\FloatBarrier