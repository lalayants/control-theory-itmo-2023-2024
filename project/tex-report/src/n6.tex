\section{ГЛАВА 6. СЛЕЖЕНИЕ И КОМПЕНСАЦИЯ}

\subsection{Решение задачи компенсации}
Рассмотрим систему вида:
\[
    \begin{cases}
        \dot{x} = A_1x + B_1u + B_2w \\
        z = C_2x + D_2 w
    \end{cases},
\]
где $w$:
\[
    \dot{w} = A_2w
\]

Для данной системы можем синтезировать регулятор вида $u = K_1x + K_2w$, гарантирующий:
\begin{equation*}
    \lim_{t\to\infty} z(t) = 0
\end{equation*}

$K_1$ можем выбрать как матрицу регулятора, синтезированного любым способом. Матрицу $K_2$ найдем следующим образом:
\[
    \begin{cases}
        PA_2 - A_1P = B_1Y + B_2\\
        C_2P + D_2 = 0 \\
        K_2 = Y - K_1P
    \end{cases}
\]

\[
    \begin{cases}
        \dot{x} = A_1x + B_1 (K_1x + K_2w) + B_2w = (A_1 + B_1 K_1) x + (B_2 + B_1 K_2)w \\
        z = C_2x + D_2 w
    \end{cases},
\]

\[A_1 = \begin{bmatrix}
        0.00 &  1.00 &  0.00 &  0.00\\
        0.00 &  0.00 &  1.00 &  0.00\\
        0.00 &  0.00 &  0.00 &  1.00\\
        0.00 &  0.00 &  11.00 &  0.00
      \end{bmatrix}\]
      \[A_2 = \begin{bmatrix}
        0.00 &  0.10 &  0.00 &  0.00\\
       -0.10 &  0.00 &  0.00 &  0.00\\
        0.00 &  0.00 &  0.00 &  1.00\\
        0.00 &  0.00 & -1.00 &  0.00
      \end{bmatrix}\]
      \[B_1 = \begin{bmatrix}
        0.00\\
        1.00\\
        0.00\\
        1.00
      \end{bmatrix}\]
      \[B_2 = \begin{bmatrix}
        0.00 &  0.00 &  0.00 &  0.00\\
        0.33 &  0.67 &  1.00 &  1.33\\
        0.00 &  0.00 &  0.00 &  0.00\\
        3.67 &  7.33 &  11.00 &  14.67
      \end{bmatrix}\]
      \[C_2 = \begin{bmatrix}
        0.00 &  0.00 &  1.00 &  0.00
      \end{bmatrix}\]
      \[D_2 = \begin{bmatrix}
        0.00 &  0.00 &  0.00 &  0.00
      \end{bmatrix}\]
      \[C_1 = \begin{bmatrix}
        1.00 &  0.00 &  0.00 &  0.00\\
        0.00 &  0.00 &  1.00 &  0.00
      \end{bmatrix}\]
      \[D_1 = \begin{bmatrix}
        0.00 &  0.00 &  0.00 &  0.00\\
        0.00 &  0.00 &  0.00 &  0.00
      \end{bmatrix}\]
      \[K_1 = \begin{bmatrix}
        0.32 &  1.08 & -28.01 & -8.54
      \end{bmatrix}\]
      \[spec(A + B_1 K_1) = \begin{bmatrix}
       -3.47 & -3.17 & -0.41 + 0.35j & -0.41 -0.35j
      \end{bmatrix}\]
      \[K_2 = \begin{bmatrix}
       -36.79 & -254.29 &  0.29 & -29.73
      \end{bmatrix}\]
\begin{figure}[ht!]
        \centering
        \includegraphics[width=\textwidth]{src/figs/task6_1_states.jpg}
        \caption{Динамика системы.}
        \label{fig:task6_1_states}
\end{figure}

\begin{figure}[ht!]
        \centering
        \includegraphics[width=\textwidth]{src/figs/task6_1_z.jpg}
        \caption{Регулируемый выход системы.}
        \label{fig:task6_1_z}
\end{figure}

Как видно, для нелинейной системы регулятор не справляется с задачей. Ошибка угла колеблется около 0. Тем не менее, полученный результат можно использовать для хоть какой-то компенсации, так как поведение такой системы почти соответсвует желаемому.
\FloatBarrier

\subsection{Решение задачи слежения}
Рассмотрим систему вида:
\[
    \begin{cases}
        \dot{x} = A_1x + B_1u + B_2w \\
        z = C_2x + D_2w
    \end{cases},
\]
где $w$:
\[
    \dot{w} = A_2w
\]

Для данной системы можем синтезировать регулятор вида $u = K_1x + K_2w$, гарантирующий:
\begin{equation*}
    \lim_{t\to\infty} z(t) = 0
\end{equation*}

$K_1$ можем выбрать как матрицу регулятора, синтезированного любым способом. Матрицу $K_2$ найдем следующим образом:
\[
    \begin{cases}
        PA_2 - A_1P = B_1Y + B_2\\
        C_2P + D_2 = 0 \\
        K_2 = Y - K_1P
    \end{cases}
\]

\[
    \begin{cases}
        \dot{x} = A_1x + B_1 (K_1x + K_2w) + B_2w = (A_1 + B_1 K_1) x + (B_2 + B_1 K_2)w \\
        z = C_2x + D_2 w
    \end{cases},
\]
\[A_1 = \begin{bmatrix}
        0.00 &  1.00 &  0.00 &  0.00\\
        0.00 &  0.00 &  1.00 &  0.00\\
        0.00 &  0.00 &  0.00 &  1.00\\
        0.00 &  0.00 &  11.00 &  0.00
      \end{bmatrix}\]
      \[A_2 = \begin{bmatrix}
        0.00 &  0.10 &  0.00 &  0.00\\
       -0.10 &  0.00 &  0.00 &  0.00\\
        0.00 &  0.00 &  0.00 &  1.00\\
        0.00 &  0.00 & -1.00 &  0.00
      \end{bmatrix}\]
      \[B_1 = \begin{bmatrix}
        0.00\\
        1.00\\
        0.00\\
        1.00
      \end{bmatrix}\]
      \[B_1 = \begin{bmatrix}
        0.00 &  0.00 &  0.00 &  0.00\\
        0.00 &  0.00 &  0.00 &  0.00\\
        0.00 &  0.00 &  0.00 &  0.00\\
        0.00 &  0.00 &  0.00 &  0.00
      \end{bmatrix}\]
      \[C_2 = \begin{bmatrix}
        0.00 &  0.00 &  1.00 &  0.00
      \end{bmatrix}\]
      \[D_2 = \begin{bmatrix}
        0.10 &  0.10 &  0.10 &  0.10
      \end{bmatrix}\]
\begin{figure}[ht!]
        \centering
        \includegraphics[width=\textwidth]{src/figs/task6_2_states.jpg}
        \caption{Динамика системы.}
        \label{fig:task6_2_states}
\end{figure}

\begin{figure}[ht!]
        \centering
        \includegraphics[width=\textwidth]{src/figs/task6_2_target.jpg}
        \caption{Регулируемый выход системы.}
        \label{fig:task6_2_z}
\end{figure}
Как видно, задача слежения выполнена. Нелинейная система имела чуть большую ошибку вначале, но в итоге тоже свела ее к 0.
\FloatBarrier