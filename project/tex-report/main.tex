\documentclass[16pt]{article}

\usepackage{report}

\usepackage[utf8]{inputenc} % allow utf-8 input
\usepackage[english, russian]{babel}
\usepackage[T1]{fontenc}    % use 8-bit T1 fonts
\usepackage[colorlinks=true, linkcolor=black, citecolor=blue, urlcolor=blue]{hyperref}       % hyperlinks
\usepackage{url}            % simple URL typesetting
\usepackage{booktabs}       % professional-quality tables
\usepackage{amsfonts}       % blackboard math symbols
\usepackage{nicefrac}       % compact symbols for 1/2, etc.
\usepackage{microtype}      % microtypography
\usepackage{graphicx}
\usepackage{natbib}
\usepackage{doi}
\usepackage{mathtools}
\usepackage{minted}
\usepackage{ dsfont }
\usepackage{graphicx}
\usepackage{placeins}
\usepackage{ amssymb }
\graphicspath{ {./figs/} }

\setcitestyle{aysep={,}}



\title{Курсовой проект}

\author{
Студент \\
Кирилл Лалаянц\\
R33352\\
336700\\
Вариант - 11\\
\\
Преподаватель\\
Пашенко А.В. \\
\\
\\
\\
Факультет Систем Управления и Робототехники\\
\\
ИТМО\\
}

% Uncomment to remove the date
\date{14.05.2024}

% Uncomment to override  the `A preprint' in the header
\renewcommand{\headeright}{Курсовой проект}
\renewcommand{\undertitle}{Отчет}
\renewcommand{\shorttitle}{}


\begin{document}
\maketitle
\newpage
\tableofcontents
\thispagestyle{empty}

\newpage
\setcounter{page}{1}
% \section{Программная реализация}
C исходным кодом можно ознакомиться \href{https://github.com/lalayants/control-theory-itmo-2023-2024}{в репозитории на Github.}


\newpage
\subsection{Задание 1}
Имеем систему:
\[\dot{x} = A x + Bu \]
\[x(t_1) = \int_0^{t_1} B u(t) dt\]
\subsubsection{Управляемость через матрицу управляемости}
\(U = [B | AB | \hdots | A^{n-1}B]\) для \(A \in R^{n \times n}\) и \(B \in R^{n \times m}\) -- матрица управляемости системы. Если ее ранг равен \(n\) -- система управляема.

\subsubsection{Управляемость и Грамиан управляемости}
\[P(t_1) = \int_0^{t_1} e^{At} B B^T e^{A^T t} dt\]
У управляемой системы Грамиан управляемости положительно определен в любой момент t.

\subsubsection{Управляемость через собственные числа матрицы системы}
\[\forall \lambda \in spec(A): rank(A - \lambda I | B) = n \Longleftrightarrow \text{Матрица управляема} \]

\subsubsection{Управляемость через Жорданову форму}
Жорданова форма матрицы \(A = P J P^{-1}\):
\[
        \dot{x} = P J P^{-1} x + Bu 
\]

Пусть \(\hat{x} = P^{-1}x\), тогда получим Жорданову форму системы:
\[
        \dot{\hat{x}} = J\hat{x} + P^{-1}Bu = J\hat{x} + \hat{B}u  \\
\]

Система в Жордановой форме полностью управляема, если:
\begin{itemize}
    \item каждому собственному числу соответсвует только одна Жорданова клетка. 
    \item элементы матрицы входного воздействия, соответсвующие последним строкам клеток -- не нулевые.
\end{itemize}

\subsubsection{Програмное управление системой}
Для вычисления управления, необходимого для достижения состояния \(x_1\) к моменту времени \(t_1\) достаточно рассчитать:
\[u(t) = B^T e^{A^T (t_1 - t)} (P(t_1))^{-1} x_1\]


\subsubsection{Результаты}
Вариант задания:
\[ A = \begin{bmatrix}
        3 & 4 & -1 \\
        -10 & -11 & -4 \\
        10 & 10 & 3 \\
        \end{bmatrix}; 
        B = \begin{bmatrix}
                -2 \\
                5 \\
                -3 \\
                \end{bmatrix}; 
        x_1 = \begin{bmatrix}
                -2 \\
                1 \\
                -1 \\
                \end{bmatrix}; 
        t_1 = 3
\]

Матрица управляемости U:
\[ U = \begin{bmatrix}
        -2 & 17 & -62 \\
        5 & -23 & -1 \\
        -3 & 21 & 3 \\
        \end{bmatrix};
\]
\[rankU = 3 = n \rightarrow \text{система управляема.}\]

Получены собственные числа \(spec(A) = [-2+5j, -2 -5j, -1+0j]\). Каждое из них удовлетворяет выражению\(rank(A - \lambda I | B) = n \) -- то есть управлямо.

Так же рассмотрим управляемость собственных чисел через форму Жордана:
\[
        J = \begin{bmatrix}
                -1 & 0 & 0 \\
                0 & -2 - 5j & 0 \\
                0 & 0 & -2 + 5j \\
                \end{bmatrix};
        \hat{B} = \begin{bmatrix}
                2 \\
                -1.5 + 1.5j \\
                -1.5 - 1.5j \\
                \end{bmatrix};
\]
Как видно, условия выполнены -- еще одно подтверждение, что пара (А, B) управляема.

Так как матрица полностью управляема, любая точка принадлежит управвляемому подпространству системы, в том числе и \(x_1\).

Получен Грамиан управляемости:
\[
        P(t_1) = \begin{bmatrix}
        1.20 & -1.34 & 0.23 \\
        -1.34 & 2.76 & -1.12 \\
        0.23 & -1.12 & 1.47 \\
        \end{bmatrix};
\]
\[
        spec(P(t_1)) = [4, 0.27, 1.15]
\]Грамиан положительно определен, система полностью управляема.

На рисунке \ref{fig:task1} приведены результаты моделирования системы. Как видно, она приняла желаемо состояние за нужное время.
\begin{figure}[ht!]
        \centering
        \includegraphics[width=\textwidth]{src/figs/task1.jpg}
        \caption{Результаты моделирования задания 1.}
        \label{fig:task1}
\end{figure}


% \begin{cases}
%     \dot{x} = P J P^{-1} x + Bu \\
%     y = Cx + Du \\
% \end{cases},
\FloatBarrier

\newpage
\subsection{Задание 2}
\subsubsection{Теория}
В этом задании выводится ограничение на управление \(||u(t)|| \leq \mu\). Тогда система уравнений принимает вид: 
\[
        \begin{cases}
                \begin{bmatrix}
                    P &  x_0\\
                    x_0^T &  1 \\
                \end{bmatrix} \succ 0 \\
                \\
                \begin{bmatrix}
                    P &  Y^T\\
                    Y &  \mu^2I \\
                \end{bmatrix} \succ 0 \\
                P \succ 0 \\
                PA^T + AP + 2 \alpha P + Y^T B^T + BY \preccurlyeq 0  \\
                K = YP^{-1}\\
        \end{cases} 
\]




\subsubsection{Результаты}
Сначала проведено исследование влияния ограничения для \(alpha = 1\).  На рисунках \ref{fig:task2_1_states} -- \ref{fig:task2_1_us}.

При минимальном \(\mu\) система имеет собственные числа \( \begin{bmatrix}
    -1.00 + 5.50j & -1.00 + -5.50j & -1.00 + 0.00j & -4.00 + 0.00j
   \end{bmatrix} \), максимально прижавшись к необходимой степени устойчивости всеми управляемыми числами. По мере ослабления ограничения, числа начинают отдаляться от границы.
\begin{figure}[ht!]
    \centering
    \includegraphics[width=\textwidth]{src/figs/task2_1_states.jpg}
    \caption{Результаты моделирования состояний системы для \(\alpha = 1\) при различных ограничениях.}
    \label{fig:task2_1_states}
\end{figure}
\begin{figure}[ht!]
    \centering
    \includegraphics[width=\textwidth]{src/figs/task2_1_us.jpg}
    \caption{Результаты моделирования управления системой для \(\alpha = 1\) при различных ограничениях.}
    \label{fig:task2_1_us}
\end{figure}

Для всех остальных степеней устойчивости результат аналогичен. Все управляемы собственные числа принимают значение степени устойчивости.
\begin{figure}[ht!]
    \centering
    \includegraphics[width=\textwidth]{src/figs/task2_2_states.jpg}
    \caption{Результаты моделирования состояний системы для различных \(\alpha\) с минимизированным управлением.}
    \label{fig:task2_2_states}
\end{figure}
% \begin{cases}
%     \dot{x} = P J P^{-1} x + Bu \\
%     y = Cx + Du \\
% \end{cases},

\[
\alpha = 0.1; 
K = \begin{bmatrix}
  0.00\\
 -0.61\\
 -0.11\\
 -0.23
\end{bmatrix};
\sigma(A+BK) = \begin{bmatrix}
 -0.10 + 5.15j\\
 -0.10 + -5.15j\\
 -0.10 + 0.00j\\
 -4.00 + 0.00j
\end{bmatrix};
\]
\[
\alpha = 1.0; 
K = \begin{bmatrix}
  0.00\\
 -1.37\\
 -0.38\\
 -0.36
\end{bmatrix};
\sigma(A+BK) = \begin{bmatrix}
 -1.00 + 5.50j\\
 -1.00 + -5.50j\\
 -1.00 + 0.00j\\
 -4.00 + 0.00j
\end{bmatrix};
\]
\[
\alpha = 2.0; 
K = \begin{bmatrix}
  0.00\\
 -2.80\\
 -0.88\\
 -0.38
\end{bmatrix};
\sigma(A+BK) = \begin{bmatrix}
 -2.00 + 6.13j\\
 -2.00 + -6.13j\\
 -2.00 + 0.00j\\
 -4.00 + 0.00j
\end{bmatrix};
\]
\[
\alpha = 3.0; 
K = \begin{bmatrix}
  0.00\\
 -5.13\\
 -1.58\\
 -0.19
\end{bmatrix};
\sigma(A+BK) = \begin{bmatrix}
 -3.00 + 6.94j\\
 -3.00 + -6.94j\\
 -3.00 + 0.00j\\
 -4.00 + 0.00j
\end{bmatrix};
\]



\FloatBarrier
\FloatBarrier

\newpage
\subsection{Задание 3}
Имеем систему:
\[
\begin{cases}
    \dot{x} = A x \\
    y = Cx  \\
\end{cases}
\]
\[
        y(t) = C e^{At} x(0)
\]
\subsubsection{Наблюдаемость через матрицу наблюдаемости}
\(V = [C | CA | \hdots | CA^{n-1}]^T\) для \(A \in R^{n \times n}\) и \(C \in R^{k \times n}\) -- матрица наблюдаемости системы. Если ее ранг равен \(n\) -- система наблюдаема.

\subsubsection{Наблюдаемость и Грамиан наблюдаемости}
\[Q(t_1) = \int_0^{t_1} e^{A^T t} C^T C e^{A t} dt\]
У наблюдаемой системы Грамиан наблюдаемости положительно определен в любой момент t.

\subsubsection{Управляемость через собственные числа матрицы системы}
\[\forall \lambda \in spec(A): rank(\begin{bmatrix}
        A - \lambda I  \\
        C \\
        \end{bmatrix}) = n \Longleftrightarrow \text{Матрица наблюдаема} \]

\subsubsection{Управляемость через Жорданову форму}
Жорданова форма матрицы \(A = P J P^{-1}\):
\[
        \begin{cases}
                \dot{x} = P J P^{-1} x \\
                y = Cx  \\
        \end{cases}
\]

Пусть \(\hat{x} = P^{-1}x\), тогда получим Жорданову форму системы:
\[
        \begin{cases}
                \dot{\hat{x}} = J\hat{x} \\
                y = CP\hat{x} = \hat{C}\hat{x} \\
        \end{cases}
\]

Система в Жордановой форме полностью наблюдаема, если:
\begin{itemize}
    \item каждому собственному числу соответсвует только одна Жорданова клетка. 
    \item элементы матрицы выходов, соответсвующие первым столбцам клеток -- не нулевые.
\end{itemize}

\subsubsection{Начальные условия системы}
Для вычисления начальных условий системы, достаточно рассчитать:
\[x(0) = (P(t_1))^{-1} \int_{0}^{t_1} e^{A^T t} C^T y(t) dt\]


\subsubsection{Результаты}
Вариант задания:
\[ A = \begin{bmatrix}
        -21 & -38 & 6 \\
        8 & 13 & -4 \\
        -6 & -14 & -1 \\
        \end{bmatrix}; 
        C = \begin{bmatrix}
                9  & 8 & -2
                \end{bmatrix}; 
        y = 3 e^{-5x}\cos{2x} - e^{-5x}\sin{2x}; 
        t_1 = 3
\]

Матрица наблюдаемости V:
\[ V = \begin{bmatrix}
        9 & 8 & -2 \\
        -113 & -210 & 24 \\
        549 & 1228 & 138 \\
        \end{bmatrix};
\]
\[rankU = V = n \rightarrow \text{система наблюдаема.}\]

Получены собственные числа \(spec(A) = [-5 + 2j, -5 - 2j, 1+0j]\). Каждое из них наблюдаемо, что еще раз показывает наблюдаемость системы.

Так же рассмотрим наблюдаемость собственных чисел через форму Жордана:
\[
        J = \begin{bmatrix}
                1 & 0 & 0 \\
                0 & -5 - 2j & 0 \\
                0 & 0 & -5 + 2j \\
                \end{bmatrix};
        \hat{C} = \begin{bmatrix}
                8 &
                17 + 10j &
                17 - 10j \\
                \end{bmatrix};
\]
Как видно, условия выполнены -- еще одно подтверждение, что пара (A, C) наблюдаема.

Получен Грамиан наблюдаемости:
\[
        Q(t_1) = \begin{bmatrix}
        12861 & 25733 & -12846\\
        25733 & 51508 & -25706 \\
        -12846 & -25706 & 12835 \\
        \end{bmatrix};
\]
\[
        spec(P(t_1)) = [77197, 1.34, 6.40]
\]Грамиан положительно определен, система полностью наблюдаема.

На рисунке \ref{fig:task3} приведены результаты моделирования системы. Как видно, при моделировании она действительно пришла в нужный вектор наблюдения за отведенное время.
\begin{figure}[ht!]
        \centering
        \includegraphics[width=\textwidth]{src/figs/task3.jpg}
        \caption{Результаты моделирования задания 3.}
        \label{fig:task3}
\end{figure}
Так как система полностью наблюдаема, иных начальных состояние быть не может. 

\FloatBarrier

\newpage
\subsection{Синтез \(H_\infty\)-регулятора по выходу.}
\begin{equation}
    \begin{cases}
        AP + PA^T + B_1B_1^T - PC_1^T(D_1D_1^T)^{-1}C_1P = 0 \\
        L = - PC_1^T(D_1^TD_1)^{-1} \\
        A^TQ + QA + C_2^TC_2 - QB_2(D_2^TD_2)^{-1}B_2^TQ = 0 \\
        K = -(D_2^TD_2)^{-1}B_2^TQ
    \end{cases}
\end{equation}

Представим систему в виде:
\begin{equation}
    \begin{cases}
        \begin{bmatrix}
            \dot{x} \\ \dot{e}
        \end{bmatrix} =
        \begin{bmatrix}
            A + B_2K & -B_2K \\
            -(LD_1B_1)\gamma^{-2}B_1^TQ & A + LC_1 + (LD_1B_1)\gamma^{-2}B_1^TQ
        \end{bmatrix} \begin{bmatrix}
            x \\ e
        \end{bmatrix} +
        \begin{bmatrix}
            B_1 \\
            LD_1 + B_1
        \end{bmatrix}w \\
        z = \begin{bmatrix}
            C_2 +D_2K & -D_2K
        \end{bmatrix}\begin{bmatrix}
            x \\ e
        \end{bmatrix}
    \end{cases}
\end{equation}


\subsubsubsection{gamma = 1.4}
\[spec(A-B_2 K) = [-0.42  2.43]\]
\[K = \begin{bmatrix}
 -1.01 & -2.02
\end{bmatrix}\]
\[Q = \begin{bmatrix}
  1.01 &  1.01\\
  1.01 &  2.02
\end{bmatrix}\]
\[L = \begin{bmatrix}
    -1.89\\
    -1.16
   \end{bmatrix}\]
\[ W = \left[\begin{matrix}\frac{- 1.0 i \omega^{3} - 3.8 \omega^{2} + 1.6 i \omega - 1.1}{1.0 \omega^{4} - 3.8 i \omega^{3} - 5.9 \omega^{2} + 4.2 i \omega + 1.1} & \frac{- 0.9 i \omega^{3} - 4.8 \omega^{2} + 9.81 i \omega + 5.9}{1.0 \omega^{4} - 3.8 i \omega^{3} - 5.9 \omega^{2} + 4.2 i \omega + 1.1} & \frac{4.2 \omega^{2} - 5.4 i \omega - 1.1}{1.0 \omega^{4} - 3.8 i \omega^{3} - 5.9 \omega^{2} + 4.2 i \omega + 1.1}\\\frac{- 4.2 i \omega - 1.1}{1.0 \omega^{4} - 3.8 i \omega^{3} - 5.9 \omega^{2} + 4.2 i \omega + 1.1} & \frac{- 1.0 i \omega^{3} - 3.8 \omega^{2} + 5.9 i \omega}{1.0 \omega^{4} - 3.8 i \omega^{3} - 5.9 \omega^{2} + 4.2 i \omega + 1.1} & \frac{4.2 \omega^{2} - 1.1 i \omega}{1.0 \omega^{4} - 3.8 i \omega^{3} - 5.9 \omega^{2} + 4.2 i \omega + 1.1}\\\frac{4.2 \omega^{2} - 1.1 i \omega}{1.0 \omega^{4} - 3.8 i \omega^{3} - 5.9 \omega^{2} + 4.2 i \omega + 1.1} & \frac{- 4.2 i \omega - 1.1}{1.0 \omega^{4} - 3.8 i \omega^{3} - 5.9 \omega^{2} + 4.2 i \omega + 1.1} & \frac{4.2 i \omega^{3} + 1.1 \omega^{2}}{1.0 \omega^{4} - 3.8 i \omega^{3} - 5.9 \omega^{2} + 4.2 i \omega + 1.1}\end{matrix}\right] \]
\[||W||_{H_2} = 3.9739213167564933\]
\[||W||_{H_\infty} = 5.4385479975362845 \]

\begin{figure}[ht!]
    \centering
    \includegraphics[width=\textwidth]{src/figs/task4_0_amps.jpg}
    \caption{АЧХ системы.}
    \label{fig:task4_0_amps}
  \end{figure}
  
  \begin{figure}[ht!]
    \centering
    \includegraphics[width=\textwidth]{src/figs/task4_0_zs.jpg}
    \caption{Регулируемый выход системы.}
    \label{fig:task4_0_zs}
  \end{figure}
  
  \begin{figure}[ht!]
    \centering
    \includegraphics[width=\textwidth]{src/figs/task4_0_sing.jpg}
    \caption{Сингулярные числы.}
    \label{fig:task4_0_sing}
  \end{figure}

  \FloatBarrier
  \subsubsubsection{gamma = 2}
\[spec(A-B_2 K) = [-0.42  2.43]\]
\[K = \begin{bmatrix}
 -1.01 & -2.02
\end{bmatrix}\]
\[Q = \begin{bmatrix}
  1.01 &  1.01\\
  1.01 &  2.02
\end{bmatrix}\]
\[L = \begin{bmatrix}
 -1.89\\
 -1.16
\end{bmatrix}\]
\[ W = \left[\begin{matrix}\frac{- 1.0 i \omega^{3} - 3.8 \omega^{2} + 1.6 i \omega - 1.1}{1.0 \omega^{4} - 3.8 i \omega^{3} - 5.9 \omega^{2} + 4.2 i \omega + 1.1} & \frac{- i \omega^{3} - 4.8 \omega^{2} + 9.8 i \omega + 5.9}{1.0 \omega^{4} - 3.8 i \omega^{3} - 5.9 \omega^{2} + 4.2 i \omega + 1.1} & \frac{4.2 \omega^{2} - 5.4 i \omega - 1.1}{1.0 \omega^{4} - 3.8 i \omega^{3} - 5.9 \omega^{2} + 4.2 i \omega + 1.1}\\\frac{- 4.2 i \omega - 1.1}{1.0 \omega^{4} - 3.8 i \omega^{3} - 5.9 \omega^{2} + 4.2 i \omega + 1.1} & \frac{- 1.0 i \omega^{3} - 3.8 \omega^{2} + 5.9 i \omega}{1.0 \omega^{4} - 3.8 i \omega^{3} - 5.9 \omega^{2} + 4.2 i \omega + 1.1} & \frac{4.2 \omega^{2} - 1.1 i \omega}{1.0 \omega^{4} - 3.8 i \omega^{3} - 5.9 \omega^{2} + 4.2 i \omega + 1.1}\\\frac{4.2 \omega^{2} - 1.1 i \omega}{1.0 \omega^{4} - 3.8 i \omega^{3} - 5.9 \omega^{2} + 4.2 i \omega + 1.1} & \frac{- 4.2 i \omega - 1.1}{1.0 \omega^{4} - 3.8 i \omega^{3} - 5.9 \omega^{2} + 4.2 i \omega + 1.1} & \frac{4.2 i \omega^{3} + 1.1 \omega^{2}}{1.0 \omega^{4} - 3.8 i \omega^{3} - 5.9 \omega^{2} + 4.2 i \omega + 1.1}\end{matrix}\right]\]
\[||W||_{H_2} = 3.9739213167564933\]
\[||W||_{H_\infty} = 5.4385479975362845 \]

  \begin{figure}[ht!]
    \centering
    \includegraphics[width=\textwidth]{src/figs/task4_1_amps.jpg}
    \caption{АЧХ системы.}
    \label{fig:task4_1_amps}
  \end{figure}
  
  \begin{figure}[ht!]
    \centering
    \includegraphics[width=\textwidth]{src/figs/task4_1_zs.jpg}
    \caption{Регулируемый выход системы.}
    \label{fig:task4_1_zs}
  \end{figure}
  
  \begin{figure}[ht!]
    \centering
    \includegraphics[width=\textwidth]{src/figs/task4_1_sing.jpg}
    \caption{Сингулярные числы.}
    \label{fig:task4_1_sing}
  \end{figure}

  \FloatBarrier

  \subsubsubsection{gamma = 10}
\[spec(A-B_2 K) = [-0.42  2.43]\]
\[K = \begin{bmatrix}
 -1.01 & -2.02
\end{bmatrix}\]
\[Q = \begin{bmatrix}
  1.01 &  1.01\\
  1.01 &  2.02
\end{bmatrix}\]
\[L = \begin{bmatrix}
 -1.89\\
 -1.16
\end{bmatrix}\]
\[ W = \left[\begin{matrix}\frac{- 1.0 i \omega^{3} - 3.8 \omega^{2} + 1.6 i \omega - 1.1}{1.0 \omega^{4} - 3.8 i \omega^{3} - 5.9 \omega^{2} + 4.2 i \omega + 1.1} & \frac{- i \omega^{3} - 4.8 \omega^{2} + 9.8 i \omega + 5.9}{1.0 \omega^{4} - 3.8 i \omega^{3} - 5.9 \omega^{2} + 4.2 i \omega + 1.1} & \frac{4.2 \omega^{2} - 5.4 i \omega - 1.1}{1.0 \omega^{4} - 3.8 i \omega^{3} - 5.9 \omega^{2} + 4.2 i \omega + 1.1}\\\frac{- 4.2 i \omega - 1.1}{1.0 \omega^{4} - 3.8 i \omega^{3} - 5.9 \omega^{2} + 4.2 i \omega + 1.1} & \frac{- 1.0 i \omega^{3} - 3.8 \omega^{2} + 5.9 i \omega}{1.0 \omega^{4} - 3.8 i \omega^{3} - 5.9 \omega^{2} + 4.2 i \omega + 1.1} & \frac{4.2 \omega^{2} - 1.1 i \omega}{1.0 \omega^{4} - 3.8 i \omega^{3} - 5.9 \omega^{2} + 4.2 i \omega + 1.1}\\\frac{4.2 \omega^{2} - 1.1 i \omega}{1.0 \omega^{4} - 3.8 i \omega^{3} - 5.9 \omega^{2} + 4.2 i \omega + 1.1} & \frac{- 4.2 i \omega - 1.1}{1.0 \omega^{4} - 3.8 i \omega^{3} - 5.9 \omega^{2} + 4.2 i \omega + 1.1} & \frac{4.2 i \omega^{3} + 1.1 \omega^{2}}{1.0 \omega^{4} - 3.8 i \omega^{3} - 5.9 \omega^{2} + 4.2 i \omega + 1.1}\end{matrix}\right] \]
\[||W||_{H_2} = 3.9739213167564933\]
\[||W||_{H_\infty} = 5.4385479975362845 \]

  \begin{figure}[ht!]
    \centering
    \includegraphics[width=\textwidth]{src/figs/task4_2_amps.jpg}
    \caption{АЧХ системы.}
    \label{fig:task4_2_amps}
  \end{figure}
  
  \begin{figure}[ht!]
    \centering
    \includegraphics[width=\textwidth]{src/figs/task4_2_zs.jpg}
    \caption{Регулируемый выход системы.}
    \label{fig:task4_2_zs}
  \end{figure}
  
  \begin{figure}[ht!]
    \centering
    \includegraphics[width=\textwidth]{src/figs/task4_2_sing.jpg}
    \caption{Сингулярные числы.}
    \label{fig:task4_2_sing}
  \end{figure}

  \FloatBarrier
\FloatBarrier

\newpage
\section{ГЛАВА 5. СТАБИЛИЗАЦИЯ МАЯТНИКА: LQR И ФИЛЬТР КАЛМАНА}

\subsection{Синтез линейно-квадратичного регулятора}
В этом задании выводится регулятор заданной степени устойчивости для системы:
\[
        \begin{cases}
                \text{Объект управления: }\dot{x} = A x + Bu \\
                \text{Регулятор: }u = -K x \\
        \end{cases} \rightarrow
        \dot{x} = A x - BKx = (A-BK)x
\]
LQR позволяет оптимизировать критерий качества:
\[J = \int_0^\infty (x^T Q x + u^T R u)dt \]
Выбор cотношения матриц \(Q\) и \(R\) позволяет управлять временем сходимости и величиной подаваего управления: чем больше \(\frac{Q}{R}\), тем больше управление и быстрее сходимость.

\(K\) получается решением следующих уравнений:
\[
\begin{cases}
    A^T P + P A + Q - PBR^{-1}B^TP = 0\\
    K = -R^{-1} B^T P \\
\end{cases}
\]

Теоретический минимум критерия качества:
\[J_{min} = x_0^T P x_0\]
\FloatBarrier

\subsection{Исследование линейно-квадратичного регулятора}
Все работает отлично, при небольшом отклонении от горизонтали в начальный момент.
\[Q = 0.1; R = 10.0; K_0 = \begin{bmatrix}
        -0.10 & -0.54 &  25.09 &  7.61
       \end{bmatrix}\]
       \[eig(A+BK_0) = \begin{bmatrix}
        -0.22 + 0.21j & -0.22 + -0.21j & -3.37 + 0.00j & -3.27 + 0.00j
       \end{bmatrix}\]
       \[Q = 1.0; R = 1.0; K_1 = \begin{bmatrix}
        -1.00 & -2.40 &  34.91 &  10.76
       \end{bmatrix}\]
       \[eig(A+BK_1) = \begin{bmatrix}
        -3.86 + 0.00j & -2.87 + 0.00j & -0.81 + 0.50j & -0.81 + -0.50j
       \end{bmatrix}\]
       \[Q = 10.0; R = 0.1; K_2 = \begin{bmatrix}
        -10.00 & -17.70 &  117.61 &  38.08
       \end{bmatrix}\]
       \[eig(A+BK_2) = \begin{bmatrix}
        -14.54 + 0.00j & -2.42 + 1.00j & -2.42 + -1.00j & -1.01 + 0.00j
       \end{bmatrix}\]

\begin{figure}[ht!]
        \centering
        \includegraphics[width=\textwidth]{src/figs/task5_states.jpg}
        \caption{Динамика компонент системы.}
        \label{fig:task5_states}
\end{figure}

\begin{figure}[ht!]
        \centering
        \includegraphics[width=\textwidth]{src/figs/task5_us.jpg}
        \caption{Управление.}
        \label{fig:task5_u}
\end{figure}
\FloatBarrier

\subsection{Синтез фильтра Калмана}
\[      
        \text{Объект управления: }
        \begin{cases}
                \dot{x} = A x + Bu + f, \text{ \(f\) -- внешнее возмущение}\\
                y = Cx + \xi, \text{ \(\xi\) -- помеха измерений}
        \end{cases} 
\]
Матрицы \(Q\) и \(R\) обозначают, насколько сильно мы оцениваем влиянию \(f\) и \(\xi\).

\(L\) получается решением следующих уравнений:
\[
\begin{cases}
    A P + P A^T + Q - PC^TR^{-1}CP = 0\\
    L = -P C^T R^{-1}\\
\end{cases}
\]
Ниже приведена ошибка слежения за системой из прошлог пункта.
\[Q = 0.01; R = 0.01; L = \begin{bmatrix}
        1.76 &  0.37\\
        1.11 &  1.88\\
        0.37 &  6.69\\
        1.26 &  21.97
      \end{bmatrix}\]
\begin{figure}[ht!]
        \centering
        \includegraphics[width=\textwidth]{src/figs/task5_3.jpg}
        \caption{Ошибка.}
        \label{fig:task5_3}
\end{figure}
\FloatBarrier

\subsection{ LQG для линейной модели}
В этом задании выводится наблюдатель регулятор для системы:
\[
        \begin{cases}
                \dot{x} = A x + B  K \hat{x} + f \\
                y = Cx + DK\hat{x} + \xi \\
                \dot{\hat{x}} = A \hat{x} + B  K \hat{x} + L(\hat{y} - y) \\
                \hat{y} = C \hat{x} + D K \hat{x} \\
                \hat{x} = x - e \\
        \end{cases} \rightarrow
        \begin{cases}
            \begin{bmatrix} 
                \dot{x} \\
                \dot{e}
            \end{bmatrix} = 
            A_{new}
            \begin{bmatrix} 
              x \\
              e
            \end{bmatrix} 
          + B_{new} 
          \begin{bmatrix} 
            f \\
            \xi
          \end{bmatrix} 
            \\
            A_{new} = 
            \begin{bmatrix} 
                A + BK & -BK\\
                0 & A + LC
            \end{bmatrix} \\
            B_{new} = 
            \begin{bmatrix} 
                I & 0\\
                I & L
            \end{bmatrix} \in R^{2n \times (n + m)}
         \end{cases}
\]
\[Q = 0.01; R = 0.01; L = \begin{bmatrix}
        -1.76 & -0.37\\
        -1.11 & -1.88\\
        -0.37 & -6.69\\
        -1.26 & -21.97
       \end{bmatrix}\]
       \[spec(A+LC) = \begin{bmatrix}
        -0.87 + 0.50j & -0.87 + -0.50j & -2.88 + 0.00j & -3.84 + 0.00j
       \end{bmatrix}\]
       \[Q = 1; R = 1; K= \begin{bmatrix}
         1.00 &  2.40 & -34.91 & -10.76
       \end{bmatrix}\]
       \[spec(A + BK) = \begin{bmatrix}
        -3.86 + 0.00j & -2.87 + 0.00j & -0.81 + 0.50j & -0.81 + -0.50j
       \end{bmatrix}\]
\begin{figure}[ht!]
        \centering
        \includegraphics[width=\textwidth]{src/figs/task5_LQG_lin.jpg}
        \caption{Динамика системы и ошибки.}
        \label{fig:task5_4}
\end{figure}
\FloatBarrier

\subsection{ LQG для нелинейной модели}
\[Q = 0.01; R = 0.01; L = \begin{bmatrix}
        1.76 &  0.37\\
        1.11 &  1.88\\
        0.37 &  6.69\\
        1.26 &  21.97
      \end{bmatrix}\]
      \[Q = 0.01; R = 0.01; L = \begin{bmatrix}
       -1.76 & -0.37\\
       -1.11 & -1.88\\
       -0.37 & -6.69\\
       -1.26 & -21.97
      \end{bmatrix}\]
      \[spec(A+LC) = \begin{bmatrix}
       -0.87 + 0.50j & -0.87 + -0.50j & -2.88 + 0.00j & -3.84 + 0.00j
      \end{bmatrix}\]
      \[Q = 1; R = 1; K= \begin{bmatrix}
        1.00 &  2.40 & -34.91 & -10.76
      \end{bmatrix}\]
      \[spec(A + BK) = \begin{bmatrix}
       -3.86 + 0.00j & -2.87 + 0.00j & -0.81 + 0.50j & -0.81 + -0.50j
      \end{bmatrix}\]
\begin{figure}[ht!]
        \centering
        \includegraphics[width=\textwidth]{src/figs/task5_LQG_non_lin.jpg}
        \caption{Динамика системы и ошибки.}
        \label{fig:task5_5}
\end{figure}
\FloatBarrier
\FloatBarrier

\newpage
\section{ГЛАВА 6. СЛЕЖЕНИЕ И КОМПЕНСАЦИЯ}

\subsection{Решение задачи компенсации}
Рассмотрим систему вида:
\[
    \begin{cases}
        \dot{x} = A_1x + B_1u + B_2w \\
        z = C_2x + D_2 w
    \end{cases},
\]
где $w$:
\[
    \dot{w} = A_2w
\]

Для данной системы можем синтезировать регулятор вида $u = K_1x + K_2w$, гарантирующий:
\begin{equation*}
    \lim_{t\to\infty} z(t) = 0
\end{equation*}

$K_1$ можем выбрать как матрицу регулятора, синтезированного любым способом. Матрицу $K_2$ найдем следующим образом:
\[
    \begin{cases}
        PA_2 - A_1P = B_1Y + B_2\\
        C_2P + D_2 = 0 \\
        K_2 = Y - K_1P
    \end{cases}
\]

\[
    \begin{cases}
        \dot{x} = A_1x + B_1 (K_1x + K_2w) + B_2w = (A_1 + B_1 K_1) x + (B_2 + B_1 K_2)w \\
        z = C_2x + D_2 w
    \end{cases},
\]

\[A_1 = \begin{bmatrix}
        0.00 &  1.00 &  0.00 &  0.00\\
        0.00 &  0.00 &  1.00 &  0.00\\
        0.00 &  0.00 &  0.00 &  1.00\\
        0.00 &  0.00 &  11.00 &  0.00
      \end{bmatrix}\]
      \[A_2 = \begin{bmatrix}
        0.00 &  0.10 &  0.00 &  0.00\\
       -0.10 &  0.00 &  0.00 &  0.00\\
        0.00 &  0.00 &  0.00 &  1.00\\
        0.00 &  0.00 & -1.00 &  0.00
      \end{bmatrix}\]
      \[B_1 = \begin{bmatrix}
        0.00\\
        1.00\\
        0.00\\
        1.00
      \end{bmatrix}\]
      \[B_2 = \begin{bmatrix}
        0.00 &  0.00 &  0.00 &  0.00\\
        0.33 &  0.67 &  1.00 &  1.33\\
        0.00 &  0.00 &  0.00 &  0.00\\
        3.67 &  7.33 &  11.00 &  14.67
      \end{bmatrix}\]
      \[C_2 = \begin{bmatrix}
        0.00 &  0.00 &  1.00 &  0.00
      \end{bmatrix}\]
      \[D_2 = \begin{bmatrix}
        0.00 &  0.00 &  0.00 &  0.00
      \end{bmatrix}\]
      \[C_1 = \begin{bmatrix}
        1.00 &  0.00 &  0.00 &  0.00\\
        0.00 &  0.00 &  1.00 &  0.00
      \end{bmatrix}\]
      \[D_1 = \begin{bmatrix}
        0.00 &  0.00 &  0.00 &  0.00\\
        0.00 &  0.00 &  0.00 &  0.00
      \end{bmatrix}\]
      \[K_1 = \begin{bmatrix}
        0.32 &  1.08 & -28.01 & -8.54
      \end{bmatrix}\]
      \[spec(A + B_1 K_1) = \begin{bmatrix}
       -3.47 & -3.17 & -0.41 + 0.35j & -0.41 -0.35j
      \end{bmatrix}\]
      \[K_2 = \begin{bmatrix}
       -36.79 & -254.29 &  0.29 & -29.73
      \end{bmatrix}\]
\begin{figure}[ht!]
        \centering
        \includegraphics[width=\textwidth]{src/figs/task6_1_states.jpg}
        \caption{Динамика системы.}
        \label{fig:task6_1_states}
\end{figure}

\begin{figure}[ht!]
        \centering
        \includegraphics[width=\textwidth]{src/figs/task6_1_z.jpg}
        \caption{Регулируемый выход системы.}
        \label{fig:task6_1_z}
\end{figure}

Как видно, для нелинейной системы регулятор не справляется с задачей. Ошибка угла колеблется около 0. Тем не менее, полученный результат можно использовать для хоть какой-то компенсации, так как поведение такой системы почти соответсвует желаемому.
\FloatBarrier

\subsection{Решение задачи слежения}
Рассмотрим систему вида:
\[
    \begin{cases}
        \dot{x} = A_1x + B_1u + B_2w \\
        z = C_2x + D_2w
    \end{cases},
\]
где $w$:
\[
    \dot{w} = A_2w
\]

Для данной системы можем синтезировать регулятор вида $u = K_1x + K_2w$, гарантирующий:
\begin{equation*}
    \lim_{t\to\infty} z(t) = 0
\end{equation*}

$K_1$ можем выбрать как матрицу регулятора, синтезированного любым способом. Матрицу $K_2$ найдем следующим образом:
\[
    \begin{cases}
        PA_2 - A_1P = B_1Y + B_2\\
        C_2P + D_2 = 0 \\
        K_2 = Y - K_1P
    \end{cases}
\]

\[
    \begin{cases}
        \dot{x} = A_1x + B_1 (K_1x + K_2w) + B_2w = (A_1 + B_1 K_1) x + (B_2 + B_1 K_2)w \\
        z = C_2x + D_2 w
    \end{cases},
\]
\[A_1 = \begin{bmatrix}
        0.00 &  1.00 &  0.00 &  0.00\\
        0.00 &  0.00 &  1.00 &  0.00\\
        0.00 &  0.00 &  0.00 &  1.00\\
        0.00 &  0.00 &  11.00 &  0.00
      \end{bmatrix}\]
      \[A_2 = \begin{bmatrix}
        0.00 &  0.10 &  0.00 &  0.00\\
       -0.10 &  0.00 &  0.00 &  0.00\\
        0.00 &  0.00 &  0.00 &  1.00\\
        0.00 &  0.00 & -1.00 &  0.00
      \end{bmatrix}\]
      \[B_1 = \begin{bmatrix}
        0.00\\
        1.00\\
        0.00\\
        1.00
      \end{bmatrix}\]
      \[B_1 = \begin{bmatrix}
        0.00 &  0.00 &  0.00 &  0.00\\
        0.00 &  0.00 &  0.00 &  0.00\\
        0.00 &  0.00 &  0.00 &  0.00\\
        0.00 &  0.00 &  0.00 &  0.00
      \end{bmatrix}\]
      \[C_2 = \begin{bmatrix}
        0.00 &  0.00 &  1.00 &  0.00
      \end{bmatrix}\]
      \[D_2 = \begin{bmatrix}
        0.10 &  0.10 &  0.10 &  0.10
      \end{bmatrix}\]
\begin{figure}[ht!]
        \centering
        \includegraphics[width=\textwidth]{src/figs/task6_2_states.jpg}
        \caption{Динамика системы.}
        \label{fig:task6_2_states}
\end{figure}

\begin{figure}[ht!]
        \centering
        \includegraphics[width=\textwidth]{src/figs/task6_2_target.jpg}
        \caption{Регулируемый выход системы.}
        \label{fig:task6_2_z}
\end{figure}
Как видно, задача слежения выполнена. Нелинейная система имела чуть большую ошибку вначале, но в итоге тоже свела ее к 0.
\FloatBarrier
\FloatBarrier

\newpage
\newpage
\section{Заключение}
В этой работе прошло применение всего, что было изученно за полугодние, на нелинейной системе.
\subsection{Выводы}
\begin{enumerate}
   \item LQR показал себя стабильнее всех 
   \item методы управления линейными системами можно применять для нелинейных, если они близки к точке равновесия в пространстве состояний, иначе линеаризация ломается и система становится неустойчивой.
\end{enumerate}

\end{document}