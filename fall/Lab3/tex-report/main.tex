\documentclass[16pt]{article}

\usepackage{report}

\usepackage[utf8]{inputenc} % allow utf-8 input
\usepackage[english, russian]{babel}
\usepackage[T1]{fontenc}    % use 8-bit T1 fonts
\usepackage[colorlinks=true, linkcolor=black, citecolor=blue, urlcolor=blue]{hyperref}       % hyperlinks
\usepackage{url}            
\usepackage{booktabs}       
\usepackage{amsfonts}       
\usepackage{nicefrac}      
\usepackage{microtype} 
\usepackage{graphicx}
\usepackage{natbib}
\usepackage{mathrsfs}
\usepackage{doi}
\usepackage{mathtools}
\usepackage{graphicx}
\usepackage{listings}
\usepackage{pythonhighlight}
\usepackage{mathtools}
\graphicspath{ {./figs/} }


\setcitestyle{aysep={,}}



\title{ЛР \textnumero 3 <<Вынужденное движение>>}

\author{
Студент \\
Кирилл Лалаянц\\
R33352\\
336700\\
Вариант - 6\\
\\
Преподаватель\\
Пашенко А.В. \\
\\
\\
Факультет Систем Управления и Робототехники\\
\\
ИТМО\\
}

% Uncomment to remove the date
\date{13.09.2023}

% Uncomment to override  the `A preprint' in the header
\renewcommand{\headeright}{ЛР \textnumero 3 <<Вынужденное движение>>}
\renewcommand{\undertitle}{Отчет}
\renewcommand{\shorttitle}{}


\begin{document}
\maketitle
\newpage
\tableofcontents
\thispagestyle{empty}

\newpage
\setcounter{page}{1}
\section{Вводные данные}
\subsection{Цель работы}
В этой работе будет проведенно исследование следующих вопросов:
\begin{itemize}
    \item Вынужденное движение системы в зависимости от корней ее характеристического уравнения. 
    \item Анализ зависимости качества переходных процессов от корней характеристического уравнения системы третьего порядка. 
    \item Проверка свойства свертки как произведения Лапласа. 
\end{itemize} 

\newpage
\section{Выполнение работы}
\label{sec:headings}


\subsection{Задание 1. Свободное движение.}

\subsubsection{Теория}
Хотим получить ДУ системы вида:
\[\ddot{y} + a_1\dot{y} + a_0y = u\]
Cначала получим характеристическое уравнение для частного решения вида:
\[\ddot{y} + a_1\dot{y} + a_0y = 0\]
Имеем два корня -- \(\lambda_1\) и \(\lambda_2\). Тогда:
\[ (p - \lambda_1)(p-\lambda_2) = p^2 - (\lambda_1 + \lambda_2)p + \lambda_1\lambda_2 = 0,\]
умножив которое на \(y\) получаем:
\[\ddot{y} - (\lambda_1 + \lambda_2)\dot{y} + \lambda_1\lambda_2y = 0.\]

Для данного ДУ известно частичное решение:
\[ y_{0}(t) = c_1 e^{\lambda_1 t} + c_2 e^{\lambda_2 t},\]
параметры \(c_1\) и \(c_2\) которого вычисляются из начальных условий.
\\
Также подставим характеристическое уравнение в знаменатель TF \(W(p)\):
\[W(p) = \frac{1}{p^2 - (\lambda_1 + \lambda_2)p + \lambda_1\lambda_2}\]
Полученная TF соотвествует системе с желаемыми корнями характеристического уравнения.

\subsubsection{Программная реализация}
\begin{python}
def task1_output(m1, m2, initial_values, us, ts, plot_name, save_name):
    poly = sympy.simplify((p - m1) * (p - m2))
    coeffs = sympy.Poly(poly, p).all_coeffs()
    ss = control.tf2ss(control.tf(1, np.array(coeffs, dtype=np.float64)))
    ss_reachable = control.canonical_form(ss, form="reachable")[0]
    responses = {}
    for key_u, u in us.items():
        responses[key_u] = []
        for initial_value in initial_values:
            response = control.forced_response(ss_reachable, U=u(ts), X0=initial_value, T=ts)
            responses[key_u].append((initial_value, response))
    
    plot_task1(responses, ts, plot_name, save_name)
\end{python}
\subsubsection{Результаты}
Сначала было проведено исследования системы с двумя вещественными корнями.
\begin{figure}[h]
    \centering
    \includegraphics[width=\textwidth]{task1_1}
    \caption{Результат двух устойчивых апереодических мод.}
    \label{fig:fig1}
\end{figure}


По графигам на рисунке 1 видно, что устойчивая система при подаче константного воздействия -- стабилизируется; при неограниченно растущем -- уходит в бесконечность; при периодическом -- коллеблется.

Исследование системы с нейтральным и устойчивым корнем дало следующие результаты.
\begin{figure}[h]
    \centering
    \includegraphics[width=\textwidth]{task1_3}
    \caption{Результат нейтральной и устойчивой апереодических мод.}
    \label{fig:fig2}
\end{figure}
Видно, что системы стабильна только при переодическом воздействии. Такое поведение легко объясняется решением ДУ аналитически.

Последняя исследованная система имела два неустойчивых колебательных корня.
\begin{figure}[h]
    \centering
    \includegraphics[width=\textwidth]{task1_8}
    \caption{Результат двух неустойчивых колебательных мод.}
    \label{fig:fig3}
\end{figure}
Результат, ожидаемо, крайне неустойчив.

\pagebreak
\subsection{Задание 2. Качество переходных процессов.}
\subsubsection{Теория}
В этом задании TF получается аналогично первому заданию. После этого определяется два параметра для оценки переходного процесса:
\itemize
\item \(T_{5\%} : \forall t > T_{5\%} \hookrightarrow \frac{|y(t) - y(T_{\infty})|}{y(T_{\infty})} < 0.05\) -- время переходного процесса;
\item \(\Delta\sigma = |\frac{y_{max} - y_\infty}{y_\infty}|\) -- перерегулирование;


\subsubsection{Программная реализация}
\begin{python}
    y = sympy.Function("y")
t = sympy.Symbol("t")

def get_ss_reachable(modes):
    poly = 1
    for mode in modes:
        poly = sympy.simplify(poly * (p - mode))
    coeffs = sympy.Poly(poly, p).all_coeffs()
    
    ss = control.tf2ss(control.tf(1, np.array(coeffs, dtype=np.float64)))
    ss_reachable = control.canonical_form(ss, form="reachable")[0]
    return ss_reachable


def get_state_limit(ss_reachable, T0, initial_values_T0):
    params = np.concatenate((ss_reachable.A[0, :], ss_reachable.B[0, :]))
    a2, a1, a0, b0 = map(float, -params)
    b0 = -b0
    
    ics={y(T0): initial_values_T0[0], y(t).diff(t, 1).subs(t, T0): initial_values_T0[1], y(t).diff(t, 2).subs(t, T0): initial_values_T0[2]}
    ode_sympy = sympy.dsolve(y(t).diff(t, 3) + a2 * y(t).diff(t, 2) + a1 * y(t).diff(t, 1) + a0*y(t) - 1, X0=0, ics=ics)
    time_of_limit = 10**10
    state_limit = ode_sympy.subs(t, time_of_limit).rhs
    while abs(1 - state_limit/ode_sympy.subs(t, time_of_limit * 10).rhs) > 0.001:
        time_of_limit *= 10
        state_limit = ode_sympy.subs(t, time_of_limit).rhs
    return state_limit

def model_until_5percent_band(ss_reachable, initial_values, state_limit):
    percent5_interval = state_limit * 0.05
    t_max = 10
    while True:
        ts = get_t(t_max)
        response = control.forced_response(ss_reachable, U=1, X0=initial_values[::-1], T=ts)

        t_5_percent = 0
        for i in range(len(ts)):
            if abs(response.outputs[i] - state_limit) > percent5_interval:
                t_5_percent = ts[i]
                t_5_percent_i = i
                y_5_percent = response.outputs[i]
        if t_5_percent <= t_max*0.8:
            ts = get_t(t_max)

            response = control.forced_response(ss_reachable, U=1, X0=initial_values[::-1], T=ts)
            
            return (ts, response.outputs, t_5_percent, y_5_percent, t_5_percent_i * 2)
        t_max *= 1.5
    
def get_overshooting(response_outputs, ts, state_limit):
    overshooting_values = response_outputs - state_limit
    if overshooting_values[0] > 0:
        overshooting_values *= -1
    overshooting_counter = 0
    while overshooting_values[overshooting_counter] < overshooting_values[overshooting_counter + 1]:
        overshooting_counter += 1
        if overshooting_counter >= len(overshooting_values) - 1:
            break
    
    absolute_overshooting = overshooting_values[overshooting_counter]
    relative_overshooting = overshooting_values[overshooting_counter] / state_limit
    y_overshooting = response_outputs[overshooting_counter]
    t_overshooting = ts[overshooting_counter]
    return (t_overshooting, y_overshooting, relative_overshooting, absolute_overshooting)
    
    

def task2_analyse(modes, T0 = 0, initial_values_T0 = [0, 0, 0]):
    results = {}
    for title, mode in modes.items():
        ss_reachable = get_ss_reachable(mode)
        ss_limit = get_state_limit(ss_reachable, T0, initial_values_T0)
        ts, ss_y, t_5p, y_5p, _ = model_until_5percent_band(ss_reachable, initial_values_T0, ss_limit)
        t_os, y_os, os_rel, _ = get_overshooting(ss_y, ts, ss_limit)
        results[title] = [mode, ts, ss_y, ss_limit, t_5p, y_5p, t_os, y_os, os_rel]
    return results
\end{python}
\pagebreak
\subsubsection{Результаты}
\begin{figure}[h!]
    \centering
    \includegraphics[width=\textwidth]{task2_diff_real_with_I}
    \caption{Зависимость от мнимой части комплексных корней.}
    \label{fig:fig5}
\end{figure}
Сначала было проведенно исследование зависимости от мнимой части комплексных корней (рис. \ref{fig:fig5}).
Можно заметить, что при ее уменьшении, время переходного процесса и перерегулирование падают.

Затем было проведено исследование зависимости от вещественной части мнимых корней (рис. \ref{fig:fig4}). Результат ожидаем -- чем меньше вещественная часть, тем быстрее сходимость и меньше перерегулирование.
\begin{figure}[h!]
    \centering
    \includegraphics[width=\textwidth]{task2_diff_I}
    \caption{Зависимость от вещественной части комплексных корней.}
    \label{fig:fig4}
\end{figure}

Последний эксперимент был нацелен на исследование чисто вещественных корней. 
Результат (рис. \ref{fig:fig6}) вновь оказался предсказуем: чем меньше вещественная часть -- тем быстрее.

В целом можно заметить, что время переходного процесса и величина перерегулирования во многом зависят от самого близкого к нулю по вещественной оси корня. Все остальные корни оказывают меньшее влияние.
\begin{figure}[h!]
    \centering
    \includegraphics[width=\textwidth]{task2_diff_real_no_I}
    \caption{Зависимость от вещественных корней.}
    \label{fig:fig6}
\end{figure}

\pagebreak
\subsection{Задание 3. (Необязательное) Свертка, как произведение образов Лапласа.}
\subsubsection{Теория}
В этом задании проверим свойство свертки как преобразования Лапласа. Имеем систему:
\[
    W(s) = \frac{6}{(s+2)^4}, u(t) = 1.5 + 0.6t + sin(6t)
\]
Далее можем работать с этой системой тремя способами:
\begin{enumerate}
\item \(y_{i.r.} = \mathscr{L}^-1\{W\} \hookrightarrow y = (y_{i.r.} * u)(t)\) -- по свойству свертки;
\item \(y(t) = W[u(t)]\) -- классическое моделирование системы через TF;
\item \(U = \mathscr{L}\{u(t)\} \hookrightarrow y(t) = (WU)[\delta(0)]\) -- по свойству перемножения образов Лапласа;
\end{enumerate}


\subsubsection{Программная реализация}
\begin{python}
dt = 0.001
ts = get_t(10, dt=0.001)
u_f = lambda t: 4 * np.cos(2*t) +  0.5 * t
u_sympy = 4 * sympy.cos(2*t) + 0.5 * t
u_lambda = sympy.lambdify(t, u_sympy, 'numpy')
U_sympy = sympy.laplace_transform(u_sympy, t, s)[0]
W_s_denum_coeffs = sympy.Poly((s+2)**4, s).all_coeffs()
W_s_denum_coeffs = list(map(float, W_s_denum_coeffs))
W_sympy = 6 / sympy.Poly((s+2)**4, s)
W_control = control.tf([6], W_s_denum_coeffs)
#  Convolutin
y_ir_sympy = sympy.inverse_laplace_transform(W_sympy, s, t)
y_ir_lambda = sympy.lambdify(t, y_ir_sympy, 'numpy')
y_full_convolution = np.convolve(y_ir_lambda(ts), u_f(ts)) * dt

#  Forced response
y_2 = control.forced_response(W_control, T=ts, U=u_lambda(ts)).outputs

#  Multiplication of Laplace images
U_sympy = U_sympy.simplify()
Y_sympy = U_sympy * W_sympy 
TF = Y_sympy.simplify()
num_3 = list(map(float,sympy.Poly(sympy.fraction(TF)[0],s).all_coeffs()))
denum_3 = list(map(float,sympy.Poly(sympy.fraction(TF)[1],s).all_coeffs()))
tf_3_2=control.tf(num_3, denum_3)
y_3 = control.impulse_response(tf_3_2, T=ts).outputs
\end{python}

\subsubsection{Результаты}
Как видно по рисунку~\ref{fig:fig7}, результаты совпали. В очередной раз не получилось подловить математиков на обмане.
\begin{figure}[h]
    \centering
    \includegraphics[width=\textwidth]{task3}
    \caption{Сравнение различных способов.}
    \label{fig:fig7}
\end{figure}

\newpage
\section{Заключение}
В этой работе было проведенно исследование следующих вопросов:
\begin{itemize}
    \item Вынужденное движение системы в зависимости от корней ее характеристического уравнения. 
    \item Анализ зависимости качества переходных процессов от корней характеристического уравнения системы третьего порядка. 
    \item Проверка свойства свертки оригиналов преобразования Лапласа. 
\end{itemize} 
\subsection{Выводы}
\begin{enumerate}
   \item Проведено моделирование вынужденного движение систем с различным ненулевым входным воздействием.
   \item На практике проверенно влияние мод на характер поведения системы.
   \item Наглядно проверенно, что свертка оригиналов равносильна перемножению образов Лапласа.
\end{enumerate}

\end{document}