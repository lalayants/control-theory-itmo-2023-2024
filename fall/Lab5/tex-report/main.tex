\documentclass[16pt]{article}

\usepackage{report}

\usepackage[utf8]{inputenc} % allow utf-8 input
\usepackage[english, russian]{babel}
\usepackage[T1]{fontenc}    % use 8-bit T1 fonts
\usepackage[colorlinks=true, linkcolor=black, citecolor=blue, urlcolor=blue]{hyperref}       % hyperlinks
\usepackage{url}            
\usepackage{booktabs}       
\usepackage{amsfonts}       
\usepackage{nicefrac}      
\usepackage{microtype} 
\usepackage{graphicx}
\usepackage{natbib}
\usepackage{mathrsfs}
\usepackage{doi}
\usepackage{mathtools}
\usepackage{graphicx}
\usepackage{listings}
\usepackage{pythonhighlight}
\usepackage{mathtools}
\usepackage{amsmath}
\graphicspath{ {./figs/} }


\setcitestyle{aysep={,}}



\title{ЛР \textnumero 5 <<Типовые динамические звенья>>}

\author{
Студент \\
Кирилл Лалаянц\\
R33352\\
336700\\
Вариант - 6\\
\\
Преподаватель\\
Пашенко А.В. \\
\\
\\
Факультет Систем Управления и Робототехники\\
\\
ИТМО\\
}

% Uncomment to remove the date
\date{13.11.2023}

% Uncomment to override  the `A preprint' in the header
\renewcommand{\headeright}{ЛР \textnumero 5 <<Типовые динамические звенья>>}
\renewcommand{\undertitle}{Отчет}
\renewcommand{\shorttitle}{}


\begin{document}
\maketitle
\newpage
\tableofcontents
\thispagestyle{empty}

\newpage
\setcounter{page}{1}
\section{Вводные данные}
\subsection{Цель работы}
В этой работе будет проведенно исследование следующих вопросов:
\begin{itemize}
    \item Типовые динамические звенья.
    \item АЧХ, ЛЧХ, ФЛЧХ.
\end{itemize} 

\subsection{Воспроизведение результатов}
Все результаты можно воспроизвести с помощью \href{https://github.com/lalayants/control-theory-itmo-2023}{репозитория}.

\subsection{Обозначения}
Здесь и далее \(\theta(t)\) -- функция Хэвисайда.

\newpage
\section{Выполнение работы}
\label{sec:headings}


\subsection{Brushed DC motor 2.0.}

\subsubsection{Теория}
Даны уравнения двигателя постоянного тока независимого возбуждения:
\[J\dot{w} = M, M = k_mI, I = \frac{U + \varepsilon}{R}, \varepsilon = \varepsilon_i + \varepsilon_s, \varepsilon_i = -k_ew, \varepsilon_s=-L\dot{I}.\]

Путем несложных преобразований получаем:
\begin{align}
    \begin{bmatrix}
        \dot{w} \\
        \dot{M}
    \end{bmatrix}
    = 
    \begin{bmatrix}
        0 & \frac{1}{ J} \\
        -\frac{k_mk_e}{ L} & -\frac{R}{L}
    \end{bmatrix}
    \begin{bmatrix}
        w \\
        M
    \end{bmatrix}
    + 
    \begin{bmatrix}
        0 \\
        \frac{k_m}{L}
    \end{bmatrix}
    U
\end{align}
\begin{align}
    w 
    = 
    \begin{bmatrix}
        1 & 0
    \end{bmatrix}
    \begin{bmatrix}
        w \\
        M
    \end{bmatrix}
    + 
    \begin{bmatrix}
        0
    \end{bmatrix}
    U
\end{align}
Откуда получаем TF:
\[
    W(s) = \frac{k_m}{JLs^2 + JRs + k_ek_m}
\]
Тогда ДУ:
\[
    JL\ddot{w} + JR\dot{w} + k_ek_mw = k_mU
\]
\[
    \ddot{w} + \frac{R}{L}\dot{w} + \frac{k_ek_m}{JL}w = \frac{k_m}{JL}U
\]
\[
    \ddot{w} + 3.908\dot{w} + 48.81w = 134.2U
\]
\[
    3.908^2 - 4*48.81 < 0 \Rightarrow \text{колебательное звено второго порядка}.
\]
Для частотных характеристик подставим в ТF \(s = iw\):
\[
    W(w) = \frac{k_m}{-JLw^2 + JRiw + k_ek_m}
\]

Применив обратное преобразование Лапласса для \(W\) получаем:
\[w_{i.r.} = \frac{2 k_{m} e^{- \frac{R t}{2 L}} \sin{\left(\frac{t \sqrt{\frac{- \frac{R^{2}}{L} + \frac{4 k_{e} k_{m}}{J}}{L}}}{2} \right)} \theta\left(t\right)}{J L \sqrt{\frac{- \frac{R^{2}}{L} + \frac{4 k_{e} k_{m}}{J}}{L}}}\]
\[w_{i.r.} = 20 e^{- 1.95 t} \sin{\left(6.7 t \right)} \theta\left(t\right)\]
Применив обратное преобразование Лапласса для \(W / s\) получаем:
\[w_{s.r.} = k_{m} \left(\left(- \frac{e^{- \frac{R t}{2 L}} \cos{\left(\frac{t \sqrt{- \frac{J R^{2} - 4 L k_{e} k_{m}}{J}}}{2 L} \right)}}{k_{e} k_{m}} - \frac{R e^{- \frac{R t}{2 L}} \sin{\left(\frac{t \sqrt{\frac{- \frac{R^{2}}{L} + \frac{4 k_{e} k_{m}}{J}}{L}}}{2} \right)}}{L k_{e} k_{m} \sqrt{\frac{- \frac{R^{2}}{L} + \frac{4 k_{e} k_{m}}{J}}{L}}}\right) \theta\left(t\right) + \frac{\theta\left(t\right)}{k_{e} k_{m}}\right)\]
\[w_{s.r.} = 0.36 \left(- 2.2 e^{- 1.95 t} \sin{\left(6.7 t \right)} - 7.56 e^{- 1.95 t} \cos{\left(6.7 t \right)}\right) \theta\left(t\right) + 2.75 \theta\left(t\right)\]
\subsubsection{Результаты}
На графике (Рис. \ref*{fig:fig1}) представлен результат выполнения задания. Как видно, практический результат совпал с теоретическим.
\begin{figure}[h!]
    \centering
    \includegraphics[width=\textwidth]{1_2}
    \caption{Результат выполнения задания.}
    \label{fig:fig1}
\end{figure}

\subsection{Конденсируй. Интегрируй. Умножай.}

\subsubsection{Теория}
Дано уравнение конденсатора:
\[I = C\frac{dU}{dt}.\]

Путем несложных преобразований получаем:
\begin{align}
    \begin{bmatrix}
        \dot{U}
    \end{bmatrix}
    = 
    \begin{bmatrix}
        0 
    \end{bmatrix}
    \begin{bmatrix}
        U
    \end{bmatrix}
    + 
    \begin{bmatrix}
        \frac{1}{C}
    \end{bmatrix}
    I
\end{align}
\begin{align}
    U
    = 
    \begin{bmatrix}
        1
    \end{bmatrix}
    \begin{bmatrix}
        U
    \end{bmatrix}
    + 
    \begin{bmatrix}
        0
    \end{bmatrix}
    I
\end{align}
Откуда получаем TF:
\[
    W(s) = \frac{1}{C s}
\]
Тогда ДУ:
\[
    C \dot{U} = I
\]
\[
    \dot{U} = \frac{I}{C}
\]
\[
    \dot{U} = 3484I \Rightarrow \text{идеальное интегрирующее звено}.
\]
Для частотных характеристик подставим в ТF \(s = iw\):
\[
    W(w) = \frac{1}{C iw}
\]

Применив обратное преобразование Лапласса для \(W\) получаем:
\[U_{i.r.} = \frac{\theta\left(t\right)}{C}\]
\[U_{i.r.} = 3484.3 \theta\left(t\right)\]
Применив обратное преобразование Лапласса для \(W / s\) получаем:
\[U_{s.r.} = \frac{t \theta\left(t\right)}{C}\]
\[U_{s.r.} = 3484.3 t \theta\left(t\right)\]
\subsubsection{Результаты}
На графике (Рис. \ref*{fig:fig2}) представлен результат выполнения задания. Как видно, практический результат совпал с теоретическим.
\begin{figure}[h!]
    \centering
    \includegraphics[width=\textwidth]{1_3}
    \caption{Результат выполнения задания.}
    \label{fig:fig2}
\end{figure}

\newpage
\subsection{ Доп. Brushed DC motor.}

\subsubsection{Теория}
Даны уравнения двигателя постоянного тока независимого возбуждения:
\[J\dot{w} = M, M = k_mI, I = \frac{U + \varepsilon}{R}, \varepsilon = \varepsilon_i, \varepsilon_i = -k_ew.\]

Путем несложных преобразований получаем:
\begin{align}
    \begin{bmatrix}
        \dot{w}
    \end{bmatrix}
    = 
    \begin{bmatrix}
        -\frac{k_mk_e}{JR} 
    \end{bmatrix}
    \begin{bmatrix}
        w
    \end{bmatrix}
    + 
    \begin{bmatrix}
        \frac{k_m}{JR}
    \end{bmatrix}
    U
\end{align}
\begin{align}
    w 
    = 
    \begin{bmatrix}
        1
    \end{bmatrix}
    \begin{bmatrix}
        w
    \end{bmatrix}
    + 
    \begin{bmatrix}
        0
    \end{bmatrix}
    U
\end{align}
Откуда получаем TF:
\[
    W(s) = \frac{k_{m}}{J R \left(s + \frac{k_{e} k_{m}}{J R}\right)}
\]
Тогда ДУ:
\[
    JR\dot{w} + k_ek_mw = k_mU
\]
\[
    \dot{w} + \frac{k_ek_m}{JR}w = \frac{k_m}{JR}U
\]
\[
    \dot{w} + 8.909w = 24.67U \Rightarrow \text{апериодическое звено первого порядка}.
\]

Для частотных характеристик подставим в ТF \(s = iw\):
\[
    W(w) = \frac{k_{m}}{J R \left(iw + \frac{k_{e} k_{m}}{J R}\right)}
\]

Применив обратное преобразование Лапласса для \(W\) получаем:
\[w_{i.r.} = \frac{k_{m} e^{- \frac{k_{e} k_{m} t}{J R}} \theta\left(t\right)}{J R}\]
\[w_{i.r.} = 24.6 e^{- 8.9 t} \theta\left(t\right)\]
Применив обратное преобразование Лапласса для \(W / s\) получаем:
\[w_{s.r.} = \frac{k_{m} \left(\frac{J R \theta\left(t\right)}{k_{e} k_{m}} - \frac{J R e^{- \frac{k_{e} k_{m} t}{J R}} \theta\left(t\right)}{k_{e} k_{m}}\right)}{J R}\]
\[w_{s.r.} = 2.7 \theta\left(t\right) - 2.7 e^{- 8.9 t} \theta\left(t\right)\]
\subsubsection{Результаты}
На графике (Рис. \ref*{fig:fig3}) представлен результат выполнения задания. Как видно, практический результат совпал с теоретическим.
\begin{figure}[h!]
    \centering
    \includegraphics[width=\textwidth]{2_1}
    \caption{Результат выполнения задания.}
    \label{fig:fig3}
\end{figure}


\newpage
\subsection{ Доп. Tachogenerator.}

\subsubsection{Теория}
Даны уравнения тахогенератора постоянного тока:
\[I = \frac{\varepsilon - U_{out}}{R}, \varepsilon = \varepsilon_i + \varepsilon_s, \varepsilon_i = -k_e\dot{\theta}, e_s = -L\dot{I}, I = \frac{U_{out}}{r}.\]

Путем несложных преобразований получаем:
\begin{align}
    \begin{bmatrix}
        \dot{U}_{out}
    \end{bmatrix}
    = 
    \begin{bmatrix}
        \frac{R_L + R}{L}
    \end{bmatrix}
    \begin{bmatrix}
        U_{out}
    \end{bmatrix}
    + 
    \begin{bmatrix}
        k_e
    \end{bmatrix}
    w
\end{align}
\begin{align}
    U_{out} 
    = 
    \begin{bmatrix}
        1
    \end{bmatrix}
    \begin{bmatrix}
        U_{out}
    \end{bmatrix}
    + 
    \begin{bmatrix}
        0
    \end{bmatrix}
    w
\end{align}
Откуда получаем TF:
\[
    W(s) = \frac{k_{e}}{s + \frac{R}{L} + \frac{R_{l}}{L}}
\]
Тогда ДУ:
\[
    \dot{U}_{out} + \frac{R_L + R}{L}U_{out} = k_mw
\]
\[
    \dot{U}_{out} + 586U_{out} = 0.3427w \Rightarrow \text{апериодическое звено первого порядка}.
\]

Для частотных характеристик подставим в ТF \(s = iw\):
\[
    W(w) = \frac{k_{e}}{iw + \frac{R}{L} + \frac{R_{l}}{L}}
\]

Применив обратное преобразование Лапласса для \(W\) получаем:
\[w_{i.r.} = k_{e} e^{- \frac{t \left(R + R_{l}\right)}{L}} \theta\left(t\right)\]
\[w_{i.r.} = 0.3427 e^{- 586 t} \theta\left(t\right)\]
Применив обратное преобразование Лапласса для \(W / s\) получаем:
\[w_{s.r.} = k_{e} \left(- \frac{L^{2} e^{- \frac{t \left(R^{2} + 2 R R_{l} + R_{l}^{2}\right)}{L R + L R_{l}}} \theta\left(t\right)}{L R + L R_{l}} + \frac{L \theta\left(t\right)}{R + R_{l}}\right)\]
\[w_{s.r.} = 0.00058 \theta\left(t\right) - 0.00058 e^{- 586 t} \theta\left(t\right)\]
\subsubsection{Результаты}
На графике (Рис. \ref*{fig:fig4}) представлен результат выполнения задания. Как видно, практический результат совпал с теоретическим.
\begin{figure}[h!]
    \centering
    \includegraphics[width=\textwidth]{2_4}
    \caption{Результат выполнения задания.}
    \label{fig:fig4}
\end{figure}


\newpage
\subsection{Доп. Spring-mass system.}

\subsubsection{Теория}
Даны уравнения пружинного маятника:
\[F_\text{упр} = -kx, F = F_{ext} + F_\text{упр}, F = ma.\]

Путем несложных преобразований получаем:
\begin{align}
    \begin{bmatrix}
        \dot{x} \\
        \ddot{x}
    \end{bmatrix}
    = 
    \begin{bmatrix}
        0 & 1\\
        -\frac{k}{m} & 0
    \end{bmatrix}
    \begin{bmatrix}
        x \\
        \dot{x}
    \end{bmatrix}
    + 
    \begin{bmatrix}
        0 \\
        \frac{1}{m}
    \end{bmatrix}
    F_{ext}
\end{align}
\begin{align}
    x
    = 
    \begin{bmatrix}
        1 & 0
    \end{bmatrix}
    \begin{bmatrix}
        x \\
        \dot{x}
    \end{bmatrix}
    + 
    \begin{bmatrix}
        0
    \end{bmatrix}
    F_{ext}
\end{align}
Откуда получаем TF:
\[
    W(s) = \frac{1}{m \left(\frac{k}{m} + s^{2}\right)}
\]
Тогда ДУ:
\[
    m\ddot{x} + kx = F_{ext}
\]
\[
    \ddot{x} + \frac{k}{m}x = \frac{1}{m}F_{ext}
\]
\[
    \ddot{x} + 9.257x = 0.02857U \Rightarrow \text{консервативное звено второго порядка}.
\]

Для частотных характеристик подставим в ТF \(s = iw\):
\[
    W(w) = \frac{1}{m \left(\frac{k}{m} - w^{2}\right)}
\]

Применив обратное преобразование Лапласса для \(W\) получаем:
\[w_{i.r.} = \frac{\sin{\left(t \sqrt{\frac{k}{m}} \right)} \theta\left(t\right)}{m \sqrt{\frac{k}{m}}}\]
\[w_{i.r.} = \frac{\sqrt{35} \sin{\left(\frac{18 \sqrt{35} t}{35} \right)} \theta\left(t\right)}{630}\]
Применив обратное преобразование Лапласса для \(W / s\) получаем:
\[w_{s.r.} = \frac{- \frac{m \cos{\left(t \sqrt{\frac{k}{m}} \right)} \theta\left(t\right)}{k} + \frac{m \theta\left(t\right)}{k}}{m}\]
\[w_{s.r.} =  - \frac{\cos{\left(\frac{18 \sqrt{35} t}{35} \right)} \theta\left(t\right)}{324} + \frac{\theta\left(t\right)}{324}\]
\subsubsection{Результаты}
На графике (Рис. \ref*{fig:fig5}) представлен результат выполнения задания. Как видно, практический результат совпал с теоретическим. Разница в ФЧХ обусловлена работой функции atan2: ее результат принадлежит полуинтервалу \((-\pi, \pi]\), из-за чего отставание по фазе на \(\pi\) интерпретируется как опережение на \(\pi\).
\begin{figure}[h!]
    \centering
    \includegraphics[width=\textwidth]{2_5}
    \caption{Результат выполнения задания.}
    \label{fig:fig5}
\end{figure}










\newpage
\section{Заключение}
В этой работе было проведенно исследование следующих вопросов:
\begin{itemize}
    \item Типовые динамические звенья.
    \item АЧХ, ЛЧХ, ФЛЧХ.
\end{itemize} 
\subsection{Выводы}
\begin{enumerate}
   \item На примерах разных систем изучены типовые динамические звенья. Теоретические результаты сошлись с практическими.
\end{enumerate}

\end{document}